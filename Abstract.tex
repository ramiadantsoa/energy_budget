Ambient temperature strongly affects the performance, e.g. fecundity, of an individual in that environment.
Empirical studies--at least for ectotherms--have reported that performance first increases but eventually decreases sharply as temperature increases. Various models have been developed to explain such hump-shaped pattern but none have explored how thermal performance varies across individuals.
In this work, we use body size as fundamental intrinsic trait and ask whether an increase in body size broadens, shrinks or shift thermal performance curve. The model uses energy as a fundamental currency and combined on ecological (foraging), physiological (metabolism) and thermodynamical (warm-up) processes. We found that there is no single expected relationship as how performance breadth changes with body size.
The outcome depends on the allometric scaling between foraging rate and metabolic rate but also the quality and quantity of the environment.
We also emphasized the importance of the timing of warm-up and the conductance between the outer part of the body and thorax in defining diel activity and ultimately the performance of ectothermic and endothermic insects.
We suggest that empirical studies should pay more attention to the relationship between foraging rate and body size, and the conductance between the thorax and the external environment.

???Although the model here is simple, it lays foundation in building a generalized fitness landscape???
