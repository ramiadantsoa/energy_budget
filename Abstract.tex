Empirical studies--at least for ectotherms--have reported that fitness first increases but eventually decreases as temperature increases.
Although body mass is a key intrinsic factor that influences fitness, its role in shaping such hump-shaped pattern remains an open question.
In this work we ask whether the performance as a function of temperature (thermal performance curve) broadens, shrinks or shifts when body mass increases.
We build a model that integrates ecological (foraging), physiological (metabolism) and thermodynamical (warm-up) processes and asks how their interplay shapes the daily net energy as a proxy for performance.
We found that there is no single expected relationship on how thermal performance curve changes with body mass.
The upper limit of the thermal performance can shrink or broaden depending on the concavity or convexity of the foraging rate as a function of body mass and the quantity and quality of resource available.
The lower limit of the thermal performance is chiefly determined by the ability to warm-up.
Warm-up ability increases for endotherms and decreases for ectotherms but wind can become a relevant factor for ectotherms such that warm-up ability peaks at intermediate body mass.
To assess which qualitative result applies to a given system, we suggest that empirical studies should pay more attention to the relationship between foraging rate and body mass, and the physical property that mediates of heat exchange between the thorax and the external environment, as well as the more detailed description of the external environment.

???Although the model here is simple, it lays foundation in building a generalized fitness landscape???
