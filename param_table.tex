% E: Some suggestions for fitting this on one page: prune words from Definition, smaller font, references as footnotes

\begin{table}
\begin{threeparttable}[b]
\caption{Values and ranges of parameters used }
\begin{tabular}{l l l l l}
\toprule
 & Definition & Value or range & Units & Ref. \\
\midrule
&\textbf{Intrinsic body size scaling} & & &  \\
$a_1$ & Coefficient for resting metabolic rate  & $\propto 1$  & $\rm{J \, s}^{-1}$ & \tnote{1} \\ % E: what does "proportional to 1" mean?
$b_1$ & Exponent for resting metabolic rate  & 0.75 &  & \tnote{2} \\
$a_2$ & Coefficient for active metabolic rate  & 10-30 $ \times a_1$ & $\rm{J \, s}^{-1}$ &  \tnote{3} \\
$b_2$ & Exponent for active metabolic rate  & 0.75-1.25 & &  \tnote{1} \\
$a_3$ & Coefficient for foraging rate  & 1 & $\rm{g \, s}^{-1}$  & \\
$b_3$ & Exponent  for foraging rate  & 0.5-1.25 &  &  \tnote{4}  \\
$c_0$ & Intercept for minimum temperature for activity & 28 & $^{\circ}\rm{C}$  & \tnote{5}\\
$c_1$ & Slope for minimum temperature for activity & 0.75 &  $\rm{g \,  ^{\circ}C^{-1}}$ &  \tnote{5} \hspace{0.1cm}* \\
%\midrule
& \textbf{Physical and thermodynamic constants} & & &  \\
$\delta $ & Mass density & $0.15 \times 10^6$  & $\rm{g \, m}^{-3}$  & \tnote{6}\\
$a_w$& Frequency of contraction & 0.25 & $\rm{s}^{-1}$   & \tnote{7} \hspace{0.1cm}*\\
s & Specific heat capacity & 3.3472 & $\rm{J \, g}^{-1}\,\rm{C}^{-1}$ & \tnote{1} \\
e & Energy per contraction & 0.04184 & $\rm{J \, g}^{-1}$ & \tnote{8} \\
$K_1$& Default conductance & 0.05 $c_p$ & $\rm{J \,s}^{-1} \, \rm{m}^{-2} \, ^{\circ}\rm{C}$  & \tnote{9} \\
$K_2$& Default convection & 1   & $\rm{J \,s}^{-1} \, \rm{m}^{-2} \, ^{\circ}\rm{C}$  & \tnote{9} \\
% E: Would it be okay to remove "default"?  It means something in the context of the figures, but not just here in the table.
%\midrule
& \textbf{Extrinsic constants} & & &  \\
$c_p$ & Molar specific heat of air  & 29.3 &  $\rm{J \, mol}^ {-1} \, \rm{C}^ {-1}$ & \tnote{9} \\
u &  Wind speed & 0.1 & $\rm{m \, s}^{-1}$ & \\
$\sigma$ & Stefan-Boltzman constant & $5.67 \times 10^{-8}$ &  $\rm{J \, m}^{-2} \rm{s}^{-1} \rm{K}^{-4}  $  &  \\
$\varepsilon$& Emissivity of gray body & 0.93& & \tnote{9} \\
$\rho$ & Energy density per gram of resource & 10-100 &  $\rm{J \, g}^{-1}$  &  \\  %  Energy density per gram  dry dung (40-80\%) of total weight, \citet{Nibaruta1980} \citet{Gittings1998}
$r_3$  & Scale factor for solar radiation & 0.5 &  &  \\
\bottomrule
\label{table:table1}
\end{tabular}
% E: I think there is a fancier latex way of displaying numbered references in a table.  But this is good enough for now.
\begin{tablenotes}
  \item[1] \citet{Heinrich1975}
  \item[2] \citet{Kleiber1947,Peters1986,Gillooly2001}
  \item[3] \citet{Bartholomew1981,Niitepold2010}
  \item[4] \citet{Pawar2012, Nervo2014, Maino2015}
  \item[5] \citet{Bartholomew1977a}
  \item[6] THR unpub.\ data
  \item[7] \citet{Bartholomew1977b}
  \item[8] \citet{Kammer1974}
  \item[9] \citet{Campbell2012}
\end{tablenotes}
\end{threeparttable}
\raggedright{*means that the value is approximated.}
\end{table}
