\section*{Results}
%%%%%%%%%%%%%%%%%%%%%%%%%%%%%%%%%%%%%%%%%%%%%%%%%%%%%%%%%%%%%%%%%%%%%%%%%%%%
\subsection*{Role of body size scaling}
\subsubsection*{Resource availability and temperature}
We first explored how performance varies as a function of the amount of resource available.
With limited resources,  there is an upper limit to the amount of energy available in the environment, which thus eventually penalizes those with high metabolic costs.
Because metabolic costs increase with  body size, for large values of $z$, net energy gain eventually decreases with body size. 
With limited resources, net energy gain is thus maximized at intermediate body size (\cref{fig2}ab). 
 When resources are unlimited (we assumed that individuals can take up to 50 times their body size), net energy gain increases with body size (\cref{fig2}c).

In the previous scenario, the value of the exponent of the foraging rate ($b_3$) does not change the qualitative pattern. % E: Not clear what the "previous scenario" is.  Maybe move this point to the end of the previous paragraph?
However,  the exponent $b_3$ can lead to qualitatively different patterns  for a combination of high temperature and high metabolic cost for activity, even with unlimited resources (\cref{fig2}def).
If the foraging rate is concave (\cref{fig1}), large individuals actually forage for a longer period because they are less efficient in gathering resources.
A longer period for  activity means the energetic cost for activity is higher.
At high temperature, the cost is magnified and  large individuals eventually have a negative net energy gain (\cref{fig2}ef, dashed lines).
The same phenomenon occurs when foraging rate is convex, thought it instead penalizes the smallest individuals (\cref{fig2}f, thick solid line).
 %%%%%%%%%%%%%%%%%%%%%%%%%%%%%%%%%%%%%%%%
% E: can the titles of these two subsubsections be made to fit together better?
\subsubsection*{Time is limiting}
We also looked at a different scenario where foraging time is limited.
In general, net energy gain increases with body size (\cref{fig3}ac).
However, when the exponent of the foraging rate is less than the exponent of the metabolic rates ($b_3 > b_2 = b_1$) and resource quality is within a certain range,  net energy gain peaks at intermediate body size.

% E: I know you've worked hard to condense this, but it still feel much more technical than everything else.  I'll come back here and see if I can add something a bit more intuitive.
Mathematically (see Appendix for complete derivation), this means that the range of the resource quality $\rho$ allowing intermediate optimal body mass is % E: Appendix 1?
\begin{equation}\label{eq:C1}
	\widetilde{E_n} < \rho < \widetilde{dE_n},
\end{equation}
where $\widetilde{E_n} < \rho $ ensures that net energy gain is positive (lower limit of the shaded areas in \cref{fig3}b) and $\rho < \widetilde{dE_n}$ ensures that the derivative with respect to body size $z$ becomes negative (upper limit of the shaded areas in \cref{fig3}b).
$\widetilde{dE_n}$ is in fact $\widetilde{E_n}$ weighted by $\dfrac{b_1}{b_3}$  and $\dfrac{b_2}{b_3}$ (see Appendix), and thus $\widetilde{dE_n}$ is greater than $\widetilde{E_n}$ if  $b_3 < b_1$ ($b_2 \geq b_1$ is always true). 
Temperature has the same multiplicative effect on $\widetilde{E_n}\textnormal{ and }\widetilde{dE_n}$ which means that the range of values for $\rho$ in \cref{eq:C1} increases with temperature (\cref{fig3}b).
%%%%%%%%%%%%%%%%%%%%%%%%%%%%%%%%%%%%%%%%%%%%%%%%%%%%%%%%%%%%%%%%%%%%%%%%%%%%%%%
\subsection*{Role of warm-up}
\subsubsection*{Minimum temperature for completing warm-up}
%For the parameter considered here, when solar radiation is not limiting and without wind (free convection), any individual can absorb and use that energy to complete warm-up.
%Here, we explore cases where solar radiation is limiting and wind is not negligible.
If the environmental temperature is too low, it may be impossible to reach the operating temperature. 
As environmental temperature increases, some species can complete warm-up and others cannot. % E: Do you mean temperature increases in math/abstract, or during the day?
Here we explore how such ability can depend on body size.
% E: "better" here is a nice short-hand, but I wonder if something more formal would be preferable, e.g., "advantageous"
For ectotherms in the absence of wind, smaller is better it increases the surface area-to-body size ratio and thus increases the ability to absorb more heat per unit of mass  (\cref{fig4}a). 
For endotherms, larger is better for the opposite reason: more heat is retained within the body  (\cref{fig4}c). 
However, with wind, ectotherms of intermediate sizes are better because small sizes are penalized due to increasing laminar convection ($h$ in \cref{eq:dTn} and \cref{fig4}b)  and large sizes are penalized because they have higher operative temperatures (\cref{eq:Tw}). 
Without the effect of operative temperature, large becomes better (supplementary figure).
%%%%%%%%%%%%%%%%%%%%%%%%%%%%%%%%%%%%%%%%
\subsubsection*{Duration of warm-up}
To add a bit more realism, we assume here that environmental temperature increases from sunrise to mid-afternoon (see Appendix 2 for the definition).
% E: I wonder if it is wise to re-raise the matter of constant temperature.  It makes the reader wonder about the assumptions behind all the other results.  Is this the only way to get something interesting to say about warm-up duration?  Did you already convince me that it is foolish to present all results with this daily temperature variation?
As expected, the duration of warm-up decreases as this intensity of solar radiation increases (\cref{fig5}a).
The decrease is not linear as the duration of  warm-up decreases abruptly and then levels off a few hours after sunrise (\cref{fig5}a).
For ectothermic individuals, the duration of warm-up increases with body mass (\cref{fig5}a).
For endothermic individuals, the same pattern occurs but is less abrupt (supplementary figures).

We found that for endotherms, different values for the conductance are favored at different times of the day (\cref{fig5}b).
If warm-up is initiated early in the day when solar radiation is weak, low conductance is better because it limits heat loss (thick line).
As the intensity of solar radiation increases, solar radiation becomes a dominant source of heat, and transferring that heat to the thorax is better achieved with high conductance (dashed line).
%%%%%%%%%%%%%%%%%%%%%%%%%%%%%%%%%%%%%%%%%%%%%%%%%%%%%%%%%%%%%%%%%%%%%%%%

\subsection*{Thermal performance across body size}
 We integrated the components above to see how they shape thermal performance and how thermal performance varies with body size.
Warm-up can have a major negative effect on performance (\cref{fig6}a).
On one hand, the colder it gets, the longer it takes to warm up and thus the less time is left for foraging.
(We assumed a fixed time for foraging, which is independent of body size.)
On the other hand, if an individual delays warm-up initiation to take advantage of more intense solar radiation, the negative effect on performance is reduced (\cref{fig5}a and \cref{fig6}a). % E: Is this true even when foraging stops at a fixed time of day?  Nice to know that early risers don't always get more done! T: I am not sure I follow you. If foraging stops at say 9am. One that warm-up at 6 am might get an advantage compared to another one that starts at 8:45 am even if it takes less amount of time for the latter to warm-up. In here, I assume they are given the same amount of time and they decide when I start the timer.   E:  Okay.  New question: is it ever better to delay warm-up?
Qualitatively, there is no difference between endotherms and ectotherms because warm-up always takes longer for larger individuals.  % E: Not sure if "no difference" is safe here.  Alternative: "Qualitatively, warm-up always takes longer for larger individuals, whether they are endotherms or ectotherms."

The exponent of foraging rate ($b_3$) drives how performance curves vary with body size.
If foraging rate is a convex function of body size, performance breadth increases with body size (thick line in \cref{fig3} and solid lines in \cref{fig6}b) and large individuals perform better than smaller ones at any temperature (thick solid line above thin solid line in \cref{fig6}b).
However, if foraging rate is a concave function of body size, performance breadth shifts with body size (dashed lines cross, \cref{fig6}).
When temperature is high, small is advantageous.
When temperature is low, large still performs better. % T: is this needed?  E: I think good to keep.
   
Finally, we looked at the influence of a reduction in resource availability (e.g., habitat loss) on thermal performance.
We found that large individuals will be greatly affected, whereas there is not much difference for small individuals (\cref{fig2}abc and \cref{fig6}c). % E: or large will be "much more affected" than small
This reduction in resource availability shrinks the performance breadth of large individuals, but it also decreases performance such that small is better along the entire temperature gradient.

Our results show that there is no single relationship between performance breadth and body size. 
Different factors---here concavity of foraging  rate and resource availability---generate three cases where large individuals have broader, shifted, or narrower performance breadth.
