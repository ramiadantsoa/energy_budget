\section*{Results}

Our main interest is how net energy gain varies as a function of body mass and temperature.
We investigate how the thermal performance curve differs---in width, location, and shape---for individuals of different body mass.
We start by exploring the role of scaling with body mass, temperature, and resource availability, without considering warm-up.
We then focus on warm-up only.
Finally, we report the combined effects of all processes on the thermal performance curve.
Throughout, we emphasize the diverse outcomes that are possible from our model, rather than provide an exhaustive description of the effects of each parameter.

%%%%%%%%%%%%%%%%%%%%%%%%%%%%%%%%%%%%%%%%%%%%%%%%%%%%%%%%%%%%%%%%%%%%%%%%%%%%

\subsection*{Does net energy gain increase or decrease with body mass?}

Larger individuals expend less energy per unit mass (they have a lower mass-specific resting metabolic rate; $b_1 < 1$ in \crefp{eq:eb}), so it is intuitive that net energy gain increases with body mass.
Our model indeed shows this behavior, at least when resources are abundant and foraging time is not constrained (\crefp{fig2}a).
In contrast, when resource quantity is more limited, net energy gain peaks at intermediate body mass (\crefp{fig2}b) because not enough energy can be obtained to sustain the high absolute metabolic cost of large-bodied individuals.
The scaling of foraging rate, $b_3$, does not affect these qualitative patterns.

Alternatively, when foraging time is limited, resource quality determines the form of net energy gain.
Net energy gain increases monotonically with body mass when resource quality is high (\crefp{fig2}d), as when resource quantity is high (\crefp{fig2}a).
Net energy gain may peak at intermediate body size, however, when resource quality is not too high (\crefp{fig2}c).
Furthermore, the foraging exponent $b_3$ affects these qualitative patterns.
Our analytical results show that net energy gain peaks at intermediate body mass if and only if $b_3 < b_2$ and resource quality is within a certain range (see Appendix).

Net energy gain can also peak at intermediate body mass if the active metabolic rate is increased, because larger individuals are penalized more strongly (\crefp{fig2}e, cf.~\crefp{fig2}a).
This effect is exaggerated at higher temperatures, where again larger individuals pay a higher price (\crefp{fig2}f).
The exponent $b_3$ now dictates whether net energy gain always increases or eventually decreases with body mass.
Even if resources are unlimited, a concave foraging rate ($b_3 < 1$) means that foraging time increases with body size.
Consequently, larger individuals are active for a longer period of time and therefore pay a higher metabolic cost.

When foraging time is fixed, higher active metabolic rate again penalizes larger individuals (\crefp{fig2}g, cf.~\crefp{fig2}d).
Higher temperature does not change the qualitative pattern in this case, but net energy gain peaks at lower body mass (\crefp{fig2}h).

% E: Please feel free to edit this.  The goal is to give the reader a simple take-away message.
Overall, we find that there may or may not be an intermediate optimal body mass when considering only how metabolic costs and resource acquisition scale with body mass and temperature.
Small individuals generally have low net energy gain because they gather little resource. % T: well the obvious answer is that they are small so their energy budget is small by definition (if we look at mass specific net energy gain, it might be higher)
Large individuals may also have low net energy gain when resource quality and quantity is low or their metabolic costs are high.
Otherwise, larger body mass allows greater net energy gain.

%%%%%%%%%%%%%%%%%%%%%%%%%%%%%%%%%%%%%%%%%%%%%%%%%%%%%%%%%%%%%%%%%%%%%%%%%%%%%%%

\subsection*{When is warm-up completed?}
% E: Maybe you can do better.  But if the other subsections are questions, this one should be, too.

We now set aside the metabolic and foraging processes in order to focus on the warm-up phase.
% E: or something fancier about behavioral thermoregulation

\subsubsection*{Minimum temperature for completing warm-up}

% T:  An important note here is that when solar radiation is high, warm-up can always be completed. Here I just explored conditions where we can get interesting results. I think it should be said somewhere...
In cold environments, it might not be possible for an individual to elevate its body temperature to the level needed for activity.
Here, we explore how the ability to complete warm-up depends on body mass.
For ectotherms and in the absence of wind, smaller size is advantageous as decreasing body mass increases the surface area-to-body mass ratio and thus increases the ability to absorb more heat from the environment per unit of mass (\crefp{fig3}a, dashed line).
However, wind elevates the minimum temperature for completing warm-up as convection has stronger effect in cooling the surface of the body.
Unlike the wind-free case, ectotherms of intermediate masses are able to complete warm-up under colder temperatures because individual with small body masses are penalized due to increasing laminar convection ($h$ in \crefp{eq:dTn}; \crefp{fig3}b).
Large masses are penalized because they have higher operative temperatures (\crefp{eq:Tw}); without the cost of higher operative temperature, large becomes better (Supplementary Figures).
% For endotherms, larger size is advantageous for the opposite reason: more of the heat generated endogenously is retained within the body
For endotherms, minimum temperature for completing warm-up decreases with increasing body mass because more of the heat generated endogenously is retained within the body due to small surface area-to-body mass ratio (\crefp{fig3}a, thick line).
In fact, there is a threshold value where endogenous heat production cancels with heat dissipation (for reasonable environmental temperature), above that value the ability to complete warm-up becomes independent of body mass (\crefp{fig3}a, thick line).

\subsubsection*{Duration of warm-up}

We also explore how the duration of warm-up changes with body mass and temperature.
In general, we find that the duration of warm-up increases with body mass and obviously decreases with increasing temperature (\crefp{fig3}b).
A special case is when endotherms are too small, warm-up can be exceedingly low as a high rate of heat dissipation results into a positive but low warm-up rate (\crefp{fig3}b, thick lines).
We found that duration of warm-up is longer, changes more rapidly with body mass and changes more significantly with temperature for ectotherms than for endotherms (\crefp{fig3}b). % T: to a certain degree, I am reluctant to compare endotherms and ectotherms...especially quantitatively
The major difference is that the duration of warm-up  is a concave function of body mass for ectotherms and a almost not linear function of body mass for endotherms (\crefp{fig3}b).
The concavity arises because the dominant process regulating warm-up rate is the surface area-to-body mass ratio (the exponent is 2/3) whereas the linearity stems from the effect of thoracic mass (the exponent is 1).
% We also found that the duration of warm-up is usually longer and the effect of temperature more important for ectotherms than for endotherms % T: (Supplementary files?)  I did not explore that much that parameter space

% T: I have not explored in depth the next results but it seems that with wind, the curve becomes a bit complicated with both endotherms and ectotherms. For endotherms, it becomes shallower for large body mass and steeper for small body mass. Although I am wondering why would the come out since it might be better to remain hidden if it is too windy. For ectotherm, given the same parameters, large and small cannot complete warm-up when too cold (10 C). The curve looks like a valley. When temperature is high (20 C) it becomes a convex function of body mass.

% E: Considered this for a sum-up, but it seems redundant.
% For the warm-up phase alone, we thus learn that for ectotherms, smaller individuals warm up more quickly only when it is not windy.
% For endotherms, larger individuals warm up more quickly, but conductance plays a large role.
% In all circumstances, warm-up is achieved much more rapidly when it begins later in the day, but this comes at the cost of reduced foraging time.
% warm-up always takes longer for larger


%%%%%%%%%%%%%%%%%%%%%%%%%%%%%%%%%%%%%%%%%%%%%%%%%%%%%%%%%%%%%%%%%%%%%%%%

%\subsection*{Thermal performance across body mass}
% \subsection*{Does thermal performance curve shrink, broaden, or shifts as a function of body mass?}
\subsection*{How does the thermal performance curve change with body mass?}

We integrated the components above to see how they shape thermal performance and how thermal performance varies with body mass.
First, we looked at how the inclusion or exclusion of warm-up can affect thermal performance.
When the environmental temperature is high, warm-up is not needed and thus does not influence the net energy gain at high temperature and by extension the upper thermal limit (\crefp{fig4}).
When the environmental temperature is low, warm-up can reduce the net energy gain because a portion of foraging time is ``wasted" for warming up.
Therefore, we focus here on cases where foraging time is limited (\cref{eq:ed}).

The colder it gets, the longer it takes to warm up, the shorter time is left for foraging, and thus the lower the net energy gain is (\crefp{fig4}bc).
As we mentioned earlier, the variation in warm-up duration (for different body mass and temperature) is more important for ectotherms than for endotherms.
Consequently, the reduction in net energy gain as temperature decreases is more significant for ectotherms than for endotherms.
In fact, warm-up time can be sufficiently long for large ectotherms that for low temperature, intermediate body mass is better (\crefp{fig4}c).

Second, we looked at the effect of convexity or concavity of the foraging rate.
The most notable effect is that the exponent $b_3$ shifts the upper thermal limit.
If foraging rate is a convex function of body mass, performance breadth increases with body mass (thick lines in \crefp{fig5}ac) and large individuals perform better than smaller ones at any temperature--the difference in lower thermal limit is caused by warm-up (thick line above thin line in \crefp{fig5}ac).
However, if foraging rate is a concave function of body mass, performance breadth shifts with body mass (thick line and thin solid cross \cref{fig5}d).
When temperature is high, small is advantageous.

Finally, we looked at the influence of a reduction in resource availability (e.g., due to habitat loss).
We found that the negative effect of resource limitation increases with body mass (\crefp{fig2}b and \crefp{fig5}e--h).
This reduction in resource availability shrinks the performance breadth of large individuals, but it also decreases performance such that small can be better along the entire temperature gradient.

% T: Say something that with the large number of parameters we did not exhaust all parameter space, we point out various qualitative result by changing some parameters
Our results show that there is no single relationship on how thermal performance curves changes with body mass.
Different factors---here concavity of foraging rate and resource availability---generate three cases where large individuals have broader, shifted, or narrower performance breadth.
