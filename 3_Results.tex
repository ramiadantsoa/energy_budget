\section*{Results}
The results are partitioned into 3 parts.
Many things go into performance. We  single out the effects here.
 Ask, when can performance peak at intermediate body size.

- Physiological and ecological processes.
When resource or time is limiting.

Fig 1 shows how net gain changes as function of body size

 Case 1: when resources are low, the gain cannot compensate for large and thus net gain decreases with body size. 
 This effect vanishes as resource increases, net energy gain increases as function of body mass.
 What it is interesting here is that exponent of foraging rate, or metabolism, and metabolic scope interferes only quantitatively. 
 
 Case 2: we now look at the role of temperature.
 Temperature increases the metabolic cost.
 Foraging is constant here but we vary the cost being active ($b_2 = 1.25, a_2 = 40$).
 Even if resources are unlimited.
 As temperature increases, it becomes more costly for large given that $b_3$ is low. 
 The ratio plays a role here.
 As a contrast, when b3 is high, only large can persist. 

 It is worth to mention that foraging time differs here and this for low b3, it takes longer for large to get 50 times the resources and thus even higher active metabolic rate   
 
 Fig 2 shows when time is limiting
 In the same scenario as above, large is always advantageous when b3 is high enough.
 When b3 is low the same hump pattern occurs.
This only happens at specific situation.
First, net energy can should increases and then decrease after certain threshold (thus looking at the derivatives)
Second, net energy should also be positive.
These conditions are fulfilled when: 

 There are two things: the $b_3$ as we have seen before but also, resource quality matters.
 Sound too restrictive. 
 

We explored the influences of two internal parameters: $b_3$ (\cref{eq:eg}) which scales the effect of body size on resource allocation and conductance $K_1$ and $K_2$ which defines the passive loss of heat during warm-up, convection, and two environmental variables: environmental temperature $T_e$ and the amount of resource available $R$.

Our main goal is to understand the role of different processes in shaping the energy budget of a species given its body mass.

% E: Good strategy to give basics of model behavior first, and then put it together for the more complicated niche questions!

\subsection*{Role of foraging and cost of metabolism}
We look at the cases that can limit the amount of resources that an individual can acquire.
The first case is when the amount of resource available is limited (\cref{fig1}).
When there is not enough resource, the high energetic demand of large individual due to higher metabolic rate cannot be compensated.
As a consequence, as net energy gain decreases as function of body size (\cref{fig1}ab).
When resources are not limiting, net energy gain increases with body size.
More importantly, the foraging exponent $b_3$ only changes the pattern qualititatively.

The situation may be different when temperature is varied.
By increasing the metabolic scope (high cost of activity), large individual shows two contrasting responses.
When foraging exponent is low ($b_3 = 0.5$ ), performance is again maximized at intermediate value even if resources are not limiting (\cref{fig1}d vs. \cref{fig1}e).
Increasing temperature puts more pressure on large individuals as the metabolic cost is higher.
When foraging exponent is high $b_3 = 1.25$, only large individual can persist (\cref{fig1}f).

The second case is when the time available for foraging is limited (\cref{fig2}).
In general, net energy gain increases as function of body size.
However, like in the previous case, when $b_3$ is low, maximum net energy gain is attained at intermediate body size (dashed line, \cref{fig2}a). 
The analytical result shows that net energy gain only peaks when two conditions  are met.
First, $b_3 < b_1 = b_2$ (the equality is by assumption and to reduce the number of free parameter).
% E: so really, the condition is that b_3 is the smallest? 
% T: yes
Second, resource quality $\rho$ is within a certain range represented by shades in \cref{fig2}a.
\cref{fig2}b shows for two different temperatures (red = warm, blue = cold) the required values of the resource quality $\rho$ so that net energy gain peaks at intermediate value of body mass.
\cref{fig2}c shows how net energy gain changes as function of body mass for the resource quality shown in \cref{fig2}b.
These conditions are quite restrictive and looks unlikely in real system.
% E: This is a great example of why it was worth constructing a math model, rather than just intuitive arguments!

In a more technical term, the range of the $\rho$ allowing intermediate optimal body mass is 
\begin{equation}\label{C1}
	\widetilde{E_n} < \rho < \widetilde{dE_n}.
\end{equation}
where,
\begin{flalign*}
\widetilde{E_n} &= \theta_1 + \theta_2, \\
\widetilde{dE_n} &= \frac{b_2}{b_3} \theta_1  +  \frac{b_1}{b_3} \theta_2.
\end{flalign*}
and $$\theta_1 = \frac{a_2}{a_3}  z^{b_2 - b_3}  e^{-E/[k (max(T_w(z_{th}),T_e(t))+ 273.15)]}$$ and $$\theta_2 =  \frac{a_1}{a_3} z^{b_1- b_3}  e^{-E/[k (T_e(t)+ 273.15)]} (\frac{24}{t_f} -1).$$

The difference between  $\widetilde{E_n}$ and  $\widetilde{d E_n}$ is that in  $\widetilde{dE_n}$, each term of  $\widetilde{E_n}$   is weighted by $\frac{b_2}{b_3}$ and $\frac{b_1}{b_3}$.
Thus, a necessary condition for optimal mass to be intermediate (\cref{C1}) is that the weights are larger than 1 i.e.  $b_3 < b_1$ ($b_2 \geq b_1$ is always true). 
As temperature increases, the term with the product $\frac{b_1}{b_3}$ becomes larger thus increasing the range of values where \cref{C1} is true (\cref{fig2}a).

Note  that the bandwidth is broader in warm than in cold environment.
This is because in warm environment , the $\theta$s are larger arithmetically and not geometrically.
It is easier to use example. 
For simplicity, if $\theta_1 = \theta_2 = 1 (10)$ small (or large) and $b_1 =b_2 = 0.75, b_3 = 0.5$.
For small, the range is $\theta_1 + \theta_2 = 2 < \rho <  0.75/ 0.5 \theta_1 +  0.75/0.5 \theta_2 = 3.$  
For large, the range is $\theta_1 + \theta_2 = 20 < \rho <  0.75/ 0.5 \theta_1 +  0.75/0.5 \theta_2 =  30.$  


\subsection*{Role of warm-up.}
\subsubsection*{Minimum conditions}
Successful warm-up is a necessary condition before foraging. 
Under perfect conditions where solar radiation is not limiting and forced convection from wind is absent , any individual can absorb use that energy to warm-up.
We explore here cases where solar radiation is limiting and wind increasing convection between the surface of the individual and the external environment.

Three different ways the minimum temperature required to complete warm-up vary as a function of body mass (\cref{fig3}).
First, the minimum temperature for warm-up increases with body mass because small individual absorbs more heat--higher surface area-to-body ratio (\cref{fig3}a).
This pattern only occurs when the individual relies only solar radiation to warm-up (ectotherm) and there is only free convection (no wind).
Second, the minimum temperature for warm-up is lowest at intermediate body size (\cref{fig3}b). 
This happens because there is now more convection due to wind. 
Small individuals are  penalized as convection increases (see \cref{eq:dTr2}).
Third, minimum temperature decreases with body size (\cref{fig3}c).
It occurs when individuals are capable of producing heat endogenously (endotherm).
In this case, the role of surface area-to-body ratio is reversed as the heat generated in the thorax dissipates less with large individuals.
% E: I repeat my comment above: This is a great example of why it was worth constructing a math model, rather than just intuitive arguments!

\subsubsection*{Duration of warm-up.}
Another aspect of warm-up is the time it takes to complete it.
As expected, duration of warm-up decreases as this intensity of solar radiation increases (it peaks at noon) (\cref{fig4}).
The decrease is not linear, warm-up time decreases abruptly and then level off few hours after sunrise (\cref{fig4}a).
For endothermic individual, the same pattern occurs although it is less abrupt compared to ectothermic individuals (\cref{fig4}b) .
For ecothermic individuals, the duration of warm-up increases with body mass (\cref{fig4}a).
For endothermic, it is not necessarily true because the smallest individual can lose too much of the heat it generates to the environment via conductance (\cref{fig4}b).

Conductance between the thorax and the rest-of-the-body plays an important role for the warm-up process and, as expected, has opposing roles for ecotherm and endotherm.
For endotherm, \cref{fig4}c shows how different values for the conductance are favored at different parts of the day.
If warm-up is initiated early in the day where solar radiation is weak, low conductance is better (thick line \cref{fig4}c).
As the intensity of solar radiation increases, it becomes a dominant source of energy and transferring that energy to the thorax is better achieved with high conductance (dashed line \cref{fig4}c).

\subsection*{Niche as a sum of physiological and behavioral processes.}
Now we integrate those components together to see how they shape the net energy gain as function of temperature. 
We also include performance when daily temperature changes. 
In that case and for simplicity, we assume that temperature increases linearly from sunrise to mid-afternoon and then decreases again (see appendix). 
There is no temperature at the previous and next sunrise.

\cref{fig5}a shows a default scenario with intermediate resource abundance.
For a range of daily temperature, net energy gain is highest at intermediate body size, large is not optimal for the same reason as in \cref{fig1}.
As resource abundance decreases, net energy gain decreases for larger individuals (solid lines \cref{fig5}b).
This makes small individual (relatively) more competitive and actually perform better at high temperature.
Also note that the upper limit of the `niche' is contracted to lower temperature (\cref{fig5}b), simply because the amount of resource available does not compensate the high energetic cost.

When we integrate warm-up processes, which is important when temperature is cold, large individual is also more affected.
Because it takes too much time to warm-up (\cref{fig4}bc) whether one is endothermic or ectothermic.
We assumed here that time spend during warm-up is subtracted in the total time available for foraging.
In short, warm-up contracts the lower part part of the niche.
Supplementary material contains more detailed results where warm-up cannot be completed so that the lower limit will be dictated by the patterns in \cref{fig3}.
...
% E: discuss endo vs ecto?

Finally, we look at variation in daily temperature.
Instead of having constant daily temperature, we assumed that increases throughout the day.
This increases in temperature further affects the largest individual (thick line in\cref{fig5}d).
The first reason is depicted in \cref{fig1}def where temperature increases disproportionately increases the energetic demand of large individuals even if resources are unlimited.
The decline of large individual will be unavoidable when resources are limiting (\cref{fig1}ab and \cref{fig5}d).



