\section*{Results}
Our main interest is to investigate how net energy gain varies as a function of body size and temperature, namely whether change in body size shrinks, broadens, or shifts thermal performance curve.
Given the large number of processes in the model, we start by exploring the role of scaling of body size, temperature, and resource availability without considering warm-up process, then we focus on warm-up only, at end we report their combined effects on the thermal performance curve.

%%%%%%%%%%%%%%%%%%%%%%%%%%%%%%%%%%%%%%%%%%%%%%%%%%%%%%%%%%%%%%%%%%%%%%%%%%%%
%\subsection*{Role of body size scaling, resource, time, and temperature}
\subsection*{Does net energy gain increase or decrease with body size?}
Intuitively, because mass-specific metabolic rate decreases with body mass, the net energy gain is expected to increase with body mass.
We found that when resource quantity is unlimited the net energy gain, indeed, increases with body mass (\cref{fig2}a).
We note that although resource are unlimited, an individual can only gather resource up to 50 times its body mass.
In contrast, when resource quantity is limited, the net energy gain peaks at intermediate body mass (\cref{fig2}b) simply because there is not enough energy to sustain the high absolute metabolic cost of large-bodied individuals.
In this scenario, the exponent $b_3$ does not affect the qualitative pattern (\cref{fig2}ab).

An alternative scenario is to assume that foraging time is limited.
In that case, resource quality determines the qualitative patterns of the net energy gain.
When resource quality is high, the net energy gain increases with body mass (\cref{fig2}c).
When resource is low, the net energy gain again peaks at intermediate body size (\cref{fig2}d).
Furthermore, changing the exponent $b_3$ alters the qualitative pattern.
Our analytical results show that the net energy gain peaks at intermediate body mass if and only if $b_3 < b_2$ and when resource quality is within a certain range (see Appendix).
For instance, when resource quality is high, the net energy gain always increases with body mass (\cref{fig2}c).

At third scenario where the net energy gain can peak at intermediate body size is when the metabolic cost is high because the environmental temperature, the coefficient $a_2$, and the exponent $b_2$ are higher.
Unlike the first scenario, the exponent $b_3$ dictates whether the net energy gain always increases or eventually decreases with body size (\cref{fig2}ef).
Even if resource are unlimited, a concave foraging rate ($b_3 < 1.$) means that foraging time increases with body size.
Consequently, large individuals are active for a longer period of time and therefore pay a  higher metabolic cost.
When foraging time is fixed, higher temperature does not change the qualitative pattern but the net energy gain peaks at lower body mass (\cref{fig2}cg, and \cref{fig2}d vs. \cref{fig2}h).
%%%%%%%%%%%%%%%%%%%%%%%%%%%%%%%%%%%%%%%%%%%%%%%%%%%%%%%%%%%%%%%%%%%%%%%%%%%%%%%

%\subsection*{Role of warm-up}
\subsection*{Warm-up potential}

\subsubsection*{Minimum temperature for completing warm-up}
%For the parameter considered here, when solar radiation is not limiting and without wind (free convection), any individual can absorb and use that energy to complete warm-up.
%Here, we explore cases where solar radiation is limiting and wind is not negligible.
If the environmental temperature is too low, body temperature might not reach the operating temperature.
%As environmental temperature increases, some species can complete warm-up and others cannot. % E: Do you mean temperature increases in math/abstract, or during the day?
Here, we explore how the ability of completing warm-up depends on body size.
For ectotherms and in the absence of wind, smaller is advantageous as decreasing size increases the surface area-to-body size ratio and thus increases the ability to absorb more heat per unit of mass  (\cref{fig3}a).
For endotherms, larger is advantageous for the opposite reason: more heat is retained within the body  (\cref{fig3}c).
However, with wind, ectotherms of intermediate sizes are better because small sizes are penalized due to increasing laminar convection ($h$ in \cref{eq:dTn} and \cref{fig3}b)  and large sizes are penalized because they have higher operative temperatures (\cref{eq:Tw}).
Without the cost of higher operative temperature, large becomes better (Supplementary Figures).
%%%%%%%%%%%%%%%%%%%%%%%%%%%%%%%%%%%%%%%%
\subsubsection*{Duration of warm-up}
% E: I wonder if it is wise to re-raise the matter of constant temperature.  It makes the reader wonder about the assumptions behind all the other results.  Is this the only way to get something interesting to say about warm-up duration?  Did you already convince me that it is foolish to present all results with this daily temperature variation?
% T: I will try to convince you now :D the results don't change much. In fact I changed the figures so that temperature remains fixed. I am not found of replacing all the results by considering temperature variation because 1- as far as I looked, it does not change much of the results. 2- It creates additional parameter as to how much does it increase during the day, is linear ok, when does it peak... 3- It adds questions about trade-off between the cost of warm-up during cold environment but less metabolic cost or now warm-up because it is hot enough but then high metabolic cost
As expected, the duration of warm-up decreases as this intensity of solar radiation increases (\cref{fig4}a).
The decrease is nonlinear as the duration of  warm-up decreases abruptly and then levels off a few hours after sunrise (\cref{fig4}a).
The same pattern occurs even if temperature increases during the day (see Supplementary Figures)
For ectothermic individuals, the duration of warm-up increases with body mass (\cref{fig4}a).
For endothermic individuals, the same pattern occurs but the difference is less pronounced and the slope shallower (see Supplementary Figures).

We found that for endotherms, different values for the conductance are favored at different times of the day (\cref{fig4}b).
If warm-up is initiated early in the day when solar radiation is weak, low conductance is better because it limits heat loss (thick line).
As the intensity of solar radiation increases, solar radiation becomes a dominant source of heat, and transferring that heat to the thorax is better achieved with high conductance (dashed line).
%%%%%%%%%%%%%%%%%%%%%%%%%%%%%%%%%%%%%%%%%%%%%%%%%%%%%%%%%%%%%%%%%%%%%%%%

%\subsection*{Thermal performance across body size}
\subsection*{Does thermal performance curve shrink, broaden, or shifts as a function of body size?}
We integrated the components above to see how they shape thermal performance and how thermal performance varies with body size.
First, we looked at how the timing of warm-up can affect thermal performance.
% Warm-up can have a major negative effect on performance (\cref{fig5}a).
On one hand, the colder it gets, the longer it takes to warm up and thus the less time is left for foraging (\cref{fig5}a) unless warm-up is delayed to take advantage of intense solar radiation (\cref{fig5}b).
(We assumed a fixed time for foraging, which is independent of body size.)
Qualitatively, warm-up always takes longer for larger individuals, whether they are endotherms or ectotherms.
As such, large will suffer more from poorly timed warm-up.

Second, we looked at the effect of convexity or concavity of the foraging rate.
% We found that the exponent of foraging rate ($b_3$) dictates thermal performance breadth.
If foraging rate is a convex function of body size, performance breadth increases with body size (solid lines in \cref{fig5}c) and large individuals perform better than smaller ones at any temperature (thick line above thin line in \cref{fig5}c).
However, if foraging rate is a concave function of body size, performance breadth shifts with body size (thick line and thin solid cross \cref{fig5}d).
When temperature is high, small is advantageous.
% When temperature is low, large still performs better.

Third, we looked at the influence of a reduction in resource availability (e.g., habitat loss) on .
We found that the negative effect decreases as function of body size (\cref{fig2}b and \cref{fig5}ef).
This reduction in resource availability shrinks the performance breadth of large individuals, but it also decreases performance such that small is better along the entire temperature gradient.


Our results show that there is no single relationship on how thermal performance curves changes with body size.
Different factors---here concavity of foraging  rate and resource availability---generate three cases where large individuals have broader, shifted, or narrower performance breadth.
