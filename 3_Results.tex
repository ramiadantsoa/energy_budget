\section*{Results}

Our main interest is how net energy gain varies as a function of body size and temperature.
We investigate how the thermal performance curve differs---in width, location, and shape---for individuals of different body size.
% E: Maybe you can write this better.  But before, it sounded like the body size of an individual might be changing.
We start by exploring the role of scaling with body size, temperature, and resource availability, without considering warm-up.
We then focus on warm-up only.
Finally, we report the combined effects of all processes on the thermal performance curve.
Throughout, we emphasize the diverse outcomes that are possible from our model, rather than provide an exhaustive description of the effects of each parameter.

%%%%%%%%%%%%%%%%%%%%%%%%%%%%%%%%%%%%%%%%%%%%%%%%%%%%%%%%%%%%%%%%%%%%%%%%%%%%

\subsection*{Does net energy gain increase or decrease with body size?}

% E: I just noticed that "body size" and "body mass" seem to be used interchangeably throughout the text and figures.  I think it would be better to pick one and stick with it everywhere.  Probably mass.

Larger individuals expend less energy per unit mass (they have a lower mass-specific resting metabolic rate; $b_1 < 1$ in \crefp{eq:eb}), so it is intuitive that net energy gain increases with body size.
Our model indeed shows this behavior, at least when resources are abundant and foraging time is not constrained (\crefp{fig2}a).
(Note that even with unlimited resources available, we restrict an individual to gather only up to 50 times its body mass in order to avoid unreasonable outcomes.) % E: can you be more specific about why this is a wise constraint?
In contrast, when resource quantity is more limited, net energy gain peaks at intermediate body mass (\crefp{fig2}b) because not enough energy can be obtained to sustain the high absolute metabolic cost of large-bodied individuals.
The scaling of foraging rate, $b_3$, does not affect these qualitative patterns.

Alternatively, when foraging time is limited, resource quality determines the form of net energy gain.
Net energy gain increases monotonically with body mass when resource quality is high (\crefp{fig2}d), as when resource quantity is high (\crefp{fig2}a).
Net energy gain may peak at intermediate body size, however, when resource quality is not too high (\crefp{fig2}c).
Furthermore, the foraging exponent $b_3$ affects these qualitative patterns.
Our analytical results show that net energy gain peaks at intermediate body mass if and only if $b_3 < b_2$ and resource quality is within a certain range (see Appendix).

Net energy gain can also peak at intermediate body size if the active metabolic rate is increased, because larger individuals are penalized more strongly (\crefp{fig2}e, cf.~\crefp{fig2}a).
This effect is exaggerated at higher temperatures, where again larger individuals pay a higher price (\crefp{fig2}f).
The exponent $b_3$ now dictates whether net energy gain always increases or eventually decreases with body size.
% because the environmental temperature, the coefficient $a_2$, and the exponent $b_2$ are higher.
% Unlike the first scenario, the exponent $b_3$ dictates whether the net energy gain always increases or eventually decreases with body size (\crefp{fig2}ef).
Even if resources are unlimited, a concave foraging rate ($b_3 < 1$) means that foraging time increases with body size.
Consequently, larger individuals are active for a longer period of time and therefore pay a higher metabolic cost.

When foraging time is fixed, higher active metabolic rate again penalizes larger individuals (\crefp{fig2}g, cf.~\crefp{fig2}d).
Higher temperature does not change the qualitative pattern in this case, but net energy gain peaks at lower body mass (\crefp{fig2}h).

% E: Please feel free to edit this.  The goal is to give the reader a simple take-away message.
Overall, we find that there may or may not be an intermediate optimal body size when considering only how metabolic costs and foraging scale with body size and temperature.
Small individuals generally have low net energy gain because they gather little resource.
Large individuals may also have low net energy gain when resource quality is low or their metabolic costs are high.
Otherwise, larger body size allows greater net energy gain.

%%%%%%%%%%%%%%%%%%%%%%%%%%%%%%%%%%%%%%%%%%%%%%%%%%%%%%%%%%%%%%%%%%%%%%%%%%%%%%%

\subsection*{When is warm-up completed?}
% E: Maybe you can do better.  But if the other subsections are questions, this one should be, too.

We now set aside the metabolic and foraging processes in order to focus on the warm up phase.
% E: or something fancier about behavioral thermoregulation

\subsubsection*{Minimum temperature for completing warm-up}

%For the parameter considered here, when solar radiation is not limiting and without wind (free convection), any individual can absorb and use that energy to complete warm-up.
%Here, we explore cases where solar radiation is limiting and wind is not negligible.
In cold environments, it might not be possible for an individual to elevate its body temperature to the level needed for activity.
Here, we explore how the ability to complete warm-up depends on body size.
For ectotherms and in the absence of wind, smaller size is advantageous as decreasing size increases the surface area-to-body size ratio and thus increases the ability to absorb more heat from the environment per unit of mass (\crefp{fig3}a).
For endotherms, larger size is advantageous for the opposite reason: more of the heat generated endogenously is retained within the body (\crefp{fig3}c).
However, with wind, ectotherms of intermediate sizes are able to complete warm-up even at colder temperatures because small sizes are penalized due to increasing laminar convection ($h$ in \crefp{eq:dTn}; \crefp{fig3}b).
Large sizes are penalized because they have higher operative temperatures (\crefp{eq:Tw}); without the cost of higher operative temperature, large becomes better (Supplementary Figures).

% E: What about endotherms with wind?  I presume it is similar to Fig 4c, but would be good to say.

\subsubsection*{Duration of warm-up}

When warm-up begins later in the day, its duration decreases because the intensity of solar radiation increases (\crefp{fig4}).
% The decrease is nonlinear as the duration of warm-up decreases abruptly and then levels off a few hours after sunrise (\crefp{fig4}a).  % E: This only describes the shape of the curves.  If you want to comment further, it would be more useful to explain why the intensity of solar radiation changes non-linearly.
The same pattern occurs even if temperature increases during the day (see Supplementary Figures).
For ectothermic individuals, the duration of warm-up always increases with body mass (\crefp{fig4}a).
For endothermic individuals, the same is true but the difference is less pronounced and the slope shallower (see Supplementary Figures).

For endotherms, different values for the conductance are advantageous at different times of the day (\crefp{fig4}b).
If warm-up is initiated early in the day when solar radiation is weak, low conductance is better because it limits heat loss.
As the intensity of solar radiation increases, solar radiation becomes a dominant source of heat, and transferring that heat to the thorax is better achieved with high conductance (dashed line).
% E: How does warm-up duration change with body size for endotherms?

% E: Considered this for a sum-up, but it seems redundant.
% For the warm-up phase alone, we thus learn that for ectotherms, smaller individuals warm up more quickly only when it is not windy.
% For endotherms, larger individuals warm up more quickly, but conductance plays a large role.
% In all circumstances, warm-up is achieved much more rapidly when it begins later in the day, but this comes at the cost of reduced foraging time.
% warm-up always takes longer for larger


%%%%%%%%%%%%%%%%%%%%%%%%%%%%%%%%%%%%%%%%%%%%%%%%%%%%%%%%%%%%%%%%%%%%%%%%

%\subsection*{Thermal performance across body size}
% \subsection*{Does thermal performance curve shrink, broaden, or shifts as a function of body size?}
\subsection*{How does the thermal performance curve change with body size?}

We integrated the components above to see how they shape thermal performance and how thermal performance varies with body size.
First, we looked at how the timing of warm-up can affect thermal performance.
% Warm-up can have a major negative effect on performance (\cref{fig5}a).
On one hand, the colder it gets, the longer it takes to warm up and thus the less time is left for foraging (\cref{fig5}a) unless warm-up is delayed to take advantage of intense solar radiation (\cref{fig5}b).
(We assumed a fixed time for foraging, which is independent of body size.)  % E: fixed time for foraging, or fixed time for warm-up + foraging?
Qualitatively, warm-up always takes longer for larger individuals, whether they are endotherms or ectotherms. % E: would be good to have a figure from the previous section to cite here
As such, large will suffer more from poorly timed warm-up.
% XXX: related to warmup2

Second, we looked at the effect of convexity or concavity of the foraging rate.
% We found that the exponent of foraging rate ($b_3$) dictates thermal performance breadth.
If foraging rate is a convex function of body size, performance breadth increases with body size (solid lines in \cref{fig5}c) and large individuals perform better than smaller ones at any temperature (thick line above thin line in \cref{fig5}c).
However, if foraging rate is a concave function of body size, performance breadth shifts with body size (thick line and thin solid cross \cref{fig5}d).
When temperature is high, small is advantageous.
% When temperature is low, large still performs better.
% XXX: related to scaling (no warmup)

Third, we looked at the influence of a reduction in resource availability (e.g., habitat loss).
We found that the negative effect decreases as function of body size (\cref{fig2}b and \cref{fig5}ef).
This reduction in resource availability shrinks the performance breadth of large individuals, but it also decreases performance such that small is better along the entire temperature gradient.


Our results show that there is no single relationship on how thermal performance curves changes with body size.
Different factors---here concavity of foraging  rate and resource availability---generate three cases where large individuals have broader, shifted, or narrower performance breadth.
