\section*{Results}

Our main interest is how net energy gain varies as a function of body mass and temperature.
We investigate how the thermal performance curve differs---in width, location, and shape---for individuals of different body mass.
We start by exploring the role of scaling with body mass, temperature, and resource availability, without considering warm-up.
We then focus on warm-up only.
Finally, we report the combined effects of all processes on the thermal performance curve.
Throughout, we emphasize the diverse outcomes that are possible from our model rather than provide an exhaustive description of the effects of each parameter.

%%%%%%%%%%%%%%%%%%%%%%%%%%%%%%%%%%%%%%%%%%%%%%%%%%%%%%%%%%%%%%%%%%%%%%%%%%%%

\subsection*{Does net energy gain increase or decrease with body mass?}

Larger individuals expend less energy per unit mass (they have a lower mass-specific resting metabolic rate; $b_1 < 1$ in \crefp{eq:eb}), so it is intuitive that net energy gain increases with body mass.
Our model indeed shows this behavior, at least when resources are abundant and foraging time is not constrained (\crefp{fig2}a).
In contrast, when resource quantity is more limited, net energy gain peaks at intermediate body mass (\crefp{fig2}b) because not enough energy can be obtained to sustain the high absolute metabolic cost of large-bodied individuals.
The scaling of foraging rate, $b_3$, does not affect these qualitative patterns.

Alternatively, when foraging time is limited, resource quality determines the form of net energy gain.
Net energy gain increases monotonically with body mass when resource quality is high (\crefp{fig2}d), as when resource quantity is high (\crefp{fig2}a).
Net energy gain may peak at intermediate body mass, however, when resource quality is not too high (\crefp{fig2}c).
Furthermore, the foraging exponent $b_3$ affects these qualitative patterns.
Our analytical results show that net energy gain peaks at intermediate body mass if and only if $b_3 < b_2$ and resource quality is within a certain range (see Appendix).

Net energy gain can also peak at intermediate body mass if the active metabolic rate is increased, because larger individuals are penalized more strongly (\crefp{fig2}e, cf.~\crefp{fig2}a).
This effect is exaggerated at higher temperatures, where again larger individuals pay a higher price (\crefp{fig2}f).
The exponent $b_3$ now dictates whether net energy gain always increases or eventually decreases with body mass.
Even if resources are unlimited, a concave foraging rate ($b_3 < 1$) means that foraging time increases with body mass.
Consequently, larger individuals are active for a longer period of time and therefore pay a higher metabolic cost.

When foraging time is fixed, higher active metabolic rate again penalizes larger individuals (\crefp{fig2}g, cf.~\crefp{fig2}d).
Higher temperature does not change the qualitative pattern in this case, but net energy gain peaks at lower body mass (\crefp{fig2}h).

Overall, we find that there may or may not be an intermediate optimal body mass when considering only how metabolic costs and resource acquisition scale with body mass and temperature.
Small individuals generally have low net energy gain because they gather little resource.
Large individuals may also have low net energy gain when resource quality and quantity are low or their metabolic costs are high.
Otherwise, larger body mass allows greater net energy gain.

%%%%%%%%%%%%%%%%%%%%%%%%%%%%%%%%%%%%%%%%%%%%%%%%%%%%%%%%%%%%%%%%%%%%%%%%%%%%%%%

\subsection*{When is warm-up completed?}

We now set aside the metabolic and foraging processes in order to focus on the warm-up phase.

The first component we look at is how the minimum temperature required for the completion of warm-up depends on body mass.
With sufficient solar radiation, an individual can always warm up successfully (Appendix).
In cold environments with limited solar radiation, however, it might not be possible for an individual to elevate its body temperature to the level needed for activity.
Here, we explore how the ability to complete warm-up depends on body mass.
For ectotherms and in the absence of wind, smaller mass is advantageous because decreasing body mass increases the surface area-to-body mass ratio and thus increases the ability to absorb more heat from the environment per unit of mass (\crefp{fig3}a, dashed line).
%However, wind elevates the minimum temperature for completing warm-up because convection has a stronger effect in cooling the surface of the body.
Unlike the wind-free case, ectotherms of intermediate mass are able to complete warm-up under colder temperatures because individuals with small body mass are penalized due to increasing laminar convection ($h$ in \crefp{eq:dTn}; \crefp{fig3}a).
Large individuals are penalized because they have higher operative temperatures (\crefp{eq:Tw}). %; without this additional cost, large becomes better (Supplementary Figures).
% For endotherms, larger mass is advantageous for the opposite reason: more of the heat generated endogenously is retained within the body
For endotherms, the minimum temperature for completing warm-up decreases with increasing body mass because more of the heat generated endogenously is retained within the body due to smaller surface area-to-body mass ratio (\crefp{fig3}a, thick line).
In fact, there is a threshold body mass where endogenous heat production entirely cancels heat dissipation (for reasonable environmental temperatures), above which the ability to complete warm-up becomes independent of body mass (\crefp{fig3}a, thick line).
We show in the Appendix that these results are robust for a range of parameter values.

The second component of the warm-up phase we look at is the duration of warm-up.
For ectotherms, the duration of warm-up always increases with body mass and decreases with temperature, as expected (\crefp{fig3}b).
A exception is for small endotherms: warm-up can be much slower because a high rate of heat dissipation results in a positive but low warm-up rate (\crefp{fig3}b, thick lines).
We found that the duration of warm-up is longer and more sensitive to body mass and temperature for ectotherms than for endotherms (\crefp{fig3}b).

We thus learn that small ectotherms can warm up at lower temperatures than small endotherms, except in wind.
When endotherms are able to warm up, however, they do so more quickly, especially for large individuals.

%%%%%%%%%%%%%%%%%%%%%%%%%%%%%%%%%%%%%%%%%%%%%%%%%%%%%%%%%%%%%%%%%%%%%%%%

\subsection*{How does the thermal performance curve change with body mass?}

We integrated the components above to see how they shape thermal performance and how thermal performance varies with body mass.
First, we looked at how the inclusion or exclusion of warm-up can affect thermal performance.
When the environmental temperature is high, warm-up is not needed and thus does not influence the net energy gain at high temperature, nor by extension the upper thermal limit (\crefp{fig4}abc).
When the environmental temperature is low, warm-up can reduce the net energy gain because a portion of time that could be spent foraging is expended on warm-up (\crefp{eq:Ed}).
Therefore, we focus here on cases where foraging time is limited.

The colder it gets, the longer it takes to warm up, the less time is left for foraging, and thus the lower the net energy gain (\crefp{fig4}bc, contrasted with \crefp{fig4}a).
Because the variation in warm-up duration is more important for ectotherms than for endotherms (for different body mass and temperature, \crefp{fig3}b), the reduction in net energy gain as temperature decreases is more significant for ectotherms than for endotherms.
In fact, warm-up time can be sufficiently long for large ectotherms that for low temperature, intermediate body mass is better (\crefp{fig4}c).

The most notable effect of resource acquisition is that the foraging exponent $b_3$ shifts the upper thermal limit in a size-specific manner.
If foraging rate is a convex function of body mass (\crefp{fig5}ac), performance breadth increases with body mass and larger individuals perform better than smaller ones at any temperature.
However, if foraging rate is a concave function of body mass (\crefp{fig5}bd), larger individuals do not gather enough resource at high temperature to offset their high metabolic costs, so smaller size is advantageous.
These findings are generally the same for endotherms and ectotherms, except that at low temperature, the warm-up process favors large endotherms and small ectotherms (cf.~\crefp{fig4}bc).

Finally, we looked at the effect of a reduction in resource availability, for example, due to habitat loss.
We found that the negative effect of resource limitation increases with body mass (\crefp{fig5}e--h; see also \crefp{fig2}b).
Reduction in resource availability not only shrinks the breadth of the thermal performance curve for large individuals, but it also decreases their performance such that small size can be advantageous along the entire temperature gradient.

Overall, our results show that no single relationship describes how thermal performance curves change with body mass.
Different conditions for resource acquisition can generate cases where larger individuals have broader, shifted, or narrower thermal performance curves relative to smaller individuals.
Although we did not aim to present an exhaustive description of our model's behavior, our results show the importance of the combined effects of thermoregulation, metabolism, and ecology on the energy budget.
