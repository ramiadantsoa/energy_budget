Tanjona Ramiadantsoa \newline
Department of Ecology, Evolution, and Behavior \newline
University of Minnesota \newline
1479 Gortner Ave Suite 140 \newline
St. Paul MN 55108, USA \newline
tel. +1 612 301 1856 \newline
e-mail: tramiada@umn.edu \newline


Patricia Morse, Managing Editor
The American Naturalist
The University of Chicago Press
1427 E. 60th Street
Chicago, IL 60637

E: amnat@press.uchicago.edu
T: 773-702-0446
F: 773-753-4247


Editor-in-Chief \newline
The American Naturalist  \newline \newline

August 3, 2016 \newline \newline

Dear Editor-in-Chief, \newline
I am submitting a manuscript entitled ``Net energy gain as a function of body mass and temperature" for your consideration for publication in The American Naturalist.

Temperature is an important abiotic factor, and with global warming, understanding how individuals respond to temperature becomes a pressing question.
In this work we develop a general model that use energy as a main currency and uses body mass, temperature to explain the performance of an individual.
This model differs for other budget models as we do not aim to fit to a specific data.
Instead, we probe the role of different processes namely, metabolism, foraging,  and thermoregulation, and how the relate to temperature and body size, in order to get a more general result.

The model is unique in how we study and include the process of warm-up and foraging--which has not received as much attention as metabolism--in shaping performance.
We validated but also report some unintuitive results on the process of warm-up and the role of resource quality.
The major contribution is that we identified some parameters that would be amenable for empirical testing and thus strengthen the feedback between theory and data.

% Understanding how an individual or population responds to different thermal regime is a crucial question in evolution, ecology, and in conservation.
% This work investigates the role of body mass in determining how the thermal performance of an adult insect, namely whether thermal performance breadth broadens, shrinks, or shifts when body mass increases.
% We built a model which uses net energy gain--the difference between energetic gain and cost--as a proxy for performance and explored how the interplay among physiological process (metabolic cost), ecological process (foraging), and thermodynamic process (thermoregulation) shapes the daily energy budget.
% The model aims to find a balance between theory and data and thus is based on well-defined and measurable parameters but also generates general insight.
%
% We found there is no single relationship on how thermal performance changes with body mass and identified various crucial processes that drive various qualitative results.
% In general, the exponent of the scaling of the foraging rate and the amount of resource available determine the upper limit of the thermal performance whereas the ability to complete warm-up determines the lower limit.
% To our knowledge this is the first model that not only investigates how warm-up process depends on body mass but also integrates the warm-up process in evaluating the performance of an individual.
% We also highlighted parameters that requires further empirical investigation.
% On one hand, the scaling of foraging rate seems to be as important as metabolic but unlike the latter has not been consistently measured.
% On the other hand, more attention should be paid for thermoregulation process for instance the conductance parameter which modulates heat exchange the body and the environment.

%Can I talk about the future? The model is a starting point for more elaborate works… can be framework for future work such as look for optimal body %size, evolution (and of species range)
The authors thank L.\ Burghardt, K.\ Niitep\~{o}ld, and E.\ Swanson for comments on the earlier versions of this article, and the Theory for Life group at the University of Minnesota for discussions.
We suggest Michael Kearney or Pryianga Amarasekare as Associate Editor.
This manuscript describes original work and is not under consideration by any other journal.
All authors have approved the manuscript and this submission.
We believe that this work fits the scope of the journal and hope that you would consider it fit for review.
We look forward to hearing from you.

Best regards,
Tanjona Ramiadantsoa,
Department of Ecology, Evolution, and Behavior, University of Minnesota
