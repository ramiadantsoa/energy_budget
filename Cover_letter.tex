Tanjona Ramiadantsoa \newline
Department of Ecology, Evolution, and Behavior \newline
University of Minnesota \newline
1479 Gortner Ave Suite 140 \newline
St. Paul MN 55108, USA \newline
tel. +1 612 301 1856 \newline
e-mail: tramiada@umn.edu \newline


Patricia Morse, Managing Editor
The American Naturalist
The University of Chicago Press
1427 E. 60th Street
Chicago, IL 60637

E: amnat@press.uchicago.edu
T: 773-702-0446
F: 773-753-4247


Editor-in-Chief \newline
The American Naturalist  \newline \newline

July xx, 2016 \newline \newline

Dear Editor-in-Chief, \newline
I am submitting a manuscript entitled xxx” for your consideration for publication in The American Naturalist.

This work investigates the role of body size in determining how the thermal performance of an adult insect changes with body size, namely whether thermal performance breadth broadens, shrinks, or shifts when body size increases.
We use the same concept as energy budget model and define net energy gain as a proxy for performance.
We used a process rather than pattern based model but also we strive to find a balance between theory and data.
Unlike classical energy budget model, we included and explored the role of thermoregulation, namely that an ectothermic or endothermic individual needs to complete warm-up otherwise foraging, in shaping the performance of an individual given its body size.
The idea is to quantify how low warm-up rate can penalize the individual.

We found there is no single relationship but each qualitative results depends on the allometric scaling of foraging rate, the quality and quantity of resources in the environment.
For ectotherms, we also found that the minimum temperature for completing warm-up decreases with body size unless wind becomes a factor, in that case intermediate body size is best.
We highlighted parameters that requires further empirical investigation.
On one hand, the scaling of foraging rate seems to be as important as metabolic but unlike the latter has not been consistently measured.
On the other hand, the conductance parameter in general modulates heat exchange the body and the environment can be crucial especially in facing global warming.

%Can I talk about the future? The model is a starting point for more elaborate works… can be framework for future work such as look for optimal body %size, evolution (and of species range)

This manuscript describes original work and is not under consideration by any other journal.
All authors have approved the manuscript and this submission.
We believe that this work fits the scope of the journal and hope that you would consider it fit for review.
We look forward to hearing from you.

Best regards,
Tanjona Ramiadantsoa,
Department of Ecology, Evolution, and behavior, University of Minnesota

Michael Kearney, University of Melbourne
Priyanga Amarasekare, University of California, Los Angeles
Kevin J. Gaston, University of Exeter
