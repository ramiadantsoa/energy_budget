\noindent
Empirical studies have reported that fitness first increases but eventually decreases as temperature increases, at least for ectotherms.
Although body mass is a key intrinsic factor that influences fitness, its role in shaping such hump-shaped thermal performance curves remains an open question.
In this work, we ask whether performance as a function of temperature broadens, shrinks, or shifts when body mass increases.
We build a model that integrates ecological (foraging), physiological (metabolism), and thermodynamic (warm-up) processes and asks how their interplay shapes the daily net energy gain, which we use as a proxy for performance.
We found that there is no single expected relationship of how the thermal performance curve changes with body mass, but foraging shapes its upper limit and warm-up ability determines its lower limit.
More generally, the model aims to fuel feedback between empirical and theoretical work by identifying important parameters and relationships amenable to empirical investigations, and by proposing how the three types of processes may influence species' geographic distribution.
