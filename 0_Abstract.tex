\noindent
Abstract: Empirical studies--at least for ectotherms--have reported that fitness first increases but eventually decreases as temperature increases.
Although body mass is a key intrinsic factor that influences fitness, its role in shaping such hump-shaped pattern remains an open question.
In this work, we ask whether the performance as a function of temperature (thermal performance curve) broadens, shrinks or shifts when body mass increases.
We build a model that integrates ecological (foraging), physiological (metabolism) and thermodynamical (warm-up) processes and asks how their interplay shapes the daily net energy which we use as a proxy for performance.
We found that there is no single expected relationship on how thermal performance curve changes with body mass but foraging shapes its upper limit and warm-up ability determines its lower limit.
More generally, the model aims to fuel feedback between empirical and theoretical works by identifying parameters and relationships amenable to empirical investigations, and by proposing how the three processes may influence species' biogeographic distribution.
