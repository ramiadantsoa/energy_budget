\section*{Material and methods}
Our model is based on the same principle as previous energy budget models (refs).
We define performance as the net energy gain that is the difference between energetic gain and cost.
Here, we focus on the general performance of adult income-breeding insects. 
The energetic gain is thus the amount energy acquired during the adult stage.
The energetic cost is the sum of metabolic cost when resting and metabolic cost when active (i.e, foraging).

A key feature of this model is that we include a warm-up phase before foraging.
When the environmental temperature is low, warm-up is for essential heterotherms (such as insects) because muscle needs to be within a certain range of temperature to be functional (refs).
We now describe how each quantity depends on body size and temperature.

\subsection*{Environment}
We consider three properties of the environment. 
First, we define environmental temperature $T_e$ as the temperature felt by the individual while inactive.
We implicitly assume that environmental temperature takes into account other factors such as humidity or wind speed and so on.
Because insects are so small, we further assume that environmental temperature does not depend on body size.

Second, we model solar radiation as in Campbell (2012).
The intensity of solar radiation at any given time of the day is \[SR = S_0 \cos(\psi) \]
where $\psi$ is the zenith angle and $S_0 = 1361 \mbox{W.m}^{-2}$ the maximum solar radiation at noon.
%Time of sunrise and sunset (when $|\psi| = 90^{\circ}$)
Solar radiation is needed to model warm-up phase (see appendix for more details on how $\psi$ depends on latitude and time of the year as well as how we obtained the time of sunrise).

Third, we assume that there is fixed quantity of resource available $R$ in gram per day.
$\rho$ represents the absolute energy gained per unit of resource mass when the cost of processing the food, egestion, and excretion are subtracted. 
Environment with poor quality can thus be obtained by low quantity or low quality.

\subsection*{Energetic cost}
\subsubsection*{Cost: resting metabolic rate}
Following Brown et al 2004, we assume that resting metabolic rate increases with body size and with temperature such that
\begin{equation} \label{eq:eb}
	e_b(z, t) = a_1 z^{b_1} e^{-E/[k (T_b(t)+ 273.15)]}
\end{equation}
where $z$ is body mass, $T_b(t)$ is the body temperature, at time $t$ in Celsius, $E$ and $k$ are respectively the activation energy and the Boltzman constant, $a_1$ is and $b_1$ are some constants which will be called respectively coefficient and exponent.
In \cref{eq:eb} $b_1$ is often suggested to be around 0.75 (refs.), which use as a default value in our analyses.

% T: I am wondering if this paragraph is really necessary, it looks like we are too much on the defensive.
%To model the effect of temperature, we could have used $Q_{10}$, which scales the change in reaction rates for an increase of 10 degree Celsius \citep{Precht1973}, but since the value of $Q_{10}$ can be chosen to match \cref{eq:eb} we prefer the first option to reduce the number of free parameters.
At rest, the body temperature of the individual matches that of the environmental \citep[e.g.,][]{Bartholomew1978} so that $T_b$ in \cref{eq:eb} can be replaced by $T_e(t)$.

\subsubsection*{Cost: active metabolic rate}
There is no `general' empirical results for active metabolic rate (but see Heinrich).
We assume the active metabolic rate has the same functional form as that of the resting metabolic rate, thus
\begin{equation} \label{eq:ea}
	e_a(z,t) = a_2 z^{b_2}  e^{-E/[k (\max(T_w(z_{th}), T_e(t))+ 273.15)]}
\end{equation}
where $T_w$ is the minimum thoracic temperature that would permit foraging.
For simplicity, we use a linear relationship that was derived by Bartholomew (refs) where
\begin{equation} \label{eq:Tw}
	T_w(z_{th}) = c_0+ c_1 z_{th}.
\end{equation}
$z_{th}$ is the mass of the thorax and $c_0$ and $c_1$ two free parameters.
Warm-up phase (see section Warm-up below) will determine whether an individual will be able to warm-up and thus forages.

The effect of temperature is the same as in \cref{eq:eb} except the use of the function $max$.
This is a rough approximation such that when the environmental temperature is too high, there is an additional cost of foraging---say the additional energy used to avoid overheating. 
The cost of foraging is naturally higher than cost of resting and is expressed by higher value of $a_2$, $b_2$, and stronger effect of temperature.

\subsection*{Gain: foraging}
%Expand about what is included in the foraging, movement, handling time,
Foraging rate $g(z)$ is the amount of resource acquired per unit of time (mass).
Foraging is a complicated process.
For simplicity, we chose a power law 
\[
	g(z) = a_3 z^{b_3}.
\] 
There is not direct empirical justification for the power law relationship.
However, measurements for other performances such as maximum distance, normal speed, and human weight lifting ability follows a power law (Peters book). 
We assume that $b_3$ is always positive to allow foraging rate to increase with body size.
When $b_3 > 1$ so that per unit of mass, large individuals gather more resources \citep[e.g.,][]{Nervo2014}.
When $b_3 < 1$, smaller individuals are more efficient, for instance the allometric exponent of the walking speed of beetles was 0.29 (Peters, Buddenbrock). 
We also assume that temperature is constant during foraging as (refs).

Finally, the rate of energy gain is  
\begin{equation} \label{eq:eg}
	e_g(z) = a_3 z^{b_3} \times \rho  = g(z) \times \rho.
\end{equation}

\subsection*{Warm-up}
Warm-up is a prerequisite for foraging when the temperature of the muscle is below its minimum value which occurs when the environmental temperature is low. 
Certain groups of insect, such as bees, dung beetles, and moths are capable of endogenously generating heat by contracting muscle against each other similar to shivering (refs).
These endothermic insects can thus warm-up without external source of heat. 
Most of the insects are however ectotherm and the only way to raise body temperature is to absorb energy from solar radiation.
The insect does not need to heat up the entire body but only the muscle in the thorax.
Foraging can only start when the temperature of the thorax reaches $T_w$ (\cref{eq:Tw}).

We thus distinguish part of the body: the thorax and the rest-of-the-body.
We further assume that the insect is shaped as an half a sphere, the thorax constitutes the interior of first half of the sphere.
The surface of the thorax and the rest-of-the-body can be easily calculated given the mass and the density of the insect (see appendix).

Two processes can influence the change in thoracic temperature $T_{th}$.
As the individual basks the surface (assumed to be the rest-of-the-body) heats up and the heat is then transfered to core by passive conductance.
In addition, endotherm can contract their fibrillar muscle to generate heat.
The heat generated is proportional to the frequency of contractions (refs).
We assume that frequency of contraction increases linearly with thoracic temperature $f[T_{th}]  = a_w T_{th}$ for $T_{th}> 0$ and 0 otherwise.

Thus, change in thoracic temperature $T_{th}$ is 
\begin{equation} \label{eq:dTh}
	\frac{dT_{th}}{dt} = \frac{1}{s z_{th}} (z_{th} e f[T_{th}] +  A_{th} K_1(T_r - T_{th}))
\end{equation}
where $s$ is the specific heat capacity, $e$ is the calories generated per contraction and per gram of muscle (refs), $A_{th}$ is total surface of the thorax, and $K_1$ the conductance between the thorax and the rest-of-the-body.
Warm-up for ectotherm is obtained by setting $a_w = 0$.

The exchange between the rest-of-the-body and the environment is based on further thermodynamic process. 
We consider two forms of convection here. 
For free convection (i.e., no wind), the change in rest-of-the-body temperature $T_r$ is
\begin{equation} \label{eq:dTr1} 
	\begin{split}
		\frac{dT_{r}}{dt} = & \frac{1}{s z_{r}} \Bigl( - A_{th} K_1(T_r - T_{th})  \Bigr)\\
			&+ \frac{1}{s z_{r}} \Bigl( A_r \left[ - c_p K_2 (T_r- T_e)^{1.25} (1/V)^{1/12}- \sigma \varepsilon T_r^4 + \sigma 					\varepsilon T_e^4  + r_3 SR  \right] \Bigr),
	\end{split}
\end{equation}
% E: latex note: https://www.ctan.org/pkg/amsmath  See the User Guide, specifically pages 4-6 for long equations
and for laminar convection
\begin{equation} \label{eq:dTr2}
	\begin{split}
		\frac{dT_{r}}{dt} = & \frac{1}{s z_{r}} \Bigl( - A_{th} K_1(T_r - T_{th}) \Bigl) \\
	 	  & + \frac{1}{s z_{r}} \Bigl( A_r \left[ - K_2 c_p  1.4 \times 0.135 \sqrt{u/V^{1/3}} (T_r- T_e) - \sigma \varepsilon T_r^4 				+ \sigma \varepsilon T_e^4  + r_3 SR  \right] \Bigr)
	\end{split}
\end{equation}
where $K_1$ is defined above, $c_p$ is specific capacity of the air. 
$K_2$ is a constant controlling convection between the body and the air \citep{Campbell2012}.
$\varepsilon = 0.935$  is the emissivity of gray body.
$u$ is wind speed.
$V$ is the volume of the insect and $A_r$ is the surface area of the rest-of-the-body (it is simply the surface of the whole body).

The last term of \cref{eq:dTr1} and \cref{eq:dTr2} is an approximation of more the detailed equation in \citet{Campbell2012}.
Here, we ignore view factors, reflected radiation and so on, and pool every source of radiation in $ \sigma \varepsilon T_e^4$ and SR. 
Parameters $r_3$ is used to scale and summarize the quantity of absorbed solar radiation.
 
\subsection*{Net energy budget}
We calculated  the energy budget during a 24-hour period.
Daily activity consist of resting, warming up and foraging.
We assume continuous activity here, thus requires only one warm-up phase.
We start the calculation at sunrise $t = 0$.
The individual starts to warm-up at $t_i$ and completes warm-up after $\tau_w$ (we use $t$ for time of the day and $\tau$ for duration).
Total foraging time can be fixed $\tau_f$ or as a function of resource availability $R$. 
In the latter case, $\tau_f = R/g(z)$.
The total energetic gain is given by
\[
	E_g(z,\tau_f) = \tau_f e_g(z).
\]
%
The total energetic cost is then
\begin{equation} \label{eq:et}
	E_t(z, \tau_f) = \int_0^{t_i} e_b(z, t) dt + \int_{t_i + \tau_w}^{t_i +\tau_w + \tau_f} e_a(z,t) dt + \int_{t_i+\tau_w+\tau_f}^{24} e_b(z, t) dt 
\end{equation}
$e_b$ is defined in \cref{eq:eb}  and $e_a$ in \cref{eq:ea}.
The energetic cost for endothermic insect is negligible and is omitted in \cref{eq:et}.

If the individual cannot reach its minimum thoracic temperature for warm-up, then it is forced to rest (assume the individual is smart).
Otherwise, the net energy gain is obtained from the  difference between energy gain from foraging and total energy expended, i.e.
\[ 
	E_n(z, \tau_f) = E_g(z,\tau_f) - E_t(z, \tau_f).
\]

\subsection*{Power law and parameter justifications}
One of the central assumptions of the model is the use of power law.
Resting metabolic cost have been shown to follow a power law, popularized by Brown although it has been reported more than 50 years ago (Peters).
Debates however evolved on whether the exponent 3/4 is the best value (refs).
Brown and others tried to explain why it is 3/4 (refs).

Assuming a power law for active metabolic rate is less empirically grounded.
Although, Heinrich measure active metabolic rate for beetles and found a power law with exponent 1.2.
Much of the work has been done at the  species level and focus on metabolic scope (refs).
This can be use for example for the detailed energy budget (e.g., Buckley)
But it does not scale how it changes with body size and temperature, metabolic scope can be cause by difference in coefficient or exponent.

Foraging rate is also not much measured.
It is definitely a complicated process and depends on resource distribution, ability to find them resources, competition...foraging theory.
What has been done is to calculate the total energetic cost and extrapolate the foraging rate (Nagy field metabolic rate).
Our intuition here is that foraging in the most general sense increases with body size.
Power law is the simplest and there are no empirical evidence for other functional form.
Yet, there are empirical data on other performance such as speed...
One study of dung beetles fitted a power law and found an exponent equal to 1-ish.
Visual and encounter rate. Morphology also matters



\begin{sidewaystable}
\caption{Values and ranges of parameter used }
\begin{tabular}{l l l l l}
\hline
Notation& Definition & Value or range & Unit & References \\ 
\hline
$a_1$ & Coefficient for resting metabolic rate  & 1 &  cal/$^{\circ}$C s mm$^2$ &  ?\citet{Heinrich1975} \\
$b_1$ & Exponent for resting metabolic rate  & 0.75 &  &  Peters, Brown,.. \\
$a_2$ & Coefficient for active metabolic rate  & 5-40 $ \times a_1$ &  cal/$^{\circ}$C s mm$^2$ &  Buckley, Heinrich... \\
$b_2$ & Exponent for active metabolic rate  & 0.75-1.25 & ... &  \citet{Heinrich1975} \\
$a_3$ & Coefficient for foraging rate  & 1 & ...  &  ... \\
$b_3$ & Exponent  for foraging rate  & 0.5-1.25 &  &  \citet{Nervo2014} \\
$c_p$ & molar specific heat of air  & 29.3 &  $\rm{J \, mol}^ {-1} \, \rm{C}^ {-1}$ & Campbell and Norman \\
u &  wind speed & 0.1 & $\rm{m \, sec}^{-1}$ & ...\\
$\delta $ & mass density & $0.15 \times 10^6$  & g  mm$^{-3}$  & ... \\
s & specific heat capacity & 3.3472 & $\rm{J g}^{-1} \rm{C}^{-1}$ & ... \\
$\sigma$ & Stefan-Boltzman constant & $5.67 \times 10^{-8}$ & ... & ...\\
e & Energy per contraction & 0.04184 & $\rm{J \, g}^{-1}$ &\citet{Kammer1974} \\
$a_w$& Frequency of contraction & 0.25 & ...& ?\citet{Bartholomew1977b}\\
$K_1$& Conductance between the thorax and the rest-of-the-body & 0.05 $c_p$ & ... & ... \\
$K_2$& Constant controlling convection & 1   & ... & Campbell and Norman \\
$\varepsilon$& Emissivity of gray body & 0.93& ...& ... \\
$r_3$  & Scale factor for the quantity of solar radiation absorbed & 0.5 & ... & ... \\
$c_0$ & Intersect for minimum temperature for activity & 28 & $^{\circ}\rm{C}$  & ....\\
$c_1$ & Slope for minimum temperature for activity & 0.75 &  & ....\\
$\rho$ &Energy density per gram of resource & 13-100 &  $\rm{J g}^{-1}$  & ... \\  %  Energy density per gram  dry dung (40-80\%) of total weight, \citet{Nibaruta1980} \citet{Gittings1998}
\hline
\label{table:1}
\end{tabular}
\end{sidewaystable}

