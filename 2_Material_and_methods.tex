\section*{Material and methods}
Our model is based on the same principle as previous energy budget models (refs).
We define performance as the net energy gain that is the difference between energetic gain and cost.
Here, we focus on the general performance of adult income-breeding insects. 
The energetic gain is thus the amount energy acquired during the adult stage.
The energetic cost is the sum of metabolic cost when resting and metabolic cost when active (i.e, foraging).

An key feature of this model is that we include a warm-up phase before foraging.
When the environmental temperature is low, warm-up is for essential heterotherms (such as insects) because muscle needs to be within a certain range of temperature to be functional (refs).
We now describe how each quantity depends on body size and temperature.

\subsection*{Environment}
We consider three properties of the environment. 
First, we define environmental temperature $T_e$ as the temperature felt by the individual while inactive.
We implicitly assume that environmental temperature takes into account other factors such as humidity or wind speed and so on.
Because insects are so small, we further assume that environmental temperature does not depend on body size.

Second, we model solar radiation as in Campbell (2012).
The intensity of solar radiation at any given time of the day is \[SR = S_0 \cos(\psi) \] where $S_0 = 1361 \mbox{W.m}^{-2}$. 
where $\psi$ is the zenith angle and $S_0 = 1361 \mbox{W.m}^{-2}$ the maximum solar radiation at noon.
%Time of sunrise and sunset (when $|\psi| = 90^{\circ}$)
Solar radiation is needed to model warm-up phase (see appendix for more details on how $\psi$ depends on latitude and time of the year as well as how we obtained the time of sunrise).

Third, we assume that during there is fixed quantity of resource available $R$ in gram per day.
$\varepsilon$ represents the absolute energy gained per unit of resource mass when the cost of processing the food, egestion, and excretion are subtracted. 
Environment with poor quality can thus be obtained by low quantity or low quality.

\subsection*{Energetic cost}
\subsubsection*{Cost: resting metabolic rate}
Following Brown et al 2004, we assume that resting metabolic rate increases with body size and with temperature such that
\begin{equation} \label{eq:eb}
	e_b(z, t) = a_1 z^{b_1} e^{-E/[k (T_b(t)+ 273.15)]}
\end{equation}
where $z$ is body mass, $T_b(t)$ is the body temperature, at time $t$ in Celsius, $E$ and $k$ are respectively the activation energy and the Boltzman constant, $a_1$ is and $b_1$ are some constants which will be called respectively coefficient and exponent.
In \cref{eq:eb} $b_1$ is often suggested to be around 0.75 (refs.), which use as a default value in our analyses.

% T: I am wondering if this paragraph is really necessary, it looks like we are too much on the defensive.
%To model the effect of temperature, we could have used $Q_{10}$, which scales the change in reaction rates for an increase of 10 degree Celsius \citep{Precht1973}, but since the value of $Q_{10}$ can be chosen to match \cref{eq:eb} we prefer the first option to reduce the number of free parameters.
At rest, the body temperature of the individual matches that of the environmental \citep[e.g.,]{Bartholomew1978}, $T_b$ in \cref{eq:eb} can be replaced by $T_e(t)$.

\subsubsection*{Cost: active metabolic rate}
There is no `general' empirical results for active metabolic rate (but see Heinrich).
We assume the active metabolic rate has the same functional form as that of the resting metabolic rate.
\begin{equation} \label{eq:ea}
	e_a(z,t) = a_2 z^{b_2}  e^{-E/[k (\max(T_w(z_{th}), T_e(t))+ 273.15)]}.
\end{equation}
$T_w$ is the minimum thoracic temperature that would permit foraging.
For simplicity, we use a linear relationship that was derived by Bartholomew such that :
\begin{equation} \label{eq:Tw}
	T_w(z_{th}) = c_0+ c_1 z_{th}.
\end{equation}
$z_{th}$ is the mass of the thorax and $c_0$ and $c_1$ two free parameters.
Warm-up phase (see section below) will determine whether an individual will be able to warm-up and thus forages.

The effect of temperature is the same as in \cref{eq:eb} except the use of the function $max$.
This is a rough approximation such that when the environmental temperature is too high, there is an additional cost of foraging---say the additional energy used to avoid overheating. 
The cost of foraging is naturally higher than cost of resting and is expressed by higher value of $a_2$, $b_2$, and stronger effect of temperature.

\subsection*{Gain: foraging}
Foraging rate $g(z)$ is the amount of resource acquired per unit of time (mass).
For simplicity, we chose a power law 
\[
	g(z) = a_3 z^{b_3}.
\] 
There is not direct empirical justification for the power law relationship.
However, measurement for other features such as maximum distance, normal speed, and human weight lifting ability follows a power law (Peters book). 
We assume that $b_3$ is always positive to allow foraging rate to increase with body size.
When $b_3 > 1$ so that per unit of mass, large individuals gather more resources \citep[e.g.,][]{Nervo2014}.
When $b_3 < 1$, smaller individuals are more efficient, for instance the allometric exponent of the walking speed of beetles was 0.29 (Peters, Buddenbrock). 
We also assume that temperature is constant during foraging as (refs).

Finally, the rate of energy gain is  
\begin{equation} \label{eq:eg}
	e_g(z) = a_3 z^{b_3} \times \varepsilon  = g(z) \times \varepsilon.
\end{equation}

%The total energy gained $E_g$ is constrained by foraging time and resource availability.
%If time is constraining, $E_g(z, t_f) = t_f \times e_g(z)$.
%If the amount of resource available $R$  is constraining, we assume that $t_f = R/g(z)$ and again use the previous relationship to calculate $E_g$.
%In addition, we limit the amount of resource R and individual can acquire which will be 250 times its own body mass. %sensitivity to 250.

% E: This is the section where the formulation is most "made up."  I think it's important at least to provide a biological motivation/justification for this functional form, e.g., suggesting scenarios of resource distribution or foraging behavior that make b_3 larger or smaller.  (And maybe acknowledging situations in which a power law is inappropriate.)


%
%In the next sections, we describe how we modeled daily energy budget as a function of environmental temperature and body size.
%We define environmental temperature as the temperature felt by the individual and include humidity, wind speed  and so on rather than the ambient temperature.
%The daily energy demand is partitioned into the energy expended: when inactive, when foraging (and assimilating food), and during warm-up.
%The daily energy gained is a function of body size, modeled as a power law, and the density of energy contained in the food. 

%\subsection*{Environment}
%% E: A small figure may be most effective here.  Show real example data from somewhere, your fancy trigonometric function, and your piecewise linear approximation.
%
%\subsubsection*{Environmental temperature $T_e$}
%The environmental temperature is modeled as a piecewise linear function.
%The environmental temperature reaches a minimum at sunrise then gradually increases and peaks in the afternoon (for simplicity we assumed it peaks exactly halfway between noon and sunset).
%After the peak, temperature decreases gradually until sunset, then decreases even more from sunset until the next sunrise.
%We assume that temperature at sunrise is the same as the temperature at the next sunrise so that during a 24-hour period  temperature initial is the same as temperature final (constant temperature is obtained  by assuming that the peak equals the temperature at sunrise). 
%
%\subsubsection*{Time at sunrise and sunset}
%Time at sunrise is obtained for \citet{Campbell2012} by looking at the zenith angle ($\psi$) at time $t$ and is
%\begin{equation}  \label{eq:psi}
%\cos(\psi)s = \sin(\phi) \sin(\delta) + \cos(\phi) \cos(\delta) \cos[15 (t- t_0)] 
%\end{equation}
%Here, $\phi$ denotes latitude, $\delta$ denotes solar declination which varies between 23.45$^\circ$ and  -23.45$^\circ$ at summer and winter solstice 
%($\delta$ as in equation 11.2 in \citet{Campbell2012} and only depends of the day of the year). 
%$t_0$ represents the time of solar noon and since we do not span across longitude, we fix $t_0 = 12$.
%
%\subsubsection*{Solar radiation}
%For simplicity, we assume that solar radiation is \[SR = S_0 \cos(\psi) \] where $S_0 = 1361 \mbox{W.m}^{-2}$. 
 
\subsection*{Warm-up}
Warm-up is a prerequisite for foraging when the temperature of the muscle is below its minimum value which occurs when the environmental temperature is low. 
Certain groups of insect, such as bees, dung beetles, and moths are capable of endogenously generating heat by contracting muscle against each other similar to shivering (refs).
These endothermic insect can thus warm-up without external source of heat. 
Most of the insects are however ectotherm and the only way to raise body temperature is to absorb energy from solar radiation.
The insect does not need to heat up the entire body but only the muscle in the thorax.
Foraging can only start when the temperature of the thorax reaches $T_w$ (\cref{eq:Tw}).
The body then thus consists of two parts: the thorax and the rest-of-the-body.
We further assume that the insect is shaped as an half a sphere, the thorax constitutes the interior of first half of the sphere.
The surface of the thorax and the rest-of-the-body can be easily calculated given the mass and the density of the insect (see appendix).
% E: give more examples, since everyone we talk to says that endothermy is a freak occurrence in insects
%Their thoracic temperature rises by absorbing solar radiation (e.g., by basking). 

Two processes can influence the change in thoracic temperature $T_{th}$.
As the individual basks the surface (assumed to be the rest-of-the-body) heats up and the heat is then transfered to core by passive conductance.
In addition, endotherm can contract their fibrillar muscle to generate heat.
The heat generated is proportional to the frequency of contractions (refs).
We assume that frequency of contraction increases linearly with thoracic temperature $f[T_{th}]  = a_w T_{th}$ for $T_{th}> 0$ and 0 otherwise.

Thus, change in thoracic temperature $T_{th}$ is 
\begin{equation} \label{eq:dTh}
	\frac{dT_{th}}{dt} = \frac{1}{s z_{th}} (z_{th} e f[T_{th}] +  A_{th} K_1(T_r - T_{th}))
\end{equation}
where $s$ is the specific heat capacity, $e$ is the calories generated per contraction and per gram of muscle (refs), $A_{th}$ is total surface of the thorax, and $K_1$ the conductance between the thorax and the rest-of-the-body.
Warm-up for ectotherm is obtained by setting $a_w = 0$.

The exchange between the rest-of-the-body and the environment is based on further thermodynamic process. 
We consider two forms of convection here. 
For free convection (i.e., no wind), the change in rest-of-the-body temperature $T_r$ is
\begin{equation} \label{eq:dTr1} 
	\begin{split}
		\frac{dT_{r}}{dt} = & \frac{1}{s z_{r}} \Bigl( - A_{th} K_1(T_r - T_{th})  \Bigr)\\
			&+ \frac{1}{s z_{r}} \Bigl( A_r \left[ - c_p K_2 (T_r- T_e)^{1.25} (1/V)^{1/12}- \sigma \varepsilon T_r^4 + \sigma 					\varepsilon T_e^4  + r_3 SR  \right] \Bigr),
	\end{split}
\end{equation}
% E: latex note: https://www.ctan.org/pkg/amsmath  See the User Guide, specifically pages 4-6 for long equations
and for laminar convection
\begin{equation} \label{eq:dTr2}
	\begin{split}
		\frac{dT_{r}}{dt} = & \frac{1}{s z_{r}} \Bigl( - A_{th} K_1(T_r - T_{th}) \Bigl) \\
	 	  & + \frac{1}{s z_{r}} \Bigl( A_r \left[ - K_2 c_p  1.4 \times 0.135 \sqrt{u/V^{1/3}} (T_r- T_e) - \sigma \varepsilon T_r^4 				+ \sigma \varepsilon T_e^4  + r_3 SR  \right] \Bigr)
	\end{split}
\end{equation}
where $K_1$ is defined above, $c_p$ is specific capacity of the air. 
$K_2$ is a constant controlling convection between the body and the air \citep{Campbell2012}.
$\varepsilon = 0.935$  is the emissivity of gray body.
$u$ is wind speed.
$V$ is the volume of the insect and $A_r$ is the surface area of the rest-of-the-body (it is simply the surface of the whole body).

The last term of \cref{eq:dTr1} and \cref{eq:dTr2} is an approximation of more the detailed equation in \citet{Campbell2012}.
Here, we ignore view factors, reflected radiation and so on, and pool every source of radiation in $ \sigma \varepsilon T_e^4$ and SR. 
Parameters $r_3$ is used to scale and summarize of absorbed solar radiation.
 
\subsection*{Net energy budget}
The 24-hour period is partitioned into rest, warm-up and foraging.
We assume continuous activity here, thus requires only one warm-up phase.
We start the calculation at sunrise $t = 0$.
The individual starts to warm-up at $t_i$ and completes warm-up after $\tau_w$ ( we use $t$ for time of the day and $\tau$ for duration).
Total foraging time can be fixed $\tau_f$ or as a function of resource availability $R$. 
In the latter case, $\tau_f = R/g(z)$.
The total energetic gain is 
\[
	E_g(z,\tau_f) = \tau_f e_g(z).
\]

The total energetic cost is then
\begin{equation} \label{eq:et}
	E_t(z, \tau_f) = \int_0^{t_i} e_b(z, t) dt + \int_{t_i + \tau_w}^{t_i +\tau_w + \tau_f} e_a(z,t) dt + \int_{t_i+\tau_w+\tau_f}^24 e_b(z, t) dt 
\end{equation}
$e_b$ is defined in \cref{eq:eb}  and $e_a$ in \cref{eq:ea}.
The energetic cost for endothermic insect is negligible and are omitted in \cref{eq:et}.

If the individual cannot reach its minimum thoracic temperature for warm-up, then it is forced to rest (assume the individual is smart).
Otherwise, the net energy gain is obtained from the  difference between energy gain from foraging and total energy expended, i.e.
\[ 
	E_n(z, \tau_f) = E_g(z,\tau_f) - E_t(z, \tau_f).
\]


%\begin{sidewaystable}
%\caption{Values and ranges of parameter used }
%\begin{tabular}{l l l l l}
%\hline
%Definition& Notation & Value or range & Unit & References \\ 
%\hline
%Conductance & $C$ & $5- 70 \times 10^{-7}$ &  cal/$^{\circ}$C s mm$^2$ &  \citet{Heinrich1975} \\
%Lowest operative temperature& $c_0$ & 28 & $^{\circ} \rm{C}$ &  \citet{Bartholomew1977}\\
%Slope of the increase in operative temperature & $c_1$ & 1.5 &  &  \citet{Bartholomew1977} \\
%Slope of increase in frequency of contraction & $a_w$ & 0.25 &  &  \\
%Specific heat & $s$ & 0.8 & cal/g $^{\circ} C$ & \\
%Muscle density & $\delta$ & $1.06 \times 10^{-3}$ & g/mm$^3$ & \\
%Energy per contraction & $u_w$ & 0.01 & cal/g &\citet{Kammer1974} \\
%Resting coefficient at at 22 $^{\circ}$ C for one gram& $a_1$ &  $10^{-4}- 6 \times 10^{-3}$ & cal/ g s & \citet{Kammer1974}\\
%& & & & \citet{Bartholomew1977b}\\
%& & & & \citet{Bartholomew1981} \\
%Allometric exponent of resting & $b_1$ & 0.75 & & \citet{Brown2004} \\
%Active coefficient for one gram & $a_2$ & $10^{-3} - 2 \times 10^{-1}$  & cal/g s & \citet{Bartholomew1977}\\
%& &  & & \citet{Bartholomew1977b}\\
%& &&& \citet{Bartholomew1981} \\
%Allometric exponent of activity &  $b_2$ & 0.75 - 1.25& &  \citet{Heinrich1975}\\
%& & &  & \citet{Bartholomew1977}\\
%& &  &  & \citet{Bartholomew1977b}\\
%& & & & \citet{Bartholomew1981}\\
%Allometric coefficient of resource collection & $a_3$ & 0.0083 xxx& g/s & \citet{Ortega-Martinez2014}\\
%Allometric exponent of resource collection & $b_3$ & 0.5-1.5& & \citet{Nervo2014} \\
%Energy density per gram of resource &$\varepsilon$ & 4 &  cal/g & \citet{Nibaruta1980} \\ % Energy density per gram  dry dung (40-80\%) of total weight
%& & & &  \citet{Gittings1998} \\
%\hline
%\label{table:1}
%\end{tabular}
%\end{sidewaystable}



% T: note that I did not explore $b_2$ but looked at $a_2$. Fixing the former is important to get some sort of background result although I think (not deep thinking) that changing $b_2$ will give the similar results by adjusting other parameters.
%Just a draft to see how things may or will be added
%\subsubsection*{Fecundity rate}
%The remaining energy is then converted to offspring according to the same power law $a_3 z^{b_4}$.
%The fecundity rate can mean the total energy needed to reach adult size.
%%I am not sure but do juvenile bees need to forage or they stay inside the hive until they are mature
%% http://www.beesource.com/resources/usda/honey-bee-life-history/
%
%%\subsubsection*{Survival}
%%It is possible to include two mortality rates both for juvenile and adult.
%%I don't think it will make much difference though (maybe wrong intuition) :D
%
%% E: Your Introduction actually sort of convinced me that simply using e_r as the performance metric may be better.  Could mention some of the complications in converting energy to fitness, but then say that many of them depend enough on the ecological context that the conversion will not be considered here.
%% E: Alternatively, perhaps survival effects are needed to get interesting-enough answers...
