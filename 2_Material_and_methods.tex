\section*{Material and methods}
The model investigates the daily performance of an adult insect with deterministic body size and is suited to compare various body sizes within monophyletic group.
We define performance as net energy gain which is the difference between energetic gain and cost.
The energetic gain is the amount energy acquired whereas  the energetic cost is the sum of metabolic cost while at resting and during activity (i.e., foraging).
The model also contains thermoregulation phase that precedes activity, this is because muscles are only operational when they are at certain temperature---when the environment is cold, warm-up phase is necessary. 
First, we describe the external environment.
Second, we show how we model energetic cost and gain as a function of body size and temperature based on empirical relationships.
Third, we used thermodynamical principle to describe changes in body temperature during warm-up.
Finally, we stitched these components altogether to define net gain. 

\subsection*{Environment}
We consider three properties of the environment. 
First, we define environmental temperature $T_e$ as the temperature felt by the individual while inactive.
We implicitly assume that environmental temperature takes into account other factors such as humidity or wind speed and so on.
Because insects are small enough, we further assume that environmental temperature does not depend on body size.

Second, we model solar radiation as in \citet{Campbell2012}.
The intensity of solar radiation at any given time of the day is \[S_R = S_0 \cos(\psi) \]
where $\psi$ is the zenith angle and $S_0 = 1361 \mbox{W.m}^{-2}$ the maximum solar radiation at noon.
%Time of sunrise and sunset (when $|\psi| = 90^{\circ}$)
Solar radiation is needed to model warm-up phase (see appendix for more details on how $\psi$ depends on latitude and time of the year as well as how we obtained the time of sunrise).

Third, we assume that there is fixed quantity of resource available $R$ in gram per day.
$\rho$ represents the energy density per unit of resource mass. %when the cost of processing the food, egestion, and excretion are subtracted. 
Environment with poor quality can thus be obtained by low quantity or low quality.

\subsection*{Energetic cost}
\subsubsection*{Cost: resting metabolic rate}
Following \citet{Brown2004}, we assume that resting metabolic rate increases with body size and with temperature such that
\begin{equation} \label{eq:eb}
	e_b(z, t) = a_1 z^{b_1} e^{-E/[k (T_b(t)+ 273.15)]}
\end{equation}
where $z$ is body mass, $T_b(t)$ is the body temperature, at time $t$ in Celsius, $E$ and $k$ are respectively the activation energy and the Boltzman constant, $a_1$ is and $b_1$ are some constants which will be called respectively coefficient and exponent.
In \cref{eq:eb} $b_1$ is often suggested to be around 0.75 \citep{Kleiber1947, Peters1986, Brown2004}, which we use as a default value in our analyses.

% T: I am wondering if this paragraph is really necessary, it looks like we are too much on the defensive.
%To model the effect of temperature, we could have used $Q_{10}$, which scales the change in reaction rates for an increase of 10 degree Celsius \citep{Precht1973}, but since the value of $Q_{10}$ can be chosen to match \cref{eq:eb} we prefer the first option to reduce the number of free parameters.
At rest, the body temperature of the individual matches that of the environmental \citep[e.g.,][]{Bartholomew1978} so that $T_b$ in \cref{eq:eb} can be replaced by $T_e(t)$.

\subsubsection*{Cost: active metabolic rate}
During activity, an individual expends more energy than rest. 
For simplicity, we assume the active metabolic rate has the same functional form as that of the resting metabolic rate, thus
\begin{equation} \label{eq:ea}
	e_a(z,t) = a_2 z^{b_2}  e^{-E/[k (\max(T_w(z_{th}), T_e(t))+ 273.15)]}
\end{equation}
where $T_w$ is the minimum thoracic temperature that would permit foraging.
Warm-up phase (see section Warm-up below) will determine whether an individual will be able to warm-up and eventually forages.
Large bodied are often have higher temperature during activity \citep{Bartholomew1977a}.
For simplicity, we use a linear relationship \citep{Bartholomew1977a}, i.e.
\begin{equation} \label{eq:Tw}
	T_w(z_{th}) = c_0+ c_1 z_{th}.
\end{equation}
$z_{th}$ is the mass of the thorax and $c_0$ and $c_1$ two free parameters.
Thus unlike resting metabolic rate, the temperature effect depends on body size.
The effect of temperature is the same as in \cref{eq:eb} except the use of the function $max$.
This is a rough approximation such that when the environmental temperature is too high, there is an additional cost of foraging---say the additional energy used to avoid overheating. 
The cost of foraging is naturally higher than cost of resting and is expressed by higher value of $a_2$, $b_2$, and stronger effect of temperature.

\subsection*{Gain: foraging}
%Expand about what is included in the foraging, movement, handling time,
Foraging rate $g(z)$ is the amount of resource acquired per unit of time (mass).
Foraging is a complicated process.
For simplicity, we assume a power law 
\[
	g(z) = a_3 z^{b_3}.
\] 
Several studies have shown that...

%\cref{fig1} illustrates different forms of the shapes (concavity and convexity) as function of the exponents.
...

However, measurements for other performances such as maximum distance, normal speed, and human weight lifting ability follows a power law (Peters book). 
We assume that $b_3$ is always positive to allow foraging rate to increase with body size.
When $b_3 > 1$ so that per unit of mass, large individuals gather more resources \citep[e.g.,][]{Nervo2014}.
When $b_3 < 1$, smaller individuals are more efficient, for instance the allometric exponent of the walking speed of beetles was 0.29 (Peters, Buddenbrock). 
We also assume that temperature is constant during foraging as (refs).

Finally, the rate of energy gain is  
\begin{equation} \label{eq:eg}
	e_g(z) = a_3 z^{b_3} \times \rho  = g(z) \times \rho.
\end{equation}

\subsection*{Warm-up}
Warm-up is a prerequisite for foraging when the temperature of the muscle is below its minimum value which occurs when the environmental temperature is low. 
However, they don't need to warm-up the entire body but only those with muscle. 
Here, we limit to the thorax thus  $T_w$ (\cref{eq:Tw}) is a function of the thoracic mass rather than the body mass.

The most common strategy is to absorb solar radiation.
Heat is transfered to the thorax from the surface of the body by passive conductance \citep{Bakken1976}.
A second strategy is to endogenously generate heat by contracting muscle against each other similar to shivering \citep[e.g.,][]{Kammer1974}.
We assume that frequency of contraction increases linearly with thoracic temperature $f[T_{th}]  = a_w T_{th}$ for $T_{th}> 0$ and 0 otherwise.
With a slight abuse of terminology, we call the first one  ectotherm and the second  endotherm. % T: this can be a bit confusing 


We pooled into a couple differential equation that tracks changes in the thoracic temperature and rest-of-the-body.
Geometrically and for simplicity we  assume that the insect/individual is shaped as an half a sphere, the thorax constitutes the interior of first half of the sphere.
The surface of the thorax and the rest-of-the-body can be easily calculated given the mass and the density of the insect (see appendix).
The distinction between ectotherm and endotherm is obtained by setting $a_w$ to zero.
comparing performance is only within group.

Change in thoracic temperature $T_{th}$ is 
\begin{equation} \label{eq:dTh}
	\frac{dT_{th}}{dt} = \frac{1}{s z_{th}} (z_{th} e f[T_{th}] +  A_{th} K_1(T_r - T_{th}))
\end{equation}
where $s$ is the specific heat capacity, $e$ is the calories generated per contraction and per gram of muscle (refs), $A_{th}$ is total surface of the thorax, and $K_1$ the conductance between the thorax and the rest-of-the-body.
Warm-up for ectotherm is obtained by setting $a_w = 0$.

The exchange between the rest-of-the-body and the environment is based on further thermodynamic process. 

\begin{equation} \label{eq:dTr} 
	%\begin{split}
		\frac{dT_{r}}{dt} =  \frac{1}{s z_{r}} \Bigl( - A_{th} K_1(T_r - T_{th})  \Bigr)
			+ \frac{1}{s z_{r}} \Bigl( A_r \left[ - c_p K_2 h(T_r -T_e, V)- \sigma \varepsilon T_r^4 + \sigma \varepsilon T_e^4  + r_3 S_R  \right] \Bigr),
%\end{split}
\end{equation}
We consider two forms of convection here where $ h(T_r -T_e, V) = (T_r- T_e)^{1.25} (1/V)^{1/12 }$ for free convection (no wind) and $ h(T_r -T_e, V) =  1.4 \times 0.135 \sqrt{u/V^{1/3}} (T_r- T_e) $ for laminar convection.

%For free convection (i.e., no wind), the change in rest-of-the-body temperature $T_r$ is
%\begin{equation} \label{eq:dTr1} 
%	\begin{split}
%		\frac{dT_{r}}{dt} = & \frac{1}{s z_{r}} \Bigl( - A_{th} K_1(T_r - T_{th})  \Bigr)\\
%			&+ \frac{1}{s z_{r}} \Bigl( A_r \left[ - c_p K_2 (T_r- T_e)^{1.25} (1/V)^{1/12}- \sigma \varepsilon T_r^4 + \sigma 					\varepsilon T_e^4  + r_3 SR  \right] \Bigr),
%	\end{split}
%\end{equation}
%% E: latex note: https://www.ctan.org/pkg/amsmath  See the User Guide, specifically pages 4-6 for long equations
%and for laminar convection
%\begin{equation} \label{eq:dTr2}
%	\begin{split}
%		\frac{dT_{r}}{dt} = & \frac{1}{s z_{r}} \Bigl( - A_{th} K_1(T_r - T_{th}) \Bigl) \\
%	 	  & + \frac{1}{s z_{r}} \Bigl( A_r \left[ - K_2 c_p  1.4 \times 0.135 \sqrt{u/V^{1/3}} (T_r- T_e) - \sigma \varepsilon T_r^4 				+ \sigma \varepsilon T_e^4  + r_3 SR  \right] \Bigr)
%	\end{split}
%\end{equation}
where $K_1$ is defined above, $c_p$ is specific capacity of the air. 
$K_2$ is a constant controlling convection between the body and the air \citep{Campbell2012}.
$\varepsilon = 0.935$  is the emissivity of gray body.
$u$ is wind speed.
$V$ is the volume of the insect and $A_r$ is the surface area of the rest-of-the-body (it is simply the surface of the whole body).

The last term of \cref{eq:dTr}  is an approximation of more the detailed equation in \citet{Campbell2012}.
Convection was obtained from the same refs??
Here, we ignore view factors, reflected radiation and so on, and pool every source of radiation in $ \sigma \varepsilon T_e^4$ and $S_R$. 
Parameters $r_3$ is used to scale and summarize the quantity of absorbed solar radiation.

We  solve the ODE numerically using NDSolve implemented in \citet{Mathematica10}.
The solution that depicts the change in thoracic temperature  ($T_{th}(t)$) is analyzed to see if the minimum temperature  required for activity $T_w$ is reached.
In that case, we can also solve the duration of the warm-up $\tau_w$ which corresponds by definition to $T_{th}(\tau_w) = T_w$. 

 
\subsection*{Net energy budget}
We now stitched all the components above to calculate  the energy budget during a 24-hour period.
Daily activity consist of resting, warming up and foraging (activity).
We assume continuous activity here, thus requires only one warm-up phase.
We start the calculation at sunrise $t = 0$.
Total foraging time can be fixed $\tau_f$ or as a function of resource availability $R$. 
In the latter case, $\tau_f = R/g(z)$.
We included warm-up into two ways.
If warm-up cannot be completed, foraging does not occur and thus $\tau_f = 0$.
If warm-up can be complete, we subtract the duration of warm-up $\tau_w$ from the total foraging time $\tau_f$. 
The total daily energetic gain is given by
\[
	E_g(z,\tau_f - \tau_w) = (\tau_f - \tau_w) e_g(z).
\]
%
If we assume that warm-up starts at $t_w$ (we use t to denote the time of the day and $\tau$ for duration), then the total daily  energetic cost is then
\begin{equation} \label{eq:et}
	E_t(z, \tau_f) = \int_0^{t_i} e_b(z, t) dt + \int_{t_i + \tau_w}^{t_i + \tau_f } e_a(z,t) dt + \int_{t_i+\tau_f}^{24} e_b(z, t) dt 
\end{equation}
$e_b$ is defined in \cref{eq:eb}  and $e_a$ in \cref{eq:ea}.
The energetic cost for endothermic insect is negligible \citep{Heinrich1975} and is omitted in \cref{eq:et}.


Otherwise, the net energy gain is obtained from the  difference between energy gain from foraging and total energy expended, i.e.
\[ 
	E_n(z, \tau_f) = E_g(z,\tau_f) - E_t(z, \tau_f).
\]

\subsection*{Power law and parameter justifications}
The central assumption here is that relationship between body size and that metabolic and between body size and foraging rate is represented by a power law.

A general pattern is that resting metabolic rate scales with body size with an exponent 0.75 \citep{Kleiber1947, Peters1986, Brown2004}.
Such value has been reported from unicellular organisms to mammals \citep{Brown2004}.
Although there is a debate about the actual values \citep[e.g.,][]{Isaac2010}, we adopt that assumption to diminish the number of free parameters and explore different values of the exponent of the other rates which are not established.

The power law relationship for active metabolic rate has less empirical ground.
Since many very few studies have spanned the measurement for a range of body size that look specifically to active metabolic rate.
A notable example is the one by \citet{Bartholomew1978} who found a power law with exponent $b_2 = 1.17$.
In general, what is known is specific values of the metabolic scope which is the ratio between maximum active metabolic and resting metabolic rates.
Several studies have reported that oxygen consumption can be as high as 70 times than resting although a typical values is about 40 \citep{Bartholomew1981}.  
In our model, metabolic scope can then be adjusted either by varying $a_2$  or by $b_2$ whether it is mass independent or not. 

Recent studies have explored rate of energetic gain and recovered a power law relationship \citep{Maino2015, Pawar2012}.
There seems to be no single exponent $b_3$.
For instance, its value can depend on dimension of the search space with a value of 0.85 for a two-dimensional search space to 1.06 in three-dimensional search space \citep{Pawar2012}.
Body size can actually influence other processes, for instance walking speed  can scale with a power 0.29 \citep{Peters1986}, with dominance competition excerted by larger individual, exponent might also scale superlinearly.
Our goal however is not to use argue about the homogeneity of these values but instead to explore the consequences of their heterogeneities.
\cref{fig1} depicts how the exponent influences the shape of the power law.

Another form that can be used to model the effect of temperature is the use of $Q_{10}$ which is the change in rate with a 10 $^ {\circ} \rm{C}$ increase in the body temperature \citep{Precht1973}.
We however opted for the Arhenius coefficient by \citet{Brown2004} because it reduces the number of free parameter but also it approximates the temperature effect for $Q_10 = 2.45$.  
Finally, to  insure that the coefficients are not different, the model is best for species with narrow taxonomical scope and thus restricted to monophyletic group, or even within species. 

\cref{table1} summarize different parameter values we used and explored in this work.

\begin{sidewaystable}
\caption{Values and ranges of parameter used }
\begin{tabular}{l l l l l}
\hline
Notation& Definition & Value or range & Unit & References \\ 
\hline
$a_1$ & Coefficient for resting metabolic rate  & 1 &  cal/$^{\circ}$C s mm$^2$ &  ?\citet{Heinrich1975} \\
$b_1$ & Exponent for resting metabolic rate  & 0.75 &  &  Peters, Brown,.. \\
$a_2$ & Coefficient for active metabolic rate  & 5-40 $ \times a_1$ &  cal/$^{\circ}$C s mm$^2$ &  Buckley, Heinrich... \\
$b_2$ & Exponent for active metabolic rate  & 0.75-1.25 & ... &  \citet{Heinrich1975} \\
$a_3$ & Coefficient for foraging rate  & 1 & ...  &  ... \\
$b_3$ & Exponent  for foraging rate  & 0.5-1.25 &  &  \citet{Nervo2014} \\
$c_p$ & molar specific heat of air  & 29.3 &  $\rm{J \, mol}^ {-1} \, \rm{C}^ {-1}$ & Campbell and Norman \\
u &  wind speed & 0.1 & $\rm{m \, sec}^{-1}$ & ...\\
$\delta $ & mass density & $0.15 \times 10^6$  & g  mm$^{-3}$  & ... \\
s & specific heat capacity & 3.3472 & $\rm{J g}^{-1} \rm{C}^{-1}$ & ... \\
$\sigma$ & Stefan-Boltzman constant & $5.67 \times 10^{-8}$ & ... & ...\\
e & Energy per contraction & 0.04184 & $\rm{J \, g}^{-1}$ &\citet{Kammer1974} \\
$a_w$& Frequency of contraction & 0.25 & ...& ?\citet{Bartholomew1977b}\\
$K_1$& Conductance between the thorax and the rest-of-the-body & 0.05 $c_p$ & ... & ... \\
$K_2$& Constant controlling convection & 1   & ... & Campbell and Norman \\
$\varepsilon$& Emissivity of gray body & 0.93& ...& ... \\
$r_3$  & Scale factor for the quantity of solar radiation absorbed & 0.5 & ... & ... \\
$c_0$ & Intersect for minimum temperature for activity & 28 & $^{\circ}\rm{C}$  & ....\\
$c_1$ & Slope for minimum temperature for activity & 0.75 &  & ....\\
$\rho$ &Energy density per gram of resource & 13-100 &  $\rm{J g}^{-1}$  & ... \\  %  Energy density per gram  dry dung (40-80\%) of total weight, \citet{Nibaruta1980} \citet{Gittings1998}
\hline
\label{table1}
\end{tabular}
\end{sidewaystable}

