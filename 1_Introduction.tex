\section*{Introduction}
Niche is one of the most important concepts in ecology.
Niche was popularized by \citet{Hutchinson1957} who described species' niche as a hypervolume in a n-dimensional space.
Moving from Hutchinson to concrete description of the niche is nontrivial.
In some cases, niche axis is can be clear especially for abiotic variables like temperature or humidity but what the other dimensions are made of can be controversial.
The concept was set aside by ecologists but recently started to play a more dominant role again % (or revival, resurgence).. 
This is partly due terminological clarifications, for instance the distinction between the Grinellian and Eltonian niche \citep{Chase2003} but also because of the crucial need to predict species' response to current rapid human-induced environmental changes \citep[e.g.,][]{Parmesan2006, Kearney2009}. %one can think about environmental filtering too
The geometrical abstraction by \citet{Hutchinson1957} was particularly useful to illustrate one key property of the niche: its breadth.
Niche breadth can however take two forms whether it is the fundamental or realized niche.
The realized niche has been more dominant in both theoretical and empirical works, yet understanding fundamental is a foundation of the realized niche.

%In fact species with narrow niche breadth is more vulnerable to environmental changes \citep{Henle2004}.

%another way to present is to talk about spectrum, this is at the one end, although not necessarily true for 
The principal goal in empirical study is to quantify the performance of a given species along few niche axes.
There are generally three main approaches \citep{Holt2009}.
The first approach consists of placing individuals in at different environmental gradients and measure their performance (not necessarily fitness) \citep{Birch1953, Elliott1982, Angert2005, Frazier2006}.
As a function of the experiment, it can give the fundamental or realized niche \citep{Birch1953, Elliott1982, Angert2005, Frazier2006}.
%The problem is that it is quite tedious to do (can be slow, requires more resources, applicable to only few species
The second approach consists of a highly detailed mechanistic model and try to combine details of processes and data at the molecular level, thermodynamic, physiological and behavioral to determines the performance \citep{Kooijman2009, Kearney2009, Buckley2008}.
This approach generally focus on the fundamental niche.
%approach is to mechanistically construct how an individual trait interacts with the environment using a bottom-up approach %Marr-Pirt or Monod model for early energy budget model
%The main approach is called energy budget model and is based on detailed physiological, thermodynamic and behavioral processes. 
%The main problem with these two approaches is that they are tedious and applies to few well-studied species.
%called brute force by Holt, some key parameters are still needs to be measured and is not often easy to get them (Nisbet2000). 
The third approach is to use statistical methods that infers the niche from correlation between species occurrence and the environmental conditions \citep{Guisan2005, Austin2007, Elith2009}. %also Thullier2008 
Unlike the previous two, this last approach is less data greedy and applicable to a wide variety of species.
It is suited to tackle conservation problems.
In these approach, there is often a striking discrepancy between the depth of comprehension and data.
The mechanistic probing can be tedious whereas third approach gives a correlation and tells little about the fundamental niche.

 Theoretical studies are often exclusive to the property of the realized niche. 
 %Theoretical studies put emphases on the realized niche especially competition (resource or niche partitioning) and there are a myriad of approaches (often unlinked) depending on which phenomenon is under investigation.  
The broken stick model is used to explain species composition (abundance or relative abundance) of communities \citep{MacArthur1957}.
Yet, it does not tell what is actually the niche.
Linking it to Hutchinson is difficult.
%Although it has explanatory power, it does not tell how species interact with the environments. focus on partitioning the abstract niche rather than what is actually the niche axis. Oblivious to what niche axis actually are.
Studies of coexistence often seeks mechanisms that would allow species to partition the niche in space such as the competition colonization trade-off \citep[e.g.,][]{Levins1971,Tilman1994} or partition the niche in time such as storage effect \citep{Skellam1951, Chesson2000}. %note that storage-effect also applies in space but the temporal aspect is the easiest 
Other models focus on the strength of competition.
For instance, assuming a given trait has Gaussian resource utilization curve, competition  between two traits is quantified by the amount of overlap between their curves \citep{MacArthur1967, Roughgarden1979}.% and lead to principle of limiting similarities.
In that latter, fundamental niche is given for granted and used to study competition.

%Finally, some other models investigates the evolution of niche width based on quantitative genetic framework \citep[reviewed in][]{Futuyma1988}.
In the early age, Levins that they will be entangled not to be able to separate.
There is a divide between theory and data (Angilletta2009).
Mechanistic model such as DEB uses theory but they are too data dependent.
In times, the parameters cannot be measured and thus limit their applicability.
In addition the great level of processes prevents the understanding of how the processes influence the performance.
It was shown that they can fit data very well.
In fact that is the goal.
Theory on the other hand, surf too much on the abstraction.
Yet there is no general theory about fundamental niche.

Our goal in this work is to develop a model to study fundamental niche.
A specific question we try to answer is to compare performances along environmental gradient for body trait.
USE Futuyma here about physiological and behavioral....
The point is to find a balance between generality and realism.
In the latter we choose two of the most important variables: body size (trait) and temperature (niche axis).
Practically, there is a vast literature on the relationship between body size, temperature and the physiological and behavioral processes.
These knowledge can be used to select and root modeling assumptions to empirical data and thus brings a degree of realism.
We use the same approach as energy budget model but put emphasizes on the role of the processes rather than fitting empirical patterns.
General pattern.
We can then answer a specific question such as: what condition niche breadth (or thermal performance breadth) increases or decreases as a function of body size.

%To produce general results, theoretical models often make simplifying assumptions for mathematical convenience.
%In niche models that actually consider a relationship between the trait and say performance such as in \citet{Roughgarden1979}, the assumptions are often violated.
%For instance, resource utilization curve is Gaussian but there is no evidence that say large beak size cannot exploit small seed as efficiently as large ones. % do we need reference here
%The symmetrical nature of the niche is not necessarily true.
%Experimental studies  along temperature gradient shows a left skewed performance curve \citep{Angilletta2009}.
%Finally, a trade-off in performance is not empirically verified (refs). %find that refs, otherwise use warmer is better.
%In general, this incongruence generated a gap between theoretical and empirical works \citep{Amarasekare2003}.
%
%In this work, we develop a theoretical models to reconstruct the niche; more precisely a physiological and behavioral model that derives the fundamental niche for given a trait and along an environmental gradient.
%The conceptual goal is to find a balance between realism and abstraction.
%In doing so, we choose two of the most important variables: temperature as niche axis and body size as the focal trait.
%The model is for adult insects...%T: not sure if there is a need to place it the intro 
%Practically, there is a vast literature on the relationship between body size, temperature and the physiological and behavioral processes.
%These knowledge can be used to select and root modeling assumptions to empirical data and thus brings a degree of realism.
%We use the same approach as energy budget model but put emphasizes on the role of the processes rather than fitting empirical patterns.
%We want to derive a null model for performance as a function of body size and answering a specific question that is: under what condition niche breadth (or thermal performance breadth) increases or decreases as a function of body size.
%
%MEETING NOTES
% Data driven and then pure theory....competition, fundamental...
% Expand, what is included in foraging...
% cost of shivering
% why are parameters of interest.
% fig for limited time
% emphasize generality/realism.
% Model parameters...discussion first paragraphs into methods. 
% not say it all depends. highlight the important ones.