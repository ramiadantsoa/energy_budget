\section*{Introduction}
A principal goal in ecology is to describe and understand processes that influence species distribution.
The act of describing and explaining is not often mutually inclusive.
In some cases, things fit naturally.
Abiotic factors can easily correlate with species range.
Temperature isocline matches range limits which provides evidences of range limitiation \citep{Root1988}.
Biotic factors also plays a role in the shaping species range \citep{Krebs2000}.
Various types of interaction within and among trophic level can both expand or contract species distribution.
In the most general case, it is not evident which factor is most important.
Sometimes, several factors are correlated and disentangling the relative effect of these  factors is a difficult task.

A surrogate approach is to flip the question around; choose a factor and see how it affects the performance of the species, given everything else is equal.
Temperature is one abiotic that has been investigated in depth.
Controlled experiments have been conducted to see how species responds to gradient of temperature.
It is often complicated to directly measure fitness and instead rely on proximate variables (performance) such as locomotion, growth rate, assimilation, fecundity rate \citep[][and reference therein]{Angilletta2009}.
A classical results is that thermal performance is non monotonic and is skewed to the left meaning that the optimal temperature (peak of the curve) is closer to the critical maximum temperature for positive performance than the minimum temperature. 
Whereas the approach looking the marginal effect of one factor is difficult of empirical study (it can be logistically difficult to keep everything else equal), it is the basic philosophy behind a theoretical approach.

Theoretical works have investigated the role of temperature in shaping performance especially in growth \citep{}. % Kozlowski, vanderhave  
Recently theoretical work has underpinned mechanisms that look directly at fitness \citep{Amarasekare2012}.
The model  breaks down thermal performance (intrinsic growth rate)  into three components: development, fecundity, and mortality.
The functional shapes of each of these components thus determines the basic properties of the curve.
Not only, the left-skewed is recovered but the model provides an explanation of why it is left-skewed--because of how the three component behaves. 
A different model that dig deep into physiological principles is called Dynamic Energy Budget (DEB) \citep{Kooijman2009}.
The model focuses on how (food) energy is assimilated and then allocated to different needs such as growth, metabolic cost (maintenance), reproduction \citep{Kooijman2009}.
When species data are available, these models are excellent in reproducing the pattern. 

Such model framework helps us understand that performance curves and how they differs  among individuals, but it does not provide a framework for understanding the intrinsic causes of those differences.
One variable that might underlie those differences is body size.
Empirical data have shown that body size is often associated with temperature.  
For instance, individual grows larger under colder thermal regime \citep{Van1996}.
Metabolic theory of ecology links with one equation body size and temperature \citep{Brown2004}.
There is even a pattern at global scale, the Bergman's rule which states that body size tends to increase with decreasing temperature \citep{Bergman1847}. 
Theoretical study tends to look at the role of body size without taking into account explicitly temperature.
The question thus remains, given everything else is equal how does thermal performance curve changes with body size?

A lot of ecological process are influenced by body size \citep{Peters1986}.
Of the most well known is metabolic rate is known to scale with body size by following a power law.
Many of empirical and theoretical studies have attempted to get the correct relationship.
An important process that is often overlooked is the role of behavioral thermoregulation for performance. 
For cold blooded animal, thermoregulation is an important part of the daily activity.
In warm environment, species thermoregulate to avoid overheating.
For instance, some insects avoid overheating during flight by dissipating excess heat from the thorax to the abdomen \citep{Verdu2012}.
In cold environment, some individuals also regulates avoid losing heat.
In a cold environment, some insects have the opposite strategy that insulate the thorax from the rest of the body to keep heat \citep{Verdu2012}.
Muscle needs to be at certain temperature to function properly thus when the environment is below that temperature, an individual needs to do warm-up. 
For cold blooded animal, the general strategy is to bask under the sun and absorb the heat.
In this case, body size have a direct influence as it mediates heat absorption (conductance) and in general exchange via surface-area to body size ratio.
Buckley included warm-up process of a species of lizard but the role of such process in shaping thermal performance in general level has not been explored. 

In this work we build a theoretical model to investigate how thermal performance varies with body size for cold blooded animals. 
Our approach is to look at the effects of three processes: physiological processes based on metabolism, ecological factor which depends on resource availability and foraging, and thermodynamical factor, thermoregulation processes.
The philosophy of the model is the same as energy budget model, here we define performance as difference between total energetic gain and energetic cost.
We narrow our taxonomic scope to insect, more precisely to adult insect with deterministic body size and income breeder. 
In general, we found the metabolism plays a secondary role in shaping thermal performance.
Instead, resource availability and allometric scaling of foraging are key in defining the upper thermal limit whereas thermoregulation ability of warm-up sets the lower thermal limit.
