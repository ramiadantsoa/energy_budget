\section*{Introduction}
The principal goal in ecology is to describe and understand processes that influence the distribution of species.
The distribution can be affected by a myriad of abiotic factors such as temperature, humidity, light  and  biotic factors such as competition, predation and so \citep{Krebs2000}.
In the most general case, it is not evident which factor is most important because these factors are correlated and disentangling their relative effects is a difficult task.

A surrogate approach is to flip the question around: choose a factor and see how it affects the performance of the species---given everything else is equal.
Temperature is one abiotic that has been investigated in depth.
Numerous empirical studies have been conducted to investigate how species respond to temperature gradient \citep[e.g.,][]{Angilletta2009}.
A direct measure fitness is often complicated and studies instead rely on proximate variables (performance) such as locomotion, growth rate, assimilation, fecundity rate \citep[][and reference therein]{Angilletta2009}.
A classical results is that thermal performance is a non-monotonic curve and is skewed to the left meaning that the optimal temperature (peak of the curve) is closer to the critical maximum temperature for positive performance than the minimum temperature. 
Such pattern for instance has lead to the claim that species in the tropics might be more vulnerable to global warming because the environmental temperature which closer to their optimal  is also  closer to the critical maximum temperature \citep{Deutsch2008}.

In general, measuring the effect of one factor and keep everything else equal  is difficult to do empirically.
In contrast, simplification is the central tenets of theoretical work.
Theoretical works have investigated the role of temperature in shaping performance especially in growth \citep{VandH1996, Kozlowski2004}.
Recently, a theoretical model by \citet{Amarasekare2012} the fitness of an individual as a function of temperature \citep{Amarasekare2012}.
The model  breaks down thermal performance (intrinsic growth rate)  into three components: development, fecundity, and mortality.
The functional shapes of each of these components thus determines the basic properties of the curve.
The model is able to explain the left-skewedness depending on how the three component intersects.
A different model that dig deep into physiological principles is called Dynamic Energy Budget (DEB) \citep{Kooijman2009}.
The model focuses on how (food) energy is assimilated and then allocated to different needs such as growth, metabolic cost (maintenance), reproduction \citep{Kooijman2009}.
When and where species data are available, these models are excellent in reproducing the pattern. 

Such models help to understand the qualitative and quantitative shape of the performance curves and how they differs  among individuals, but it does not provide a framework for understanding the intrinsic causes of those differences. % E: I like this.  Might just need to clarify that "development curve" etc. is not an "intrisic cause".
One variable that might underlie those differences is body size.
Empirical data have shown that body size is often associated with temperature.  
At global scale, the Bergman's rule which states that body size tends to increase with decreasing temperature \citep{Bergman1847}.
At the physiological level, individual grows larger under colder thermal regime \citep{Van1996}.
Metabolic theory of ecology links in one equation body size and temperature \citep{Gilliolly2001,Brown2004}.
Theoretical study tends to look at the role of body size without taking into account explicitly temperature \citep[e.g.,]{Yodzis1992, Brown1993}.
The question thus remains, given everything else is equal, how does thermal performance curve change with body size?.

Body size influences many of ecological processk \citep{Peters1986}.
%One of the most well known is that resting metabolic rate scales with body size by following a power law \citep{Kleiber1947, Peters1986, Brown2004}.
%Many empirical and theoretical studies have attempted to obtain the correct relationship, namely the universality of the 3/4 law \citep{West1997, Kozlowski1997, Isaac2010}.
When we consider performance (fitness), an important process that is often overlooked that is the role of behavioral thermoregulation in shaping performance. 
A particular example is warm-up phase for ectotherm.
Because muscle needs to be at certain temperature to function properly, when the environment is below that temperature, warm-up is necessary (refs). 
In some communities like dung beetles, it is crucial to arrive on site before resource quality degrades or get depleted %(Anderson & Coe 1974, Horgan 200,  Heinrich & Bartholomew, 1979 ).
The ability to warm-up has been hypothesized to cause diel partitioning as well as foraging behavior: e.g. pitching vs. active searching (refs).
Body size have a direct influence as it mediates heat absorption (conductance) and in general exchange via surface-area to body size ratio.
Most studies on warm-up look at equilibrium temperature and not the temporal and even less the dependency on body size, and even less its role in shaping thermal performance has not been explored.
 
In this work we build a theoretical model to investigate how thermal performance curve varies with body size.  % E: Perhaps a bit too broad here, given that you soon after narrow the scope to some insects.
We define performance as net energy gain which is the difference between total energetic gain and energetic cost. 
Our approach is similar to other as energy budget model \citep[e.g. Niche Mapper or]{Kooijman2009} but we are specifically interested in find a balance between abstraction and realism.
Whereas the model is general enough,  we narrow our taxonomic scope to insect and is suited to adult insect with deterministic body size and income breeder. 
We look at the effects of three processes: physiological processes based on metabolism, ecological factor which depends on resource availability and foraging, and thermodynamical factor, thermoregulation processes.
% E: This following point great!!
In general, we found the metabolism plays a secondary role in shaping thermal performance.
Instead, resource availability and allometric scaling of foraging are key in defining the upper thermal limit whereas thermoregulation ability of warm-up sets the lower thermal limit.
