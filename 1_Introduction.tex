\section*{Introduction}
Temperature is one of the most crucial abiotic factors and influences species' performance such as growth, fecundity, mortality, and in general fitness.
The thermal performance curve describes how such performance changes across temperature gradient.
At the present time, inferring the thermal performance curve of a species is a critical information in assess its vulnerability to global warming.
For instance, species in the tropics seems to have narrower thermal performance breadth and consequently are more vulnerable to an increase in temperature \citep{Deutsch2008}.

For ectotherms, a classical empirical result is that the thermal performance is a non-monotonic curve and is skewed to the left, meaning that the optimal temperature (peak of the curve) is closer to the critical maximum temperature than to the critical minimum temperature \citep[e.g.,][]{Angilletta2009}.
Recently, a theoretical model reconstructed the fitness of an individual as a function of temperature \citep{Amarasekare2012}.
The model breaks down thermal performance (intrinsic growth rate) into three components: development, fecundity, and mortality.
The functional shapes of each of those components and how they intersect thus determine the basic properties of the curve (the optimium, minimum, maximum, and skewness).
Dynamic Energy Budget (DEB) models, in contrast, dig deep into physiology \citep{Kooijman2009}.
The general framework focuses on how (food) energy is assimilated and then allocated to different needs such as growth, metabolic cost (maintenance), and reproduction \citep{Kooijman2009}.
When species'  data are available, these models have great power in reproducing patterns, say body size or number of offspring through time \citep{Nisbet2000}.

Such models help to understand and describe the qualitative and quantitative shapes of performance curves and how they differ among individuals, but they do not provide a framework for a deeper investigation of those differences.
One intrinsic cause that might underlie variability in performance curves is body size.
Empirical data have shown that body size is often associated with temperature.
At the global scale, Bergmann's rule states that body size tends to increase with decreasing temperature \citep{Bergmann1847}.
At the physiological level, an individual grows larger under a colder thermal regime \citep{Van1996}.
The metabolic theory of ecology links in one equation body size and temperature \citep{Gillooly2001}.
Theoretical studies that link body size and performance do not account for the explicit effect of temperature \citep[e.g.,][]{Yodzis1992, Brown1993}.
Despite these general results, an open key question is still: how does thermal performance curve change with body size?

Body size influences many physiological and ecological processes \citep{Peters1986}.
For instance, resting metabolic rate scales with body size by following a power law \citep{Kleiber1947, Peters1986, Gillooly2001,Brown2004}.
When we consider performance (fitness), an important process that is often overlooked is the role of behavioral thermoregulation in shaping performance \citep{Kearney2009b}.
A particular example is the warm-up phase for ectotherms, especially insects.
Because muscle needs to be at a certain temperature to function properly, when the environment is below that temperature, warm-up is necessary \citep[e.g.,][]{Heinrich1975}.
In some species like dung beetles, the duration and completion of warm-up is important as it is crucial to arrive on site before resource quality degrades or gets depleted \citep{Hanski1991}. %(Anderson & Coe 1974, Horgan 200,  Heinrich & Bartholomew, 1979 ).
The ability to warm up has further been hypothesized to cause diel partitioning, which facilitates coexistence \citep{Viljanen2009}.
Intuitively, large ectotherms warm up at a lower rate than smaller ones because large individuals have a smaller surface area-to-body size ratio and thus a reduced capacity to transfer heat to the body.
The warm-up aspect of thermoregulation, its dependence on body size, and especially its role in shaping thermal performance has not been explored.

In this work, we build a theoretical model to investigate how the thermal performance curve varies with body size.
We define performance as net energy gain, which is the difference between total energetic gain and energetic cost.
Our approach is similar to other energy budget models \citep[e.g.,][]{Kooijman2009}, but the goal is not to fit a particular species.
Instead, we look at the effects of three processes in shaping net energy gain: physiological processes of metabolism, ecological processes of resource availability and foraging, and thermodynamic processes of thermoregulation and warm-up.  % E: Perhaps I took too great a liberty here.  But it is good to aim for parallel structure in a complicated sentence like this. T: you are restraining yourself too much :D
The model is conceptually valid for heterotherms, but to include more explanatory detail, we narrow our taxonomic scope to insects.
In particular, the model is best suited for fully-grown, adult insects that are income breeders such as dung beetles and bees.
In general, we find that metabolism plays a secondary role in shaping thermal performance.
Instead, resource availability and allometric scaling of foraging are key in defining the upper thermal limit, whereas the ability to warm-up sets the lower thermal limit.
