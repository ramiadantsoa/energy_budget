\section*{Introduction}
Niche is one of the most important concepts in ecology.
Niche links how an individual interacts with its external environment and thus determines its temporal and spatial distribution and abundance.
The niche concept became popular with the work of Hutchinson who defined species' niche as a hypervolume in an n-dimensional space (ref).
Although the exact meaning of what a dimension is unclear (e.g., Abrams, Chase and Leibold), the geometrical abstraction helps to visualize two important properties of the niche.
First, the location (center) of the hypervolume in the space that defines where performance is maximal.
Second, the width of the hypervolume that defines the range of conditions an individual can persist in. 
Defining these two properties for a given species has become a central focus for both empirical and theoretical works.


Empirical studies derived the niche of the species by correlating current occurrence with the environmental conditions it occurs. 
This approach is one of the most important tools for predicting the effect of climate change on species' future distribution and persistence simply by intersecting the future environmental conditions with the previously measured niche (refs.).
A more direct empirical approach is to measure experimentally the performance of individuals (refs).
In that approach, one focuses usually one dimension---a slice of the hypervolume.
One of the best example is to measure performance along temperature gradient. 
In general, species responds in a non-monotonic way to temperature and is often skewed to the left (refs).

At the other end of the spectrum, mathematical models make assumptions about niche rather than deriving it. 
MacArthur and Levins assumed that resource utilization curve follows a normal distribution, such choice is often for mathematical convenience although the best statistical fit for some performances is a normal distribution (Angiletta2006). 
A good mix between theoretical and empirical approach is to use energy to construct the niche (refs).
These energy budget models aim to quantify the net energy gain of an individual by modeling the energetic gain through foraging minus the cost for maintenance and survival.
It is mechanistic and realistic in the sense that energy is modeled even at the molecular level and use first principles in thermodynamics, morphology, physiology and ecology.
These principles often have been validated or derived from empirical results (e.g., Peters, Brown).
Using energy is used as a main currency to understand the niche is a first good approximation since an individual with a negative gain will not persist in that environment.

Current energy budget model contains highly detailed mechanisms and often focus on understanding particular species (refs).
As consequence they only apply to well-studied species that have enough data to feed the model.
Buckley (ref) developed an ecophysiological model to predict future distribution of two lizards (\textit{Sceloporus undulatus} and \textit{S. gracious}).
In that model data relating to resting metabolic rate, metabolic scope, velocity, average egg production per year, energy required to produce eggs and more are available.
Although these models are able to reproduce empirical patterns, the role of the numerous variables influencing the pattern is unknown.
These models are thus pattern rather than process-oriented models.
%T: I am bit hesitant to `critize' the model simply because they are so complicated.  A program called Niche Mapper is actually patented and I cannot find documentation and modeling assumptions. I am wondering if you can find a way of expressing my problem in a respectable way? 

% T: I am also not very happy with the structure of the following paragraph. Not sure what to include? I would like to keep it as simple as possible and pour the needed details in the Material and methods.
In this work, we build an energy budget model that give emphasizes to role of the processes that shapes the niche rather than on specific species.
We chose body size as the intrinsic trait, temperature as the abiotic factor, and resource density as the biotic factor.
Body size and temperature are one of the most important ecological traits and abiotic factors, respectively (refs).
In addition, body size and temperature correlates the metabolic cost of an individual (refs).
Resource availability is the main driver of energetic input. 
Our goal is to investigate the role of physiological, ecological and behavioral mechanisms as well as the parameter values they take in defining to the two main characteristics of the niche that is the niche breadth and the location of the optimal conditions.
In practice, we ask how niche vary as a function of the body size, under what conditions niche breadth becomes broader or narrower as a function of body size and temperature, what conditions will maximize the performance for a given body size.
