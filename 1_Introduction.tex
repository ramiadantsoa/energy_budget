\section*{Introduction}
The principal goal in ecology is to describe and understand processes that influence uneven distribution of species.
In some cases, one can obtain clear answers say when abiotic factors correlate with species range.
For instance, one evidence is when temperature isocline matches  range limits \citep{Root1988}.
Evidence for the role of biotic factors in the shaping species range can be less conspicuous \citep{Krebs2000}.
Various types of interaction within and among trophic level can both expand or contract species distribution.
In the most general case, it is not evident which factor is most important because several factors are correlated and disentangling the relative effect of these factors is a difficult task.

A surrogate approach is to flip the question around: choose a factor and see how it affects the performance of the species---given everything else is equal.
Temperature is one abiotic that has been investigated in depth.
Numerous empirical studies have been conducted to investigate how species respond to temperature gradient \citep[e.g.,][]{Angilletta2009}.
A direct measure fitness is often complicated and studies instead rely on proximate variables (performance) such as locomotion, growth rate, assimilation, fecundity rate \citep[][and reference therein]{Angilletta2009}.
A classical results is that thermal performance is a non-monotonic curve and is skewed to the left meaning that the optimal temperature (peak of the curve) is closer to the critical maximum temperature for positive performance than the minimum temperature. 
Such pattern for instance has lead to the claim that species in the tropics are more vulnerable to global warming because the environmental temperature which closer to their optimal  is also  closer to the critical maximum temperature \citep{Deutsch2008}.

In general, measuring the effect of one factor and keep everything else equal  is difficult to do empirically.
In contrast, simplification is the central tenets of theoretical work.
Theoretical works have investigated the role of temperature in shaping performance especially in growth \citep{VandH1996,Koslowski2004}.
Recently, a theoretical model by \citet{Amarasekare2012} the fitness of an individual as a function of temperature \citep{Amarasekare2012}.
The model  breaks down thermal performance (intrinsic growth rate)  into three components: development, fecundity, and mortality.
The functional shapes of each of these components thus determines the basic properties of the curve.
Not only, the model is able to explain the left-skewedness depending on how the three component intersects. 
A different model that dig deep into physiological principles is called Dynamic Energy Budget (DEB) \citep{Kooijman2009}.
The model focuses on how (food) energy is assimilated and then allocated to different needs such as growth, metabolic cost (maintenance), reproduction \citep{Kooijman2009}.
When species data are available, these models are excellent in reproducing the pattern. 

Such models help to understand the qualitative and quantitative shape of the performance curves and how they differs  among individuals, but it does not provide a framework for understanding the intrinsic causes of those differences.
One variable that might underlie those differences is body size.
Empirical data have shown that body size is often associated with temperature.  
At global scale, the Bergman's rule which states that body size tends to increase with decreasing temperature \citep{Bergman1847}.
At the physiological level, individual grows larger under colder thermal regime \citep{Van1996}.
Metabolic theory of ecology links in one equation body size and temperature \citep{Brown2004}.
Theoretical study tends to look at the role of body size without taking into account explicitly temperature \citep[e.g.,]{Yodzis1992, Brown1993}.
The question thus remains, given everything else is equal how does thermal performance curve changes with body size?

Body size influences many of ecological process \citep{Peters1986}.
One of the most well known is that resting metabolic rate scales with body size by following a power law \citep{Kleiber1947, Peters1986, Brown2004}.
Many empirical and theoretical studies have attempted to obtain the correct relationship \citep{West1997, Kozlowski1997, Isaac2010}.
When we consider performance (fitness), an important process that is often overlooked that is the role of behavioral thermoregulation for performance. 
A particular example is warm-up phase. 
%For cold blooded animal, thermoregulation is an important part of the daily activity.
%In warm environment, species thermoregulate to avoid overheating.
%For instance, some insects avoid overheating during flight by dissipating excess heat from the thorax to the abdomen \citep{Verdu2012}.
%In cold environment, some individuals also regulates avoid losing heat.
%In a cold environment, some insects have the opposite strategy that insulate the thorax from the rest of the body to keep heat \citep{Verdu2012}.
Muscle needs to be at certain temperature to function properly thus when the environment is below that temperature, warm-up is necessary. 
For cold blooded animal, the general strategy is to bask under the sun and absorbs the heat.
In this case, body size have a direct influence as it mediates heat absorption (conductance) and in general exchange via surface-area to body size ratio.
\citet{Buckley2008} included warm-up process of a species of lizard but in general the role of such process in shaping thermal performance has not been explored. 

In this work we build a theoretical model to investigate how thermal performance varies with body size for cold blooded animals. 
Our approach is to look at the effects of three processes: physiological processes based on metabolism, ecological factor which depends on resource availability and foraging, and thermodynamical factor, thermoregulation processes.
The philosophy of the model is the same as energy budget model, here we define performance as difference between total energetic gain and energetic cost.
We narrow our taxonomic scope to insect, more precisely to adult insect with deterministic body size and income breeder. 
In general, we found the metabolism plays a secondary role in shaping thermal performance.
Instead, resource availability and allometric scaling of foraging are key in defining the upper thermal limit whereas thermoregulation ability of warm-up sets the lower thermal limit.
