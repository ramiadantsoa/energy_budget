\section*{Introduction}
A principal goal in ecology is to describe and understand processes that influence species distribution.
The act of describing and explaining is not often mutually inclusive.
In some cases, things fit naturally.
Abiotic factors can easily correlate with species range.
Temperature isocline matches range limits which provides evidences of range limitiation \citep{}.
Biotic factors also plays a role in the shaping species range \citep{}.
Various types of interaction within and among trophic level can both expand or contract species distribution.
In the most general case, it is not evident which factor is most important.
Sometimes, several factors are correlated and disentangling the relative effect of these  factors is a difficult task.

A surrogate approach is  flip the question, choose a factor and see how it affects the performance of the species, given everything else is equal.
Temperature is one abiotic that has been investigated in depth.
Controlled experiments have been conducted to see how species responds to gradient of temperature.
It is often complicated to directly measure fitness and instead rely on proximate variables (performance) such as locomotion, growth rate, assimilation, fecundity rate \citep[][and reference therein]{Angilletta2009}.
A classical results is that thermal performance is non monotonic and is skewed to the left meaning that the optimal temperature (peak of the curve) is closer to the critical maximum temperature for positive performance than the minimum temperature. 
Whereas the approach looking the marginal effect of one factor is difficult of empirical study (it can be logistically difficult to keep everything else equal), it is the basic philosophy behind a theoretical approach.

Theoretical works have investigated the role of temperature in shaping performance especially in growth \citep{} % Kozlowski, vanderhave  
Recently theoretical work has underpinned mechanisms that look directly at fitness \citep{Amarasekare2012}.
The model  breaks down thermal performance (intrinsic growth rate)  into three components: development, fecundity, and mortality.
The functional shapes of each of these components thus determines the basic properties of the curve.
Not only, the left-skewed is recovered but the model provides an explanation of why it is left-skewed--because of how the three component behaves. 
A different model that dig deep into physiological principles is called Dynamic Energy Budget (DEB) \citep{Kooijman2009}.
The model focuses on how (food) energy is assimilated and then allocated to different needs such as growth, metabolic cost (maintenance), reproduction \citep{Kooijman2009}.
When species data are available, these models are excellent in reproducing the pattern. 

Such model framework helps us understand that performance curves and how they differs  among individuals, but it does not provide a framework for understanding the intrinsic causes of those differences.
One variable that might underlie those differences is body size.
Body size is often associated with temperature.  
For instance, individual grows larger under colder thermal regime \citep{Van1996}.
Metabolic theory of ecology links with one equation body size and temperature \citep{Brown2004}
%...Something about large have broader distribution... \citep{Lumaret1996}.
One of the most well-known macroecological (ecogeographic) pattern is the Bergman's rule which states that body size tends to increase with decreasing temperature \citep{Bergman1847}. 
%Although the original formulation was intended for homeotherms at interspecific level, the pattern has been reported for heterotherms and at intraspecific level \citep{Blackburn1999}.
%Several hypotheses have been suggested to underlie the patterns.
%For instance, individual grows larger under colder thermal regime \citep{Van1996}.
%The original mechanism proposed by Bergman relates to surface area to volume ratio which confers larger species with higher heat conservation \citep{Blackburn1999}.
Yet, there is currently no study that investigate how the thermal niche changes with body size.
Our goal of the work is to develop and analyze a model that evaluates how different processes that links to body size generates thermal performance curves.


Remain verbal theories....
Body size and thermoregulation...
For cold blooded animals, temperature is crucial as it regulates activities.
Thermoregulation is important in both warm and cold environment.
In warm environment, species thermoregulate to avoid overheating.
For instance, some insects avoid overheating during flight by dissipating excess heat from the thorax to the abdomen \citep{Verdu2012}.
In cold environment, some individuals also regulates avoid losing heat.
In a cold environment, some insects have the opposite strategy that insulate the thorax from the rest of the body to keep heat \citep{Verdu2012}.
An overlooked component is warm-up.
Muscle needs to be at certain temperature to function properly thus when the environment is below that temperature.
And individual needs to do warm-up. 
This can be achieved by basking or under the sun or some endogeneous heat production through shivering (refs).
In this case, body size have a direct influence as it mediates heat absorption and in general exchange via surface-area to body size ratio this is called conductance.
The capacity of warm-up has recently being integrated into physiological model see Buckley for one lizard,  but there is no explicit study of how these leads to differences in overlap performance and to make comparison across body size. 
%Behavioral thermoregulation is important but its consequence on thermal performance remain unexplored \citep{Dial2008}.

%In a cold environment, activities is  preceded by warm up phase because a minimum temperature is required for muscle to be efficient (refs).
%The general strategy is to bask under the sun like many lizards are doing but some species (insects) such as bees, moths, and dung beetles are capable of generating heat endogenously by contracting their muscles \citep{Heinrich1975, Bartholomew1978, Bartholomew1981}.
%The ability to warm-up is thought to be a function of body size, for instance  bee can complete warm-up in colder environment than smaller one although there is no different in heat generated per unit of mass \citep{Kammer1974, Heinrich1975}.
%Successful warm-up is needed for foraging and thus gaining energy.
%Such process has been included in some studies.
%For instance, \citet{Buckley2008} included the operative temperature of the North American lizard that conditioned but there is no current theoretical framework that investigate the role of body size in the warm-up phase and in general in the thermal performance \citep{Dial2008}.

 In this work we build a theoretical model to investigate how thermal performance varies with body size.
 By looking at different facets of the processes.
 
 More precisely, to understand how processes resulting from body size and temperature relationship such as thermoregulation, metabolism, and foraging creates different patterns and which conditions favor large or small body.
 The model is an energy budget model, performance is defined as the difference between total energy gained and total energy lost to stay alive.
 Unlike other models of energy budget, we do not fit any species but focus on the role of various processes on performance. This is thus a process-based model not a pattern-based.
%We use physiological principle (metabolic theory) for the metabolic cost \citep{Brown2004}.
%Our approach is not to fit any species but to understand the role of different processes that influence net energy gain and in general the allometric scaling of body size.
%We narrow our taxonomic scope to insect,  more precisely to adult insect with deterministic body size and income breeder. 
In general we found that resource availability and allometric scaling of foraging and defines the upper thermal limit whereas thermoregulation ability of warm-up sets the lower thermal limit.

%%%%%%%%%%%%%%%%%%%%%%%%%%%%%%%%%%%%%%%%%%%%%%%%%%%%%%%%%%%%%%%%%%%%%%%%%%%%%%%%
% 
%SpeciesThe distribution of species are influenced two distinct factors: abiotic and biotic.
%Abiotic factors often sets a hard limit on species distribution.
%A simple example is the tree line caused by temperature.
%Biotic factors are often more flexible.
%The presence of competitor is a familiar example, see barnacles, disable range limits.
%Trophic interaction, example the absence of resource or prey precludes the  species to be there, enable range limits.
%The niche of the species is concept introduced by Hutchinson that describe the range of condition for the persistence of the species.
%One can associate limits by abiotic as the fundamental and limits by biotic as realized niche.
%Reconstructing the niche along biotic and abiotic axes is a holy grail in ecology. 
%
%Temperature is one of the most important abiotic factor. % it often shows a global gradient in latitude and altitude.
%In the face of global warming there is even a further need to understand the thermal niche of a species.
%A very popular method to reconstruct the relationship between a species and its thermal environment is to use species distribution models. 
%Species distribution models are based on statistical procedures that finds a correlation between known species' occurrence and climate variables.
%The primary goal is to predict (future) species distribution for a given climate scenarios \citep{Guisan2005, Austin2007, Elith2009}
%A more direct way to measure performance along a temperature gradient (thermal sensitivity, thermal performance) is via controlled experiment.
%It is often complicated to directly measure fitness and instead rely on proximate variables such as locomotion, growth rate, assimilation, fecundity rate \citep[][and reference therein]{Angilletta2009}.
%These studies found that the performance is hump-shaped with and skewed to the left i.e. the peak is closer to upper limit than the lower limit.
%
%Recently theoretical work has underpinned mechanisms that underlies the skewness of the thermal performance \citep{Amarasekare2012}.
%The model  breaks down thermal performance (intrinsic growth rate)  into three components: development, fecundity, and mortality.
%The functional shapes of each of these components thus determines the basic properties of the curve.
%The framework is both elegant and sound because not only these components can be measured empirically but are also rooted on physiological principles.
%A different model that dig deep into physiological principles is called Dynamic Energy Budget (DEB) \citep{Kooijman2009}.
%The model focuses on how (food) energy is assimilated and then allocated to different needs such as growth, metabolic cost (maintenance), reproduction \citep{Kooijman2009}.
%A primary use of  these models is to fit a pattern to the data and required a detailed knowledge of the species.  
%
%
%Such model framework helps us understand that performance curves and how they differs  among individuals, but it does not provide a framework for understanding the intrinsic causes of those differences.
%One variable that might underlie those differences is body size.
%Body size is often associated with temperature.  
%For instance, individual grows larger under colder thermal regime \citep{Van1996}.
%Metabolic theory of ecology links with one equation body size and temperature \citep{Brown2004}
%%...Something about large have broader distribution... \citep{Lumaret1996}.
%One of the most well-known macroecological (ecogeographic) pattern is the Bergman's rule which states that body size tends to increase with decreasing temperature \citep{Bergman1847}. 
%%Although the original formulation was intended for homeotherms at interspecific level, the pattern has been reported for heterotherms and at intraspecific level \citep{Blackburn1999}.
%%Several hypotheses have been suggested to underlie the patterns.
%%For instance, individual grows larger under colder thermal regime \citep{Van1996}.
%%The original mechanism proposed by Bergman relates to surface area to volume ratio which confers larger species with higher heat conservation \citep{Blackburn1999}.
%Yet, there is currently no study that investigate how the thermal niche changes with body size.
%Our goal of the work is to develop and analyze a model that evaluates how different processes that links to body size generates thermal performance curves.
%
%For cold blooded animals, temperature is crucial as it regulates activities.
%Thermoregulation is important in both warm and cold environment.
%In warm environment, species thermoregulate to avoid overheating.
%For instance, some insects avoid overheating during flight by dissipating excess heat from the thorax to the abdomen \citep{Verdu2012}.
%In cold environment, some individuals also regulates avoid losing heat.
%In a cold environment, some insects have the opposite strategy that insulate the thorax from the rest of the body to keep heat \citep{Verdu2012}.
%An overlooked component is warm-up.
%Muscle needs to be at certain temperature to function properly thus when the environment is below that temperature.
%And individual needs to do warm-up. 
%This can be achieved by basking or under the sun or some endogeneous heat production through shivering (refs).
%In this case, body size have a direct influence as it mediates heat absorption and in general exchange via surface-area to body size ratio this is called conductance.
%The capacity of warm-up has recently being integrated into physiological model see Buckley for one lizard,  but there is no explicit study of how these leads to differences in overlap performance and to make comparison across body size. 
%%Behavioral thermoregulation is important but its consequence on thermal performance remain unexplored \citep{Dial2008}.
%
%%In a cold environment, activities is  preceded by warm up phase because a minimum temperature is required for muscle to be efficient (refs).
%%The general strategy is to bask under the sun like many lizards are doing but some species (insects) such as bees, moths, and dung beetles are capable of generating heat endogenously by contracting their muscles \citep{Heinrich1975, Bartholomew1978, Bartholomew1981}.
%%The ability to warm-up is thought to be a function of body size, for instance  bee can complete warm-up in colder environment than smaller one although there is no different in heat generated per unit of mass \citep{Kammer1974, Heinrich1975}.
%%Successful warm-up is needed for foraging and thus gaining energy.
%%Such process has been included in some studies.
%%For instance, \citet{Buckley2008} included the operative temperature of the North American lizard that conditioned but there is no current theoretical framework that investigate the role of body size in the warm-up phase and in general in the thermal performance \citep{Dial2008}.
%
% In this work we build a theoretical model to investigate how thermal performance varies with body size.
% More precisely, to understand how processes resulting from body size and temperature relationship such as thermoregulation, metabolism, and foraging creates different patterns and which conditions favor large or small body.
% The model is an energy budget model, performance is defined as the difference between total energy gained and total energy lost to stay alive.
% Unlike other models of energy budget, we do not fit any species but focus on the role of various processes on performance. This is thus a process-based model not a pattern-based.
%%We use physiological principle (metabolic theory) for the metabolic cost \citep{Brown2004}.
%%Our approach is not to fit any species but to understand the role of different processes that influence net energy gain and in general the allometric scaling of body size.
%%We narrow our taxonomic scope to insect,  more precisely to adult insect with deterministic body size and income breeder. 
%In general we found that resource availability and allometric scaling of foraging and defines the upper thermal limit whereas thermoregulation ability of warm-up sets the lower thermal limit.
