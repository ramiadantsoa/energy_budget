\section*{Introduction}
A myriad of biotic and abiotic factors  hypothetically affect species' performance such as growth, fecundity, and in general fitness \citep{Krebs2000}.
Disentangling the relative of effects of those factors in shaping performance is generally difficult.
First, biotic species interaction can be cryptic, and even they are not, it is not obvious to tell if the interaction really matters \citep{Morales2015}.
Second, abiotic factors such as photoperiodism, temperature, and humidity are often correlated.
 
A surrogate approach is to flip the question around: choose a factor and see how it affects the performance of the species---given everything else is equal.
Temperature is one abiotic that has been intensely investigated.
Numerous empirical studies have been conducted to investigate how species respond to temperature gradient \citep[e.g.,][]{Angilletta2009}.
A direct measure of fitness is often complicated and studies instead rely on proximate variables (performance) such as locomotion, growth rate, assimilation, fecundity rate \citep[][and references therein]{Angilletta2009}.
A classical result is that thermal performance is a non-monotonic curve and is skewed to the left meaning that the optimal temperature (peak of the curve) is closer to the critical maximum temperature than the critical minimum temperature. 
Such pattern for instance has lead to the claim that species in the tropics might be more vulnerable to global warming given that they are actually in an environment that matches their optimum temperature \citep{Deutsch2008}.

In general, measuring the effect of one factor and keep everything else equal  is difficult to do empirically.
In contrast, simplification is the central tenets of theoretical works.
Theoretical works have investigated the role of temperature in shaping performance especially for growth \citep{VandH1996, Kozlowski2004}.
Recently, a theoretical model reconstructs the fitness of an individual as a function of temperature \citep{Amarasekare2012}.
The model  breaks down thermal performance (intrinsic growth rate)  into three components: development, fecundity, and mortality.
The functional shapes of each of those components and how they intersect thus determine the basic properties of the curve (optimium, minimum, maximum, and the skewness).
A different model that dig deep into physiological principles is called Dynamic Energy Budget (DEB) \citep{Kooijman2009}.
The model focuses on how (food) energy is assimilated and then allocated to different needs such as growth, metabolic cost (maintenance), reproduction \citep{Kooijman2009}.
When species'  data are available, these models are great power in reproducing the pattern. 

Such models help to understand and describe the qualitative and quantitative shape of the performance curves and how they differ among individuals, but it does not provide a framework for why they differ--- the intrinsic causes of those differences. % E: I like this.  Might just need to clarify that "development curve" etc. is not an "intrisic cause". T:  My question is the following statement: You are specifically talking about development curve  not mortality and others. I made a minor change but I am not sure if it really address it.
One variable that might underlie those differences is body size.
Empirical data have shown that body size is often associated with temperature.  
At global scale, the Bergman's rule states that body size tends to increase with decreasing temperature \citep{Bergman1847}.
At the physiological level, individual grows larger under colder thermal regime \citep{Van1996}.
Metabolic theory of ecology links in one equation body size and temperature \citep{Gillooly2001}.
Theoretical studies that link body size and performance do not take account the explicit effect temperature \citep[e.g.,][]{Yodzis1992, Brown1993}.
The question thus remains, given everything else is equal, how does thermal performance curve change with body size?.

Body size influences many of ecological process \citep{Peters1986}.
%One of the most well known is that resting metabolic rate scales with body size by following a power law \citep{Kleiber1947, Peters1986, Brown2004}.
%Many empirical and theoretical studies have attempted to obtain the correct relationship, namely the universality of the 3/4 law \citep{West1997, Kozlowski1997, Isaac2010}.
When we consider performance (fitness), an important process that is often overlooked is the role of behavioral thermoregulation in shaping performance. 
A particular example is warm-up phase for ectotherm especially for insects.
Because muscle needs to be at certain temperature to function properly, when the environment is below that temperature, warm-up is necessary \citep[e.g.,][]{Heinrich1975}. 
In some communities like dung beetles, the duration and completion of warm-up is important as it is crucial to arrive on site before resource quality degrades or gets depleted \citep{Hanski1991}. %(Anderson & Coe 1974, Horgan 200,  Heinrich & Bartholomew, 1979 ).
The ability to warm-up has further been hypothesized to cause diel partitioning which facilitates coexistence \citep{Viljanen2009}.
Body size has a major influence in warm-up process because it mediates heat exchange via surface-area to body size ratio.
The warm-up aspect of thermoregulation, its dependency on body size, and let alone its role in shaping thermal performance has not been explored.
 
In this work we build a theoretical model to investigate how thermal performance curve varies with body size.  % E: Perhaps a bit too broad here, given that you soon after narrow the scope to some insects. T: this was from previous versions, does it still hold? 
We define performance as net energy gain which is the difference between total energetic gain and energetic cost. 
Our approach is similar to others energy budget models \citep[e.g. Niche Mapper\textsuperscript{TM} or][]{Kooijman2009} but the goal is not to fit a particular species.
Instead, we look at the effects of three processes: physiological processes based on metabolism, ecological processes based on resource availability and foraging, and  thermoregulation of warm-up based on thermodynamical in shaping net energy gain.
Whereas the model is conceptually valid for heterotherm, we narrow our taxonomic scope to insect and the model is best suited for adult insect with deterministic body size and income breeder. 
In general, we found that the metabolism plays a secondary role in shaping thermal performance.
Instead, resource availability and allometric scaling of foraging are key in defining the upper thermal limit whereas thermoregulation ability of warm-up sets the lower thermal limit.
