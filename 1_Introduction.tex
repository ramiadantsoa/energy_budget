\section*{Introduction}
Niche is important concept.
Habitats are changing. 
The niche is one of the most important tools to understand pop dynamics.
With climate change, there is a need to understand how species responds to temperature.

Experimental studies about temperature to measure fitness.
All these reports a non monotonic pattern.
It is skewed to the left.
Of importance are where intersect 0.
They are called CTmin and CTmax.
Indirect measure: locomotion, growth as a proxy for fitness.

Theory/modles now exist to predict the perf in a given env.
The most notorious are DEB aimed to explain pattern.
Recently, Amarasekare and Savage developed a model that explaines the skewness.
Break down into effect of temp on mortality,  fecundity and development.
 
Unlike the previous models, actual model focus for species dist now includes behavior.
For ectotherm, thermoregulation is crucial.
Current  models now include thermo.
Buckley and operative temperature.
Review about thermoregulation.
Temperature as a cue

Body size

Most models are pattern based.
In this work, we construct a core modelThere is now general model for the role of behavior in fitness.


 

Niche is one of the most important concepts in ecology.
Niche was popularized by \citet{Hutchinson1957} who described species' niche as a hypervolume in a n-dimensional space.
Moving from Hutchinson's abstraction to concrete and empirical description of the niche is nontrivial.
The difficulty of measuring the niche as well as the emergence of non-niche based theories  (e.g. Hubbell) have dwindled the interest in ecological niche but recently started to play a more dominant role again % (or revival, resurgence).. 
This is partly due terminological clarifications, for instance the distinction between the Grinellian and Eltonian niche \citep{Chase2003} but also because of the crucial need to predict species' response to current rapid human-induced environmental changes \citep[e.g.,][]{Parmesan2006, Kearney2009}. %one can think about environmental filtering too
The geometrical abstraction by \citet{Hutchinson1957} is particularly useful to illustrate one key property of the niche: its breadth.
%Niche breadth can however take two forms whether it is the fundamental or realized niche.
Yet  characterization of niche breadth focused essentially on the realized niche in both theoretical and empirical works although understanding fundamental niche would be the first step.

%In fact species with narrow niche breadth is more vulnerable to environmental changes \citep{Henle2004}.

%another way to present is to talk about spectrum, this is at the one end, although not necessarily true for 
Empirically, the characterization of the niche generally adopts three main approaches \citep{Holt2009}.
The first approach consists of measuring the performance of  individuals in at different environmental gradients \citep{Birch1953, Elliott1982, Angert2005, Frazier2006}.
Depending on the design,the experiment would yield an approximation of fundamental or realized niche \citep{Birch1953, Elliott1982, Angert2005, Frazier2006}.
%The problem is that it is quite tedious to do (can be slow, requires more resources, applicable to only few species
The second approach consists of building models that integrate highly detailed mechanisms at the molecular level, thermodynamic, physiological and behavioral to determine  the performance \citep{Kooijman2009, Kearney2009, Buckley2008}.
This approach generally focus on the fundamental niche.
%approach is to mechanistically construct how an individual trait interacts with the environment using a bottom-up approach %Marr-Pirt or Monod model for early energy budget model
%The main approach is called energy budget model and is based on detailed physiological, thermodynamic and behavioral processes. 
%The main problem with these two approaches is that they are tedious and applies to few well-studied species.
%called brute force by Holt, some key parameters are still needs to be measured and is not often easy to get them (Nisbet2000). 
The third approach is to use statistical methods that infers the niche from correlation between species occurrence and the environmental conditions \citep{Guisan2005, Austin2007, Elith2009}. %also Thullier2008 
Unlike the previous two, this last approach is less data greedy and applicable to a wide variety of species.
It is suited to tackle conservation problems.
In these approaches, there is often a striking discrepancy between the depth of comprehension and broad application.
Tactical strategic spectrum of pop model (Holling 1966)
In first two, the mechanistic probing can be tedious whereas the last approach gives a correlation and tells little about the fundamental niche.

Theoretically, niche partitioning has been most dominant.
Study of coexistence aim to find mechanisms that would allow species to partition the niche in space such as the competition colonization trade-off \citep[e.g.,][]{Levins1971,Tilman1994} or partition the niche in time such as storage effect \citep{Skellam1951, Chesson2000}. %note that storage-effect also applies in space but the temporal aspect is the easiest  
 %Theoretical studies put emphases on the realized niche especially competition (resource or niche partitioning) and there are a myriad of approaches (often unlinked) depending on which phenomenon is under investigation.  
The broken stick model is used to explain species composition (abundance or relative abundance) of communities \citep{MacArthur1957}.
Yet, what constitutes the niche remains abstract and is simply a latent variable.
%Although it has explanatory power, it does not tell how species interact with the environments. focus on partitioning the abstract niche rather than what is actually the niche axis. Oblivious to what niche axis actually are.
Other models focus on the strength of competition along one niche axis.
For instance, assuming a given trait has Gaussian resource utilization curve, competition  between two traits is quantified by the amount of overlap between their curves \citep{MacArthur1967, Roughgarden1979}.% and lead to principle of limiting similarities.
In that latter, fundamental niche is given for granted and is used to study competition.
%Finally, some other models investigates the evolution of niche width based on quantitative genetic framework \citep[reviewed in][]{Futuyma1988}.

In the early age, Levins that they will be entangled not to be able to separate.
There is still a divide between theory and data (Angilletta2009).
DEB is in between but still too much on the data side.
Mechanistic model such as DEB uses theory but they are too data dependent.
\textit{Maybe a chance here to talk more about these energy budget model}
In times, the parameters cannot be measured and thus limit their applicability (Nisbet).
In addition the complexity of the processes prevents the understanding of how the processes influence the performance.
It was shown energy budget model can fit data very well.
In fact that is the goal.
Theory on the other hand, surf too much on the abstraction and the realized niche.
There is no general theory about fundamental niche.
Traits are not defined..


Paragraph about behavior, thermoregulation, body size allometry
Allometry of behavior Dial 2008 TREE
Hump-shaped attack rate and body size 

Our goal in this work is to develop a model to study fundamental niche. Building block. Too great sacrifice in realism
More specifically about niche breath for fundamental niche.
The goal is find a balance between realism and abstraction that would constitute a bridge between theory and data.
We thus pick two concrete variables: body size (as an intrinsic trait) and temperature (as niche axis). 
We are interested about two main processes: physiological and behavioral. %can cite Futuyma here if more refinement is needed. Or if we want to state that those are the first underlying processes.
We will root our modeling assumption on the vast literature on physiological and behavioral responses to temperature. 
We use the same approach as energy budget model but put emphasizes on the role of the processes rather than fitting empirical patterns.% Require different type of data.
We can then answer a specific question: under what condition niche breadth (or thermal performance breadth) increases or decreases as a function of body size.


%MEETING NOTES
% Data driven and then pure theory....competition, fundamental...
% Expand, what is included in foraging...
% cost of shivering
% why are parameters of interest.
% fig for limited time
% emphasize generality/realism.
% Model parameters...discussion first paragraphs into methods. 
% not say it all depends. highlight the important ones.

%To produce general results, theoretical models often make simplifying assumptions for mathematical convenience.
%In niche models that actually consider a relationship between the trait and say performance such as in \citet{Roughgarden1979}, the assumptions are often violated.
%For instance, resource utilization curve is Gaussian but there is no evidence that say large beak size cannot exploit small seed as efficiently as large ones. % do we need reference here
%The symmetrical nature of the niche is not necessarily true.
%Experimental studies  along temperature gradient shows a left skewed performance curve \citep{Angilletta2009}.
%Finally, a trade-off in performance is not empirically verified (refs). %find that refs, otherwise use warmer is better.
%In general, this incongruence generated a gap between theoretical and empirical works \citep{Amarasekare2003}.
%
%In this work, we develop a theoretical models to reconstruct the niche; more precisely a physiological and behavioral model that derives the fundamental niche for given a trait and along an environmental gradient.
%The conceptual goal is to find a balance between realism and abstraction.
%In doing so, we choose two of the most important variables: temperature as niche axis and body size as the focal trait.
%The model is for adult insects...%T: not sure if there is a need to place it the intro 
%Practically, there is a vast literature on the relationship between body size, temperature and the physiological and behavioral processes.
%These knowledge can be used to select and root modeling assumptions to empirical data and thus brings a degree of realism.
%We use the same approach as energy budget model but put emphasizes on the role of the processes rather than fitting empirical patterns.
%We want to derive a null model for performance as a function of body size and answering a specific question that is: under what condition niche breadth (or thermal performance breadth) increases or decreases as a function of body size.
%
