\section*{Introduction}
%Temperature is one of the most important abiotic variables.
%Many cellular processes requires a certain range to function.
%These can then propagate to higher level of biological organizations.
%At a global scale, temperature which correlates with latitude correlates with species richness. 
...
As the global temperature is raising, understanding species' fitness given thermal environment has become crucial.

The most common method to elucidate the relationship between a species and its thermal environment is to use species distribution models. 
Species distribution models are based on  statistical procedures that finds a correlation between known species' occurrence and climate variables.
The primary goal is to predict (future) species distribution for a given climate scenarios \citep{Guisan2005, Austin2007, Elith2009}
A more direct way to measure performance along a temperature gradient (thermal sensitivity, thermal performance) is via controlled experiment.
It is often complicated to directly measure fitness and instead rely on proximate variables such as locomotion, growth rate, assimilation, fecundity rate \citep[][and reference therein]{Angilletta2009}.
Three basic properties of interest are the peak, the upper and lower boundary of the curve.
They translate to the temperature that maximizes performance, the highest and lowest temperature above and below which performance becomes null, respectively. 
These studies found that the performance is hump-shaped with and skewed to the left i.e. the peak is closer to upper limit than the lower limit.

Recently theoretical work has underpinned mechanisms that underlies the skewness of the thermal performance \citep{Amarasekare2012}.
The model  breaks down thermal performance (intrinsic growth rate)  into three components: development, fecundity, and mortality.
The functional shapes of each of these components thus determines the basic properties of the curve.
The framework is both elegant and sound because not only these components can be measured empirically but are also rooted on physiological principles.
A different model that dig deep into physiological principles is called Dynamic Energy Budget (DEB) \citep{Kooijman2009}.
The model focuses on how (food) energy is assimilated and then allocated to different needs such as growth, metabolic cost (maintenance), reproduction \citep{Kooijman2009}.
A primary use of  these models is to fit a pattern to the data and required a detailed knowledge of the species.  

A general question is how thermal performance varies among species.
...Something about large have broader distribution... \citep{Lumaret1996}.
An intrinsic variable that often correlates with temperature is body size.
One of the most well-known macroecological (ecogeographic) pattern is the Bergman's rule which states that body size tends to increase with decreasing temperature \citep{Bergman1847}. 
Although the original formulation was intended for homeotherms at interspecific level, the pattern has been reported for heterotherms and at intraspecific level \citep{Blackburn1999}.
Several hypotheses have been suggested to underlie the patterns.
For instance, individual grows larger under colder thermal regime \citep{Van1996}.
The original mechanism proposed by Bergman relates to surface area to volume ratio which confers larger species with higher heat conservation \citep{Blackburn1999}.
...
There is currently no study that investigate or how thermal performance changes with body size.

Body size has been related to several ecological and physiological process but it is importance for thermoregulation and behavior remains unexplored let alone be the consequences for thermal performance \citep{Dial2008}.
Thermoregulation  is important for ectotherms as activity depends directly on it.
There is thermoregulation during activities.
For instance, some insects avoid overheating during flight by dissipating excess heat from the thorax to the abdomen \citep{Verdu2012}.
In a cold environment, some insects have the opposite strategy that insulate the thorax from the rest of the body to keep heat \citep{Verdu2012}.
In a cold environment, activities is  preceded by warm up phase because a minimum temperature is required for muscle to be efficient (refs).
The general strategy is to bask under the sun like many lizards are doing but some species (insects) such as bees, moths, and dung beetles are capable of generating heat endogenously by contracting their muscles \citep{Heinrich1975, Bartholomew1978, Bartholomew1981}.
The ability to warm-up is thought to be a function of body size, for instance  bee can complete warm-up in colder environment than smaller one although there is no different in heat generated per unit of mass \citep{Kammer1974, Heinrich1975}.
Successful warm-up is needed for foraging and thus gaining energy.
Such process has been included in some studies.
For instance, \citet{Buckley2008} included the operative temperature of the North American lizard that conditioned but there is no current theoretical framework that investigate the role of body size in the warm-up phase and in general in the thermal performance \citep{Dial2008}.

 In this work we build a theoretical model to investigate how thermal performance varies with body size putting emphasis on how the role of warm-up process in shaping the curve. 
The model is an energy budget model, that is performance is defined as the difference between total energy gained and total energy lost to stay alive.
We use physiological principle (metabolic theory) for the metabolic cost \citep{Brown2004}.
Our approach is not to fit any species but to understand the role of different processes that influence net energy gain and in general the allometric scaling of body size.
We narrow our taxonomic scope to insect,  more precisely to adult insect with deterministic body size and income breeder. 
In general we found that resource quality and quantity, allometric scaling of foraging and defines the upper thermal limit whereas warm-up ability sets the lower thermal limit.
