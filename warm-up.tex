\section*{Modeling warm-up process} % (fold)
\label{sec:modeling_warm_up_process}
%
We make the following assumptions.
First, daily air temperature is simplified to a sine function,
\begin{equation} \label{eq:Ta}
	T_a(t) = \overline{T_a} + \nu \sin(\omega t),
\end{equation}
where $t$ is time of the day starting at sunrise, $\overline{T_a}$ is the average daily temperature, $\nu$ is the amplitude of the fluctuation, and $\omega = 2 \pi /24 h$ is the period.

Second, the operative temperature of an individual of mass $z$ is 
\begin{equation} \label{eq:Te}
	T_e(z,t) = T_a(t) + T_d(z),
\end{equation}
where $T_d(z)$ pools the effects of thermal environment (solar radiation, convection, conductance and so on) on the individual. 
There are exact formulas for calculating and also approximating $T_d$ \citep[e.g.,][]{Stevenson1985, Angilletta2009} but instead we use the following simplified (justified?) equation
\begin{equation} \label{eq:Td}
 	T_d(z) = T_d(z_0) + \delta_T \log_{10} \left(\frac{z}{z_0} \right).   
\end{equation} 
 The equation above states that for a 10-fold increase in body mass, the operative temperatue increases by $\delta_T$ degree Celsius.
 The smallest individual we consider here has mass $z_0$ with a maximum increase $T_d(z_0)$.
 The idea comes from \citet{Stevenson1985} which suggested that an 10-fold increase in body mass raises the maximum body temperature an individual can attain by $4.5 ^\circ \rm{C}$.
 We are making an implicit assumption here that the maximum body temperature and maximum operative temperature are equal (justified?).

 Third, the change in body temperature $T_b$ follows Newton's law of heating/cooling,
 \begin{equation} \label{eq:Tb0}
 	\frac{T_b(z,t)}{dt} = \frac{k_c(z)}{Q(z)} \left( T_b(z,t) - T_e(z,t)\right),
 \end{equation}
 where $k_c(z)$ is the heat transfer coefficient and $Q(z)$ is the heat capacitance of the individual.
The heat capacitance is simply mass times the specific heat capacity i.e., $Q(z) = z s$.
We simplify the heat transfer coefficient such that it is proportional to the surface area of the individual i.e. $k_c(z) = \alpha' z^{2/3}$.
By replacing $T_e(z,t)$ in \cref{eq:Tb0} by its expression in \cref{eq:Te}, and $T_a(t)$ and $T_d(t)$ by \cref{eq:Ta} and \cref{eq:Td}, respectively, we have
\begin{equation} \label{eq:Tb}
	\frac{T_b(z,t)}{dt} = \frac{\alpha}{z^{1/3}} \left[ T_b(z,t) - \nu \sin(\omega t) - f(z) \right],
\end{equation} 
where $\alpha = \alpha'/s$, and $f(z)$ is a time independent function $f(z) = \overline{T_a} + T_d(z_0) + \delta_T \log_{10} \left( \dfrac{z}{z_0} \right)$.
%
The advantage is that \cref{eq:Tb} has a closed form solution.
% section modeling_warm_up_process (end)


\section{Cost of warm-up} % (fold)
\label{sec:cost_of_warm_up}
In the preview version, we assumed that if $\tau_f$ is the total time to acquire a quantity of resource $R$, $\tau_f = R/g(z)$, then the actual foraging time is $\tau_f - \tau_w$. 
I suggest we assume a depletion rate that is independent of body size. 
If warm-up starts at time $t_0$ then the amount resource available decreases at a certain rate (whatever linear or exponential which can also be parameterized).
It might be a better way of envisioning competition. 
The motivation is to think about an elephant dung that is gradually depleted.

% section warm_up (end)





