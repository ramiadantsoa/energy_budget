\section*{Discussion}
We have developed a model that would give the performance of species along two main gradients: temperature and  body size.
Our modeling approach has a different purpose that existing energy budget model.
In the sense that we do not fit the model to a particular system, it is not a statistical model.
In doing so, we anyway try to keep parameter close values that we find in the literature.
We assess parameter values affect the qualitative and quantitative outcomes.
Our goal is that although things are highly simplified, the meaning of parameters are still clear.
The problem with some models indeed is that generality trade-off with clear parameters.
Find a balance here.
Mentioning Brown here about what those parameters means. 
Given everything else is equal, what do we get....

One of the central assumptions of the model is the use of power law to model much of the process with relation to body size.
The power-law by Brown, although it dates back to xxx years is the most empirically based part here.
Discussions however evolved on whether the 3/4 is the best value.
There is no general empirical results for active metabolic rate. 
We have found one that scales with power 1.xxx from Heinrich. 
This of course has not been measured consistently. 
Often the focus is on metabolic scope and thus from the individual perspective.
Our modeling shows that in term of performance, energetic cost is relevant when mapped with energetic gain.

Foraging rate is also not much measured.
It is definitely a complicated process, resource distribution, finding resource, blabla...foraging theory.
What has been done is to calculated the total energetic cost (Nagy field metabolic rate).
Foraging rate is extrapolated.
Within a guild.
It is often the minimum.
Foraging in the most general sense, it increases with body size.
How can we deviate from power law? Large individual can forage longer to satisfy their energetic demands.

The role of resource distribution, when they occur and what are their quality.
Polar bear....

A particular feature of the model is to study in depth the warm-up ability of the insects.
No intuitive relationship.
Warm-up? Kearney and his shade.
There are two topics.
Physical property:
there has  been verbal assumptions on how warm-up happens.
Dung beetles: large forage and small pitch. 
Small have lower ability.
Measure conductance.
Not so intuitive.
Is it really limiting?
ROLE OF CONDUCTANCE

Behavior: when to warm-up can be decisive.
The role of competition.
Diel activity, competitive dominance.

Putting all things above together.
Modularity.
What is not intuitive: optimal body size and endo vs. ecto.

U-shape.
Is something else happening in terms of what happens in cold environment.
 
Net energy gain as performance, how to get to fitness.
Obviously, many components are missing.
Humidity, fecundity and survival.
Kearney and his shade model.
Large dung beetles tolerates heat.
Optimality can shift (e.g. kooijiman).

Body size is our central focus here.
It is a magic trait.
As stated earlier it seems to correlate with physiological functions.
Physical shape is governed by the law of thermodynamics.

Bergmann's rule and pattern of body size.
Most scrutinized trait.
Testing all the hypothesis in James' paper.

Additional application:
Mathematical models: optimum body size, carrying capacity and 
Generalized fitness landscape.

Growth is additionally a central part.
Allocation to fecundity is a major question. 
These can be answered with more sophisticated model.
As growth does not tell what is optimal fitness, so we say that omitting growth can still indicate what species will suffer more.

Empirical inquiry.
Individuals are different, we look at the general mean here. 
It would be nice to see how the variance changes.

What can we learn from models?
cite Angiletta...
In conclusion, we noticed that many parameters needs to be estimated.
b2, b3, eps, K1.

A central question we ask is how  niche as a function of temperature varies with body size.
We found that multiple scenarios are possible.
It all depends on what we fed the model with in terms of modeling assumptions and model parameters.
Unless resources are unlimited, net energy gain will eventually decrease with body size as temperature increases.

Clear results: large body will decline first.
Past mass extinction.
Effect of temperature, and resource quantity and quality.

%Or model is a highly simplified version, compared to biophisical model or energy budget model. 
%Many details are left out, although these are omitted so we can keep track of system.
%There are few tings that are crucial missing.
%The first one is the importance of humidity.
%The second one is competition.
%Growth...
%
%
%A different step is to ask what to do with performance.
%This is no way deciding what is optimal.
%Energy definition but not that one (Kooijiman, yes but not that one).
%Kooijiman, adding survival can play a significant role.
%Buckley simply converted energy to offspring ....
%Here however, given many uncertainties we did not want to include additional parameters.
%
%Mathematical are very useful in guiding intuitions and empirical work and validating them. 
%
%Modeling energy budget is a major shift in ecological thinking.
%Development in physiology and whatever had allowed to develop more more sophisticated and detailed models.
%A great example is the dynamic energy budget by Kooijiman and models each aspect of the life-history.
%One aspect of modeling is however missed in most energy budget model.
%That is the isolate and examine the role of different processes in generating patterns.
%Our goal here is to focus on three main processes: the physiological process of energetic cost, the ecological process of foraging and the behavioral process that relates to foraging.
%
%Our modeling is a simplification existing models, especially that of Buckley. 
%Other general models exist.
%For instance, Brown was trying to model changes in optimal body size for mammals in northern America (refs).
%There is a problem with the actual meaning of the parameters.
%  

%\begin{itemize}
%\item The model is used to explore the role of different factors in shaping the niche: behavior, physiology, temperature, and resource density.
%This more to explore the magnitude of change in the parameters not about their actual values. 
% E: Ah, yes, my point from the Intro!
%Some results are intuitive but the model allows to quantify the effect of the parameters.
% E: Could elaborate on what was intuitive and what wasn't.  Especially intermediate optimal body size and endo vs ecto.
%\item Importance of including multiple factors. 
%Realistic but tractable model.
%\item Talk about missing empirical estimates, e.g., for $b_2$ and $b_3$ and the need to study more foraging, resource abundance and quality.
% E: explain the power-law. Justify.
%\item The role of warm-up (possibility and duration) in defining niche. 
%Empirical estimates of conductance, does it vary with body size? 
%Warm-up and literature about diel activity.  
%Endothermic insects.
%\item Is body size appropriate?
% E: or more generally, how would one take your approach with a different focal trait?
%\item Other weaknesses: humidity, survival, no competition = this is still a fundamental niche. We did not optimize the model. 
%\item Growth!!!
%\item The use of energy as a proxy for fitness. 
%How to convert energy to fitness?
%\item Message for practical application: body size as predictor: will large suffer most?
%\end{itemize}
% E: How far are you willing to stretch to broad biological implications?  Restore anything about Bergmann's rule?
%
%
%Know the fundamental niche, reaction norm, proxy for fitness, performance curve.
%Many models
%Niche Mapper
%Buckley model
%Brown
%
%The goal here is to explore the importance of parameters, not their actual value.
%This is theory for God's sake.
%
%General model
%Complexity for parameterization
%
%Quantitative prediction....not actual values. Say role of exponent b3 on how much species will lose.
%
%Body size and temperature
%
%Endotherm and nocturnal.
%
%
%Behavior: Warm-up, timing of dial and seasonal activity (supported by Kearney)
%Limited or increased foraging time. 
%
%Importance of foraging.
%Resource availabilty. 
%Kearney vegetation cover.
%
%Non linearity in evolutionary model
%
%
%
%
