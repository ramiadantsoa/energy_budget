\section*{Discussion}
\begin{itemize}
\item The model is used to explore the role of different factors in shaping the niche: behavior, physiology, temperature, and resource density.
This more to explore the magnitude of change in the parameters not about their actual values. 
% E: Ah, yes, my point from the Intro!
This is theory for God sake.
Some results are intuitive but the model allows to quantify the effect of the parameters.
% E: Could elaborate on what was intuitive and what wasn't.  Especially intermediate optimal body size and endo vs ecto.
\item Importance of including multiple factors. 
Realistic but tractable model.
\item Talk about missing empirical estimates, e.g., for $b_2$ and $b_3$ and the need to study more foraging, resource abundance and quality.
\item The role of warm-up (possibility and duration) in defining niche. 
Empirical estimates of conductance, does it vary with body size? 
Warm-up and literature about diel activity.  
Endothermic insects.
\item Is body size appropriate?
% E: or more generally, how would one take your approach with a different focal trait?
\item Other weaknesses: humidity, survival, no competition = this is still a fundamental niche. We did not optimize the model. 
\item The use of energy as a proxy for fitness. 
How to convert energy to fitness?
\item Message for practical application: body size as predictor: will large suffer most?
\end{itemize}
% E: How far are you willing to stretch to broad biological implications?  Restore anything about Bergmann's rule?
%
%
%Know the fundamental niche, reaction norm, proxy for fitness, performance curve.
%Many models
%Niche Mapper
%Buckley model
%Brown
%
%The goal here is to explore the importance of parameters, not their actual value.
%This is theory for God's sake.
%
%General model
%Complexity for parameterization
%
%Quantitative prediction....not actual values. Say role of exponent b3 on how much species will lose.
%
%Body size and temperature
%
%Endotherm and nocturnal.
%
%
%Behavior: Warm-up, timing of dial and seasonal activity (supported by Kearney)
%Limited or increased foraging time. 
%
%Importance of foraging.
%Resource availabilty. 
%Kearney vegetation cover.
%
%Non linearity in evolutionary model
%
%


