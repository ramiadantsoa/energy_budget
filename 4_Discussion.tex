\section*{Discussion}
In this work, we developed a model to investigate how thermal performance of foraging adult insect varies with body size.
So far, there is no theoretical work that shows a clear mapping on how performance (optimal temperature, critical minimum, maximum) changes with body size. 
Our model evaluates the role of metabolism, foraging and thermoregulation which all scale allometrically with body size and all depend on temperature.
Unlike classical energy budgets \citep[e.g.,][]{Kooijman2009}, our goal is not to fit a particular empirical data but to explore the role of various processes that goes into the model.
Furthermore, we put a strong emphasis on the role of behavioral thermoregulation for heterotherm in shaping performance by analyzing in details the temporal process of warm-up.
In general, we concluded that such ability to warm-up may have a important influence on the performance in cold environment.
Whereas the upper part is limited by physiological and ecological processes such as resource availability, foraging, and metabolism.
This concur with other models of range limits (refs)% T: Emma, on top of your head?
% E: I'm not sure this is true.  The classic view (Macarthur) is that the warm range limit is shaped by interspecific interactions, while metabolism/physiology shape the cold limit.

Metabolic rate, especially resting, has been the subject of intense empirical and theoretical investigation with a major focus on  the universality of the exponent---know as the 3/4 law \citep{Peters1986,West1997, Kozlowski1997, Brown2004, Isaac2010}. 
Whereas it is interesting on its own, in defining performance across body size, the exponent of the foraging rate is more important.
Concavity favors smaller individual, convexity benefits large and differences are accentuated with increasing temperature (\cref{fig2}).
% E: Might be good to remind the reader what concave/convex means here, in terms of resource gathered per unit body mass.  Could refer to Fig 1.
Unlike the exponent of resting metabolic rate, there is no clear values for the exponent for foraging rate (see parameter justification).
Some theoretical models adopted a single value 0.75 (thus similar to the exponent of resting metabolic rate) but are not based on empirical data \citep{Yodzis1992, Brown1993}.
Recent studies and reviews have shown that the exponent $b_3$  is more variable than resting metabolic rate and factors such as spatial dimensionality of search (2 vs 3), searching ability (e.g. visual acuity or maneuverability),  species interaction (e.g. competition) can all shaped that foraging rate\citep{Pawar2012, Kalinkat2015}.
We believe that there is more variation in these rates that is worth to be looked into more details as there are more potential to drive different patterns.

A non-intuitive and interesting result is the role of resource quality when foraging time is limited.
In general, we found that performance increases with body size but for certain resource quality and a concave foraging rate,  intermediate body size is favored.
If resource quality is too low, there is not enough energetic gain and everyone suffers whereas if resource quality is too high, large are not penalized enough.
If the conditions are satisfied then resource quality alone (and not quantity) can select for different body sizes.
That would mean that optimal body size would shift to lower values are resource quality decreases even if  everything else is equal.
These dual conditions might look restrictive but our analytical results also shows that the range of resource quality allowing the pattern increases with temperature. 
There is no data to confirm these study but the result shows that it is a possibility and in general  underscore the idea that one should look at the ecological context in determining species performance \citep{Sears2015}.
The joint effect of low resource quality or quantity mixed with high energetic demands from  metabolism cost limits performance at high temperature.
%Instead of looking at endogenous factors Comparing performance is incomplete and one does need to consider a better description of the environment (resource-wise) \citet{Sears2015}.
% E: This could be a good place to speculate on implications of climate change and habitat loss.

A unique assumption of this work is that we examine the thermodynamic features of insect to classify quantitative and qualitative patterns of warm-up.
Warm-up process has been investigated empirically during the 70's and 80's for endothermic insects  such as dung beetles, bees, moths \citep{Heinrich1975, Bartholomew1978, Bartholomew1981} but to the best of our knowledge no model describes the entire warm-up phase let alone the effect of body size.
In general, the model validates the intuitive idea about  the effect of surface-area to body size ratio in heat absorption and heat retention.
The ability to complete warm-up increases with decreasing size for ectotherms and with increasing size for endotherms.
However, for ectotherms wind can create enough convection that being small is no more optimal. % E: Emphasize more explicitly that your convection finding opposes the usual intuition.
An important parameter is conductance which control heat exchange between the thorax and the surface.
There are not much data about conductance although  \citet{Bartholomew1978} found although conductance is controlled by very different layers for dung beetles, moth and bees, they have similar cooling rates. %the air beneath the elythra for dung beetles and by an insulating pile for moths and bees,.
In spite of that data, the homogeneity of such value would be surprising. % as coloration can make a difference in heat exchange \citep{Forsman2002}. 
If warm-up is crucial, we found that depending on the timing of warm-up, the optimal conductance should be higher for endotherms when warm-up occurs in the early morning  to increase the heat absorption but the optimal conductance should be lower during the rest of the day to increase heat retention.
A low conductance  can be a problem because insects also need to dissipate heat during activities.
Yet,  other studies on bees and beetles also revealed that the process of cooling happens through often different mechanisms such as abdominal pumping or evaporative cooling \citep{Heinrich1979, Verdu2012}.
....

A central question is whether these thermodynamic features are actually important in real system.  
We relate our model with few empirical studies.
% E: This is great to do.  Clarify that some of these observations can already be "understood" with the "endotherms are large" and "endotherms can operate in colder temperatures" verbal models, but your model also makes more specific predictions.
First, studies on endothermic dung beetles have shown that below a certain mass (about 2 g), individual becomes thermoconformer i.e. not capable of endogeneous thermoregulation \citep{Bartholomew1978, Verdu2006}.
We actually found a threshold condition for warm-up.
In \cref{fig:4}c (dashed line with low convection) warm-up can be completed at any given temperature (here above freezing temperature) when body mass is large.
The threshold is simply the point where heat loss equals endogenous heat production such that  below that value, warm-up is not possible.
Second, partitioning of diel activity has been proposed to facilitate coexistence between sympatric species \citep{Viljanen2009}
Data indeed revealed that smaller  individual are active during the warmer period of the day whereas large ones forage during colder ambient temperature \citep{May1985}.
The match between thermoregulation capacity and thermal niche is connected here via body size which confirms that foraging in colder ambient temperature is not possible for small endotherms.
We are not aware of dataset that would show the opposite pattern for ectotherms. % E: Refer more explicitly to the results, and emphasize that data could be collected to test.
Third, data has shown that a community of dung beetles can be exact on when they start to be active \citep[e.g.,][]{Halffter1966, Caveney1995}.
Our models show a significant difference in the duration of warm-up depending on the hour of the day especially soon after sunrise.
In such a competitive community where resources can be depleted fast \citep{Hanski1991}, arriving at the source and optimizing warm-up time by starting at the right time can be important.
Can we dare to ask if there is actually a cognitive ability in the warm-up process or is simply a mechanical response?  % E: Outside the scope of this whole modeling framework, I think.
Recent reviews emphasize the role of thermoregulation and should be included in evaluation performance \citep{Dial2008, Kalinkat2015}.


In this study, we compared performance across body sizes by calculating their absolute net energy gain.
Clearly, net energy gain is a very simplistic approximation of performance and does not mean fitness
\citet{Kozlowski1996} has already pointed out that energetic definition of fitness is wrong and adding size dependent mortality already shift optimum body size.
Conversion to fecundity, assimilation and more all can affect thermal performance.
% E: Elaborate on how the above might be attempted.
What perhaps is missing the most is that we do not include growth in our model but also what is the point of growing larger or smaller if you underperform as an adult. 
% E: Not "most" in my opinion.  Growth is just another component that could be included in an expanded framework.  Again, I think it would be worth elaborating on what that big framework might look like, and how your current work fits in.
The model is situated in a middle of what \citet{Holling1966} called tactical-strategic spectrum.
Our main goal is to have parameters that are measurable with clear biological meaning yet not too specific that would still allow to get general insight. 
What goes into the model can be an endless list, hopefully, energy is a core currency that is a starting point for future extension.

Although the model is focused on looking at individual, the model helps to pinpoint possible mechanisms behind macroecological patterns.
% E: One pattern is Bergmann's rule.  Is there another pattern?  Large goes extinct?  For the latter especially, the model doesn't really "pinpoint" true causes, but it suggests possible causes that maybe haven't yet been appreciated.
For instance, it can be used test various reasons for seeing or not Bergmann's rule \citep{Blackburn1999}.
In addition to the proposed mechanisms causing Bergmann's rule, we add the concavity of foraging rate can allow small to be optimal under warm environment, a limited resource availability can select for smaller body size as temperature increases. 
% E: Might need to remind the reader about the "proposed mechanisms."
Yet, the same resource limitation can also select for smaller individuals if resource availability correlates positively with temperature.
% E: Perhaps "favor" would be safer than "select for".
The inability to warm-up seems not to be a strong mechanism unless resource availability are temporally patchy and show a degree of synchrony or asynchrony with solar radiation and daily temperature 
As another example, habitat change due to global warming and habitat loss are the main drivers of loss of biodiversity (refs).
We have found here that large tends to be more sensitive due to lack of resource.
Temperature acts multiplicatively such that the increase in the cost is much higher arithmetically for large with the same increase in temperature. % E: explain more accessibly

In summary, our integrative model underscores that parameters such as foraging rate, conductance, resource quality that would require empirical attentions.
Theoretically, there is no single relationship on how thermal performance changes with body size.
It can increase, decrease or shifts depending on the parameter space where metabolic rate, foraging rate, thermoregulation, and resource availability are situated.
% E: Maybe swap the order here: the model shows this, and it guides empirical work into the future

%%%%%%%%%%%%%%%%%%%%%%%%%%%%%%%%%%%%%%%%%%%%%%%%%%%%%%%%%%%%%%%%

%Finally, heat exchange via conductance can even be selected for, some studies have found that color can influence thermoregulation behavior for grasshoppers (Forsman)
%It is undeniable that a good thermoregulation capacity can broaden the niche \citep{May1985} but it is unsure about the specific role of warm-up.
%Coevolution of color pattern and thermoregulatory behavior in polymorphic pygmy grasshoppers Tetrix undulata (Forsman A, Ringblom K, Civantos E, Ahnesjö J.)

%Several empirical studies in the past have emphasized the matching between thermoregulation capacity and thermal niche (refs).
%For dung beetles, it has been proposed difference in thermoregulation can facilitate the coexistence of sympatric  species by niche differentiation .e.g in diel activity (refs) 

%Diel activity relates to taxon, diet, color, body size and functional group (Vulinec 2002, Krell-Westerwalbesloh et 2004, Feer and Pincebourde 2005)
%Verdu 2006 on partitioning, less than 2 grams thermoconformers so have to flight during warm time.
%However, it is not known how species use their thermoregulatory capacity or how species  behave according to the thermal quality of the habitat,  particularly in the case of the endothermic insects. In burying beetles, body size and some morphological features (such as wing loading and insulation) affect their thermoregulation
%pattern and activity times (Merrick and Smith,2004).
%In some stingless bees, thermoregulatory differences related to body size and coloration suggests niche differentiation and different biogeographic  distributions at the interspecific level (Pereboom & Biesmeijer, 2003 ).
%
%Thus, thermoregulation may broaden a thermal niche in both space and time ( May, 1985 ).
%
%The dung beetles of a given community tend to be quite exact in the timing of their daily activities ( Halffter & Matthews, 1966; Fincher et al. , 1971; Mena et al. , 1989; Caveney et al. , 1995 ).
%
%how species behave according to the thermal quality of the habitat, particularly in the case of the endothermic insects (Verdu2007)
 
%%%%%%%%%%%%%%%%%%%%%%%%%%%%%%%%%%%%%%%%%%%%
%
%Evaluating performance has been done in the past.
%There is a bunch of literature that is about growth  (e.g. Kozlowski2004, Vanbertanlafy, van der have). 
%Other models focus on performance as a function of body size (e.g. Yodzis, Brown).
%The same assumption is that performance is defined as the difference between input and output.
%We use net energy gain here as a first proxy of performance.
%We do not mean it as fitness, as Kozlowski states, how mortality scales can change the optimum.
%%More resource can actually decrease mortality...
%A way to convert resource to offspring for instance if fecundity is linear
%Finally a key aspect that is missing is humidity.
%This is a simplifying version of DEB but they are necessary to illustrate the effect of the other factors.
%Competition to reduce resource availability, the only thing to do is to convert foraging rate so that it depends on the set of competitors.
%
%
%
%notes:
%Niche in the discussion
%In this work we wanted understand how thermal performance varies with body size.
%We wanted to answer questions like does thermal niche breadth increase or decrease with body size, does optimal temperature increase or decrease with body size.
%We focus on how the allometry of physiological processes (metabolism), ecological process (foraging), and behavioral thermoregulation (warm-up) shapes thermal performance. 
%Unlike many energy budget models \citep[e.g.,][]{Brown1993,Kooijmann2009}, this is a process-oriented rather than pattern oriented modeling.
%We strive to find a balance between realism and abstraction that makes to model fully parameterizable and flexible enough to generate general insight.
%We are in the middle of what \citet{Hooling1966} called tactical-strategic spectrum (ranging from the need to perform detailed measurements and generally defined parameters that cannot be measured).
%%T: I just like the term tactical-strategic. It can be left off.
%
%CHOWN 2007 Scaling of insect metabolic rate is inconsistent with the nutrient supply network model
%KINGSOLVER 2016	Beyond Thermal Performance Curves: Modeling Time-Dependent Effects of Thermal Stress on Ectotherm Growth Rates
%
%What are the roles of metabolism in defining performance?
%A classical results is that resting metabolic rate scales with body size with a power 3/4  \citep{Kleiber1947, Peters1986, Brown2004}.
%It has been a source of debate for instance whether it is 2/3 or 3/4 \citep[for refs][]{Yodzis1992, Isaac2010}.
%Other theoretical models tried to explain the 3/4 power \citep[e.g.,][]{West1997, Kozlowski1997}.
%Although they are interesting on their own, when the goal is to define performance the picture is incomplete without foraging.
%
%What is the role of foraging?
%Scaling of foraging rate is a determinant factor when comparing performance across body size.
%Theoretical works often assume that the exponent is also 0.75 \citep{Yodzis1992, Brown1993}.
%Such value is not often supported empirically .
%For instance, \citet{Maino2015} found that at the interspecific level, the slope can exceed 3/4 (they look at consumption rate).
%\citet{Pawar2012} did a meta-analysis and found that the exponent depends on the dimension on the search space and can reach up to 1.08.
%Low exponent, say velocity of locomotion can be low (around 0.20)  \citep{Peters1986}.
%Say search time, speed, visual acuity all can shape encounter and consumption rate \citep[ref in][]{Kalinkat2015}.
%\citet{Yodzis1992} mentioned that the ecology context can define the exponent rate without specifying the dependence on body size.
%We explored here the consequences of these different exponents within a range 0.5 and 1.25.
%The concavity or convexity of that function generally determines where performance increases monotonically (i.e. large is always better) or not (intermediate body size is better).
%It is obvious that the smallest (at the limit of body size equals zero) cannot be optimal due to physical and physiological limit.
%Another aspect is that the exponent also influences total foraging time which in turn influences the metabolic cost of activities
%Comparing performance would first require the need to clarify the parameter values within the system under study.
%
%Whereas endogenous variables are important, external environment is also crucial.
%Resources are generally limited.
%When resources are limited, large which has an higher absolute metabolic cost thus have lower net energy gain, and also narrower thermal breadth.
%An interesting result is when foraging is limited in time.
%In general, performance increases with body size but for a certain parameter space intermediate body size is favored.
%Having a concave exponent for foraging rate is not enough (i.e. large are less efficient at gathering resource per unit of mass).
%Resource quality needs to be within a certain range.
%In fact, if resource quality is too low, there is not enough energetic gain and everyone suffers (we did not evaluate negative net energy gain), whereas if it is too high, the large are not penalized enough.
%%Small dung beetles are better at extracting higher nutrient from a dung source than large ones.
%Resource quality alone and not quantity can select for different body sizes even if everything else remain the same.
%Optimal body size would shift to lower values are resource quality decreases.
%Such selection is also more likely in warmer environment.
%Comparing performance is incomplete and one does not consider a better description of the environment (resource-wise) \citet{Sears2015}.
%
%Most energetic budget models (process-oriented) do not include ignores behavioral aspect.
%A novel approach we have is to include the effect of body size on thermoregulation and thermoregulation on performance.
%We developed and analyzed a thermodynamic model that determine the warm-up process.
%Such process has been analyzed empirically during the 80's  for endotherm  such dung beetles, bee, moth \citep{Heinrich1975, Bartholomew1978, Bartholomew1981}.
%A key parameter is conductance between the body and the thorax as it determines the heat exchange between the individual and the environment.
%Much studies on thermoregulation have been done for large ecotherms (lizard) (refs).
%80's studies actually investigated the heat exchange for endothermic insects.
%Although conductance is controlled by the air beneath the elythra for dung beetles and by an insulating pile for moths and bees, cooling rates are similar \citep{Bartholomew1978}.  
%It would be interesting to know  how much conductance differ between ecotherm and endotherm.
%Color...
%%Two moths (income and capital breeding) with different warm-up style (whir vs quiver wings) are similar allometrically in energetics of endothermy \citep{Bartholomew1981}.
%Is there any adaptation for small endotherm to lose less heat, or as for mammals big is better because coat becomes thicker?
%
%The duration of warm-up can be essential.
%A dung beetle can take a while up to 40 min to warm-up \citep{Verdu2008}.
%Our  model shows that the timing of warm-up is important and too long warm-up can affect significantly performance.
%In a highly competitive community such as dung beetles reaching the site as soon as possible confers a great advantage, additionally fresh dung is more palatable \citep{Hanski1991}.
%Temporal partitioning has been reported to facilitate coexistence of sympatric species \citep{Verdu2007, Verdu2012}.
%Two species of dung beetles forage at different time because the first is better are retaining heat and thus is active during cold period and the second is better are dissipating heat (the mechanism is to control of heat transfer to the abdomen) \citep{Verdu2012}.
%Body size constraints on warm-up ability because of different scaling of surface and volume.
%As a consequence is a potential mechanisms to temporal partitioning of activity (also combined with conductance above).
%Our model actually showed that different values of conductance are preferred at different time of the day for endotherms.
%Does it happen in reality? % T: I am not sure where to place that topic, here or above
%It would be interesting to know if species that differs in their diel activity also have different conductance (this is relevant event without considering body size)  
%More studies on warm-up: how it drives diel activity (and its partitioning), is there a correspondence between the conductance and the timing of foraging i.e. are they smart enough so they can optimize the timing of warm-up? 
%Recent reviews emphasize the role of thermoregulation and should be included in evaluation performance \cref{Dial2008, Kalinkat2015}.
%
%
%\textbf{Other things (interesting?):} 
%- verbal argument that small dung beetles just wait until food shows up whereas  large do more active searching.
%- Studies have shown that below 2 g, body temperature depends on ambient temperature and above it becomes independent, role of conductance? 
%
%Macroecological consequences?
%One reason where variation in thermal performance across body size is to investigate macroecological pattern.
%One of the most known pattern is Bergmann's rule.
%There has been debate whether the rule holds or not.
%There are many examples that found the opposite, it can be common in invertebrates (refs).
%There are various explanations \citep{Chown2010}.
%Our models showed that many mechanisms can underly the pattern.
%1- resource limitation penalizes large ones: temperature plays a role in increasing cost for large.
%2- allometric scaling of foraging can penalize large ones: they don't get enough resource. 
%3- reduction in resource quality
%There are ecological factors without the need to call for physiological, if it occurs to be true, theres should be a relationship between the magnitude of decrease in body size and resource availability (can also be due to competition) moderated by the temperature effect.
%%excess of resource and little competition??
%The inverse pattern can also occur if foraging rate is highly convex...
%Use of quantitative model can give insight and verify verbal assumption.
% %Chown Gaston 2010: 
% % Large because excess of resource due to low competition (p98 1st col) thus grow faster and get bigger
% %Sawtooth pattern (intraspecific) voltinism
% %restistance hyp (Cushman et al 1993): large resist to starvation and whole section if ref is needed- and counterexample NVDI and wing length in africa
% %hostplant effect (mentioning quality) arctic as good example, those not on angiosperm. Resource limitation (Repasky 1991)
% %Pop dens scales with -0.75  and since R scales  with 0.75 thus equal ``energetic equivalence rule''- no supported Blackburn and ganston 1998
% 
%What are we missing?
%We are in the middle of what \citet{Hooling1966} called tactical-strategic spectrum (ranging from the need to perform detailed measurements and generally defined parameters that cannot be measured).
%Level of details?
%What next?
%
%Evaluating performance has been done in the past.
%There is a bunch of literature that is about growth  (e.g. Kozlowski2004, Vanbertanlafy, van der have). 
%Other models focus on performance as a function of body size (e.g. Yodzis, Brown).
%The same assumption is that performance is defined as the difference between input and output.
%We use net energy gain here as a first proxy of performance.
%We do not mean it as fitness, as Kozlowski states, how mortality scales can change the optimum.
%%More resource can actually decrease mortality...
%A way to convert resource to offspring for instance if fecundity is linear
%Finally a key aspect that is missing is humidity.
%This is a simplifying version of DEB but they are necessary to illustrate the effect of the other factors.
%Competition to reduce resource availability, the only thing to do is to convert foraging rate so that it depends on the set of competitors.
%
%
%link back to SDM, can't tell what causes absence.
%relates to piece in amarasekare, body size to fec
%       
%Ways to take the model:
%temperature and habitat loss.
%amarasekare again???? Angilletta theoretical work.
%Bergmanns
%What to measure
%Extension.
%       
%Conclusion:
%-more mec for niche theory
%- what to think about when comparing performance across body size.
%- multiplicative effect of temperature stronger effect on large
%- consequences of habitat loss on thermal performance: large will suffer most---past extinction? (ref Angiletta).
%- parameter values are unknown and empirical measurements are needed.
%- upper limit defined by physiology, lower limit defined by behavior and thermoregulation.
%Bergmann's rule.
