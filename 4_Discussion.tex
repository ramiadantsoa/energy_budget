\section*{Discussion}
 In this work we aim to understand how net energy gain changes as function of body size and temperature for adult insect.
 Amongst the myriad of factors that influence the performance, we explore on the role of  physiological, ecological, and behavioral processes and quantify their  relative importances in determining which body size perform best under a given condition.
On one hand, the model relies on power law to model how energetic demand and gain depend on body size and on thermodynamic principle to study behavioral thermoregulation. 
Although, these modeling assumptions already exist in the theoretical literature \citep[e.g.,][]{Brown1993, Kozlowski1997, Bakken1975} a unified integration of temperature, body size, and behavioral thermoregulation has not been done, let alone be the reconstruction of the fundamental niche.
%this can also  be in the intro.

One of our main result is that the exponent of the foraging rate plays a central role in determining performance across body size.
We have found that if the exponent is lower than the exponent for the metabolic cost.
This results is somehow intuitive but we actually do not know where in the parameter space we are as the relationship between body mass and foraging rate is poorly studied.
Studies on dung beetles showed that the exponent can be larger than one \citep{Nervo2014}.
In another study, foraging rate is only inferred from field metabolic cost \citep{Nagy1987}.
Our model indeed shows that a high convexity of the foraging rate increases the performance of large bodied but only when active metabolic rate is extremely high.
A high exponent of active metabolic rate ($b_2 \equiv 1.17$) has been reported for some tropical dung beetles \citep{Bartholomew1978}. 
Factorial scope which is the ratio between maximum and resting rates can actually be up to 43 for moths \citep{Bartholomew1981} although a sustained metabolic scope is thought to be less than 7 \citep{Petterson1990}.
In any case, different parameter values for active metabolic rate and especially foraging rate can already determine whether small or large will perform better under a given condition the need for other assumptions such as mortality or fecundity.

 The story is however more complicated than just foraging rate.
 Two external variables have significant role.
 First, resource availability constrains the performance of large individual.
 That is rather obvious as they have a higher cost.
 Second, the quality of resources gave a non-trivial results
 Intermediate body size is favored only under restricted conditions. 
 That is when the exponent of the foraging rate is small and resource quality is within a certain value.
 If it is too low, there is not enough energetic gain, whereas if it is too high, the large are now penalized enough.
 The likelihood that is higher when it is warm that when it is cold. 
 Would that mean that there are more room for optimal  body size in the tropics or at least in warmer environment?%very speculative, rush ideas
 Different body size can be favored if resource quality is different.
 In general, the conditions may seem to restrictive to occur.
 Nevertheless, the optimality is determined by another quantities resource quantity and quality.
 \citet{Kearney2009b} already raised the issue when about shade availability (vegetation) is important for thermoregulation.
  Response can be mediated by these factors  rather than the usual focus on temperature and endogenous processes (refs.).
 Identifying which is limiting in empirical system is crucial if one wants to predict responses to environmental changes.

In addition to physiological, behavior can influence the fundamental niche.
We looked at a specific process that is common to all cold-blooded animals such as insect: that is warm-up (refs.).
The model is highly simplified from general models of heat exchange between the individual and the environment \citep[e.g.,][]{Campbell2012}.
The general model often focus usually is on operative (equilibrium) temperature \citep[e.g.]{Angilletta2009} rather the actual process.
However, the the ability to forage and time left for foraging can be crucial in defining performance (say resource quality declines through time).
Other than the realistic model, the argument is often verbal and based on surface-area-to-body size ratio (refs?).
Most of our results aligns with  such intuition. 
Large endotherm have better capability than smaller ones because they lose less heat and the conversely applies for ectotherm, small is better because it can absorb more heat.

The intuition is however inaccurate when wind is considered. 
The relationship becomes non-monotonic. 
Small ectotherm will suffer from high convection which cancels the advantage from high surface area-to-body ratio.
%one hypothesis is that in very windy place body size does not play important role
The most important aspect is however is time.
Whereas activation is the first barrier time is a more important thing.
In general, small individual warms up faster---as long as they can warm-up. 

Conductance plays a significant role and showed a mixed strategy for endotherm.
A balance between endogenous heat production and solar absorption. 
Dilemma whether increase or decreases conductance.
eLYTRA AND WINDS INCREASES COOLING RATES

The warm-up behavior has been studied in the past, for bees, moth and beetles \citep{Kammer1974, Bartholomew1978, Bartholomew1981}.
The theory we provided here helps to explore how the process changes with body size and the environment.
An open question is whether these constraints also dictates the diel activity of species in addition to temporal partitioning due to competition. 
For instance different species of dung beetles forage at different time of the day \citep{Viljanen2010} and is assumed to be caused by competition.
The model is rather simple but one key question is whether they can optimize their warm-up , in terms of warm-up time, but also physical property that is conductance.
%Conductance for mammals is different (Bakken), and assume here it is independent of body size.
%This has not be investigated in the literature.

Energy budget model are everywhere and are becoming more and more elaborated.
One example is NicheMapper which also includes DEB by Koojiman.
Our model is a very close relative to those energy budget model and use the same principle.
They model cannot function without detailed parameters.
The role of our model is to generate assumptions, verify intuition.
 There are other simplified models that search for performance. 
Brown for mammals but parameters do not have clear meaning, the goal was to predict body size distribution 
Kozlowski (AmNat) used production rate to explain interspecific allometric exponents by body size optimization.
Temperature was not included.  
The model is aimed to seek a balance between theory and data.
Integrate different processes 
Although body size is one of the most scrutinized traits,  how thermal performance changes with body sizes remain unexplored.

What are we missing?
It is a first approximation. 
Two main components that are missing to translate to performance.
First is conversion to offspring (i.e. fecundity).
Second is mortality.
We omitted them at the moment as we put our focus on physiology and limit the number of free parameters.
Also assimilation which can be temperature dependent.
and growth.
Power law valid?
humidity
Only for income-breeding insects.
 
 U-shaped pattern of thermal performance has been reported although there is not study of how it arises.
 Our model can be served as null model.
 For instance whether there is a trade-off in performance (we did not really assume anything so that did not happen.
 Another example is where there will be a generalist or specialist (did not happened to with respect to body size)
 What is often seen is that niche shrinks.
 The upper bound is limited by physiological  constraint whereas the lower by behavioral.
 
 Individual variablity and trade-off Metcalf
  
 In most quantitative model, r and K are not dependent on the trait (ref).
 Our model can help to derive a better value carrying capacity or fecundity.
 

One of the most well-known pattern is Bergmann's rule (refs).
Large are found in colder environment although there are also counter examples especially for cold-blooded animals (refs).
The work provides a series of alternative hypotheses on what can generate or not Bergmann's rule.
The prevailing pattern that is metabolism is a concave function of body size as well surface-area-to-body but it is incomplete.
The other theory is only about growth but it does not work if being larger leads to low fitness.
We found that these are important but cannot be evaluated with considering resource (quantity and quality).
The absence of large individual at warm temperature can be due to the high metabolism with low resource limitation preventing them to establish \citep{Angert2005}. 
The absence of small in cold environment can be due to the inability to warm-up.
However, depending on whether they are ectotherm or endotherm can generate different patterns (not sure if it is limiting though see above)
...

Conclusions:
There is no unique pattern.
general theory about body size and temp, as scaling here
at population scale, some things to consider.
It depends on exponent of foraging rate, resource quantity and quality, and warm-up limitation for instance due to wind or body conductance.
The values of these parameters  are not known empirically except for few species.
The model can be extended.
The model can guide empirical study, that is why we search for a balance between theory and data.
Conservation message: large will suffer most for habitat loss and global warming.
The model can help to quantify that. 
theoretical core
 

WHAT IS MISSING NOW IS THERMAL BREADTH, CTMIN AND CT MAX


Barth and Casey active vs rest beeetels in Panama
exponent is higher when active but might be due to temperature because large are active at high temp.
Wing load scales linearly with take off temp,  linear between mass and sustained activ metathoracic temp, exponential between mass and  take-off
no diff between take-off temp and flight  
b2 and b1 similar in moth, bird, reptiles (r1)
factorial scope increases with body size but might be a Q10 effect.

SEPARATE THE EFFECT OF TEMPERATURE AND BODY MASS

Barth and Heinrigh endothermy in african dung beetles during activity
mass can reach 20 g
ball rolling velocity increases with metathorcc temperatyure (but indepedent of ball mass but maybe because the terrain in the lab was flat)
data about warm-up rate 3.7 to 5.5 C per min
2-g rules (2.5 to be accurate) metathoracic take-off temperature linear with body mass and then uniform above 2 g (loss and gain ,passive for small , no regulation)--moth flatten at 1.0g
meethatoracic temp does not differ between noct and diurn
Table 1 range of body size up to 10 fold 
body mass might be more important than wing load for the observed elevation of temperature body mass
within species, wing loading increases with body mass but not metathoracic temp -> evidence of physiological thermoregulation
manoeuvaerability better when small despite low temp (annecdote: large crash and not willing to fly again ,small agile)
Sun elevates metathoracic temp (higher diff between ambient temp), lower in the shade 
nocturnal (te one they found)  frenzie in making ball, higher temp diff with ambient than noct

cooling rate decreases with body size (exponential) air beneath elythra reduces heat loss  
moths and bees are covered with insulating pile (r1) but cooling rates are similar with db
No separation between abdomen and thorax, (unlike moth and bees)

endothermy is know since the 2nd world war (r2)
walking does not need elevated temp (25 ambient is enough)

ISOMETRIC contraction (r3) immobile when warm-up

heat transfer BAKKEN 1976

Kammer Hein bumbelbees muscules activity
10 cal/ gth per spike
Q10 3.8 for 15-25 C and 3.4 for 25-35 C
Correlation between TTh and  spikes frequency
typical spike freq 20 per sec
no diff between queen and workers per mass of thorax for 02 consumption
Muscle theremal specialization more effficitenct

 Heinrich Thermoregulation in Bumblebeses II
 
fly as low as 2 C. 
duration of continuous fllight limited by high ambient temperature ( but not long as well when < 10 C)
abdomen temp = ambient when low temp but equal to thorax at high amb temp
temp free flight diff to foraging, brood incub and tethered flight

bee warm-up (l;arge queen 0.25-0.6g) succeed at 6C but not workers(0.1-0.13g)

proof that Tth is regulated because at amb 3 C , temp diff between Tth and Tabdominal is really high, low at amb 35C. same heat prod
Increased weight (load from resource) implies higher Tth
Oxygen comsumption not correlated with Ta during free flight.
 data 2.9(24C)  cal to 15.7 cal(6.5C) for warm-up. About 375 cal/hr/g for free flight. For moth (warm up 15 cal/g Tth/hr)
 
 Barth, Vleck and Vleck: moth O2 consumption pre and post flight 
 o2 consumption can be 70 times higher than resting
 factorial scope independent of mass and Tth. 39 and 43 in moth
 of course absolute scope depends on Tth
 two moths (income and capital breeding)  with different warm-up style (whirr vs quiver wings) are similar allometrically in energetics of flight and endothermy
 Energy expended during cooling is  .7 of that of warm-up but is equivalent to 1-2 min of flight so negligible?   
 

Verdu Lobo
physiological variability within species (there is  thermoregulation difference between male and female but not a rule, equal thermoregulation is found even for sexually dimorphic--facilitation of mating, difference between mature and immature-immature less capable)
Temperature regulates the kinetic and metabolism of an organism (brown ecology)
endothermy:evolves to expand thermal niche after climatic cahnges
generally, large should have wider range   (Lumaret and Lobo 1996 Biod Let)
Endothermy should prevent species to occur in warm env (durnal), lower temperature does not show a relationship
Avoid overheating in warm condition- evolutionary adaptiation (heat dissipation) and endogenous heat production for competition
theoretical core
 
 
 Chown Gaston: 
 interaction between macroecology and physiology in insects
 shows pattern of body size for intra and inter against latitiude
 abiotic vs biotic limit
 resistance leads to tolerance to dessication and then to reduction in metabolic rate 
 Turner 1987 for diurnal pattern if thermal env and flight pattern also occurs at night
 Latitudunal size clines
 speed of growth vs differentitaion leads to larger at cold Van der Have.
 Large because excess of resource due to low competition (p98 1st col) thus grow faster and get bigger
 Sawtooth pattern (intraspecific) voltinism
 restistance hyp (Cushman et al 1993): large resist to starvation and whole section if ref is needed- and counterexample NVDI and wing length in africa
 hostplant effect (mentioning quality) arctic as good example, those not on angiosperm. Resource limitation (Repasky 1991)
 Pop dens scales with -0.75  and since R scales  with 0.75 thus equal ``energetic equivalence rule''- no supported Blackburn and ganston 1998
 not a single variable for the pattern of body size
 
 Savage, Brown 2004 r and K as a function of (z,T)
 a1 a2 claims to be species specific
 
 
 
 Thoughts: curse of dimensionality. what does model do is to pinpoint the most important factors or assess the role of different factors
 Yodzis
 tactical starategic about two extremes of modeling (Holling 1966)
enregy is the most important (above nutrient) CAlow and Townsend 1981
r3 for ref about exponent of b1
r4 about maximal ingestion rate