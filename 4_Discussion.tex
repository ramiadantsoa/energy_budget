\section*{Discussion}
% ENVIRONMENTAL FILTERING FROM NATHAN CRAFT CAN ALSO BE USED TO DISTINGUISH THE NICHE
Mathematical models are very useful in showing the role of various processes in generating empirical patterns.
These models lay on a continuum from pure abstract to highly realistic.
In the latter, the purpose is often to mimic in great detail a particular phenomenon and the model requires intense parametrization.
Energy budget model has received great attention in explain natural phenomenon.
Energy budget model use a common currency, energy, to model phenomenon from molecular scale to ecosystem (Nisbet)
The cost however it is that it is not generalizable.
Our approach here lies closer to the abstract type of modeling.
We explore how performance, net energy gain, may depend on body size, temperature, and resource quantity and quality. 
In a different perspective, how the dependences on body size and temperature shapes thermal performance.
How variation in the coefficient and exponent would affect the observed pattern.
Given everything else is equal, what do we get....

- General thing about our approach
Looking at the big picture, our model focuses on adult income-breeding insect.
Insect has been neglected with those energy budget model relative to more charismatic species.
The reason is for clarity and transparency of the model as we can easily quantify energetic gain.
In addition, we avoid the growth process.
Although growth is important and even used as a proxy for fitness.
The performance as an adult will eventually determine the true performance.
We can look at the model such that we fix adult body size and find which one does best. 

Net energy gain as performance, how to get to fitness.
Obviously, many components are missing.
Humidity, fecundity and survival.
Large dung beetles tolerates heat, more energy reserve, more competitive.
Optimality can shift if we include mortality (e.g. kooijiman).
Our point is that this is starting point and including all those processes can be overwhelming and there is not much empirical support.


- Missing parameters and main conclusion

 - The role of behavior
 Conductance
 
 - We talk about niche breadth but performance can also be measured.
 Back to Bergmann's rule.
  EXAMPLES AND COUNTEREXAMPLES OF SIZE-RULES 
 and the role of resource availability.
 
 - Talk about performance and U-pattern
 NICHE SHIFTS AS A FUNCTION OF RESOURCE AVAILABILITY, (ELLIOT 1982)
  
 - 
ANGERT 2005 MORTALITY PREVENT ESTABLISHMENT AT LOW ELEVATION, REDUCED GROWTH AND FECUNDITY AT HIGH ELEVATION

Additional applications.
Mathematical models of trait evolution, our model can be a basis to derive optimum body size along temperature gradient, derive carrying capacity as a function of trait value. 
More generally, we can look at generalized fitness landscape.
TRADE-OFFS ARE NOT ASSUMED, ARISES FROM MODEL STRUCTURE

Clear results: large body will decline first. 
We can quantitatively estimated the decrease in performance as function of the changes in the environment  (temperature and resource).
Past mass extinction, large goes first.
Effect of temperature, and pay closer attention to resource quantity and quality.

But it does not scale how it changes with body size and temperature, metabolic scope can be cause by difference in coefficient or exponent.
Our main conclusion however is that the functional form of the energetic cost is only relevant when related to energetic gain, i.e. how the exponents of the various power laws.  

How can we deviate from power law? 
Sigmoid, hump-shaped...
Another approach is to use active metabolic rate to model foraging rate.
It is also an additional assumption and shrink the breadth of possible patterns.

A particular feature of the model is to study in depth the warm-up ability of the insects.
Our model puts under the same framework endothermic and ecothermic insects.
As we simplify energy budget model, so did we with heat exchange between the individual and environment.
The model is based on Campbell for ectotherms and Bartholomew for endotherms.
Thus, rather than having realistic model, we search for pattern yet keep key variables such as surface area-to-volume ratio, solar radiation, and the role of convection.

We are trying to answer patterns of activity for insects.
Diel activity is partitioned so that large can only forage later during day because they are limited by warm-up (refs).
For dung beetles, warm-up ability can potentially explain the divide where large individual are more active forager and small individual pitch and wait (refs).
This can be due to two things.

First, whether warm-up cannot be completed when you are small.
Our model showed that there is a difference in minimum temperature for warm-up. 
It is most likely not limiting, especially in warm environment.
Moreover, for ectotherm, being small supposedly confers advantage due to higher surface area-to-body ratio. 
This is not always true when convection plays an important role and can indeed limit the warm-up ability of different body size.

Second, whether the duration of warm-up matters.
We found that it is more a penalty for large body.
The larger you are the longer it would take to warm-up.
In highly competitive community like dung beetles, an hour delay due to warm-up would greatly reduce the amount of resource available. 

Another explanation is competition.

An assumption we make is that conductance, which drives most of the pattern does not depend on body size.
For mammals, fur and skin thickness can vary with body size (Bakken1976?).
What is the situation with insects, does coloration also matters?
Can insects optimize the timing of warm-up?

Until now we reported that many relationships are not measured empirically.
Model can guide what should be measured (in our model, each parameter has a clear meaning that `can' be measured).
On another hand, we can still explore the consequence of these assumptions even if it is not empirically supported (by lack of data, not because it is wrong).

When we put all these components together, we found that thermal performance is upper bounded because physiological needs outweighs the energetic supply.
This often put large individuals into troubles especially when resources are limiting.
It makes sense, large needs more.
The lower limit is set by behavior in cold environment.
Here, large needs more time and if we subtract that from the total foraging time,  there will be a decrease in performance as it gets colder.
Small surprisingly are not much affected as warm-up time is not affected much.

The niche when conditions are more extreme at both thermal ends generate a U-shaped pattern that has been reported empirically (refs).
A central question we ask is how  niche as a function of temperature varies with body size.
We found that multiple scenarios are possible.
It all depends on what we fed the model with in terms of modeling assumptions and model parameters.
Unless resources are unlimited, net energy gain will eventually decrease with body size as temperature increases.

There are two surprising results.
First, for fixed foraging time (even without warm-up) it is quite unlikely that net energy gain peaks at intermediate body size.
Why does it depend on resource quality (I don't have a clear intuitive explanation at the moment)?
Second, the role of conductance differs at different time of the day for endotherm.
Can we measure that?  

Looking at the big picture, our model focuses on adult income-breeding insect.
Insect has been neglected with those energy budget model relative to more charismatic species.
The reason is for clarity and transparency of the model as we can easily quantify energetic gain.
In addition, we avoid the growth process.
Although growth is important and even used as a proxy for fitness.
The performance as an adult will eventually determine the true performance.
We can look at the model such that we fix adult body size and find which one does best. 

Net energy gain as performance, how to get to fitness.
Obviously, many components are missing.
Humidity, fecundity and survival.
Large dung beetles tolerates heat, more energy reserve, more competitive.
Optimality can shift if we include mortality (e.g. kooijiman).
Our point is that this is starting point and including all those processes can be overwhelming and there is not much empirical support.

Body size is our central focus here.
Explanatory power of body size.
Most measured traits and correlates with physiological functions.
Physical shape is governed by the law of thermodynamics.
This is a reference, given that the assumptions are true, one can then search for missing mechanisms if body size is not the main driver.

Bergmann's rule and pattern of body size.
We simultaneously explore 3 out of 4 mechanisms explaining body size.
Thermoregulation, physiological constraint, resource availability (what is missing is seasonality).
Without behavior, large always perform better in cold environment.
Resource availability is the most obvious as energetic demand increases with temperature.

Additional applications.
Mathematical models of trait evolution, our model can be a basis to derive optimum body size along temperature gradient, derive carrying capacity as a function of trait value. 
More generally, we can look at generalized fitness landscape.

Clear results: large body will decline first. 
We can quantitatively estimated the decrease in performance as function of the changes in the environment  (temperature and resource).
Past mass extinction, large goes first.
Effect of temperature, and pay closer attention to resource quantity and quality.


%Empirical inquiry.
%Individuals are different, we look at the general mean here. 
%It would be nice to see how the variance changes.
%
%What can we learn from models?
%cite Angiletta...
%In conclusion, we noticed that many parameters needs to be estimated.
%b2, b3, eps, K1.


%Or model is a highly simplified version, compared to biophisical model or energy budget model. 
%Many details are left out, although these are omitted so we can keep track of system.
%There are few tings that are crucial missing.
%The first one is the importance of humidity.
%The second one is competition.
%Growth...
%
%
%A different step is to ask what to do with performance.
%This is no way deciding what is optimal.
%Energy definition but not that one (Kooijiman, yes but not that one).
%Kooijiman, adding survival can play a significant role.
%Buckley simply converted energy to offspring ....
%Here however, given many uncertainties we did not want to include additional parameters.
%
%Mathematical are very useful in guiding intuitions and empirical work and validating them. 
%
%Modeling energy budget is a major shift in ecological thinking.
%Development in physiology and whatever had allowed to develop more more sophisticated and detailed models.
%A great example is the dynamic energy budget by Kooijiman and models each aspect of the life-history.
%One aspect of modeling is however missed in most energy budget model.
%That is the isolate and examine the role of different processes in generating patterns.
%Our goal here is to focus on three main processes: the physiological process of energetic cost, the ecological process of foraging and the behavioral process that relates to foraging.
%
%Our modeling is a simplification existing models, especially that of Buckley. 
%Other general models exist.
%For instance, Brown was trying to model changes in optimal body size for mammals in northern America (refs).
%There is a problem with the actual meaning of the parameters.
%  

%\begin{itemize}
%\item The model is used to explore the role of different factors in shaping the niche: behavior, physiology, temperature, and resource density.
%This more to explore the magnitude of change in the parameters not about their actual values. 
% E: Ah, yes, my point from the Intro!
%Some results are intuitive but the model allows to quantify the effect of the parameters.
% E: Could elaborate on what was intuitive and what wasn't.  Especially intermediate optimal body size and endo vs ecto.
%\item Importance of including multiple factors. 
%Realistic but tractable model.
%\item Talk about missing empirical estimates, e.g., for $b_2$ and $b_3$ and the need to study more foraging, resource abundance and quality.
% E: explain the power-law. Justify.
%\item The role of warm-up (possibility and duration) in defining niche. 
%Empirical estimates of conductance, does it vary with body size? 
%Warm-up and literature about diel activity.  
%Endothermic insects.
%\item Is body size appropriate?
% E: or more generally, how would one take your approach with a different focal trait?
%\item Other weaknesses: humidity, survival, no competition = this is still a fundamental niche. We did not optimize the model. 
%\item Growth!!!
%\item The use of energy as a proxy for fitness. 
%How to convert energy to fitness?
%\item Message for practical application: body size as predictor: will large suffer most?
%\end{itemize}
% E: How far are you willing to stretch to broad biological implications?  Restore anything about Bergmann's rule?
%
%
%Know the fundamental niche, reaction norm, proxy for fitness, performance curve.
%Many models
%Niche Mapper
%Buckley model
%Brown
%
%The goal here is to explore the importance of parameters, not their actual value.
%This is theory for God's sake.
%
%General model
%Complexity for parameterization
%
%Quantitative prediction....not actual values. Say role of exponent b3 on how much species will lose.
%
%Body size and temperature
%
%Endotherm and nocturnal.
%
%
%Behavior: Warm-up, timing of dial and seasonal activity (supported by Kearney)
%Limited or increased foraging time. 
%
%Importance of foraging.
%Resource availabilty. 
%Kearney vegetation cover.
%
%Non linearity in evolutionary model
%
%
%
%
