\section*{Discussion}

We have developed a model to investigate how the thermal performance of a foraging adult insect varies with body size.
There is a dearth of theoretical work showing clearly how performance (e.g., optimal temperature, critical minimum or maximum temperature) changes with body size.
We take a first step by exploring the role of metabolism, foraging, and thermoregulation, which all scale allometrically with body size and all depend on temperature in shaping the net daily net energy gain of an individual.
We work up from basic physiological and ecological processes, and we additionally put a strong emphasis on the role of thermoregulation for heterotherms in shaping performance by analyzing in detail the temporal process of warm-up.
We identify several key parameters that have not been measured but which have important roles in shaping thermal performance across body size.
We quantify how warm-up can decrease performance in cold environments.
We further find that the performance curve in warm environments is limited by physiological and ecological processes such as resource availability, foraging, and metabolism.
We do not claim that these are the only processes that shape thermal performance---other factors like competition, heat stress, and cold tolerance can be just as important---but net energy gain can be a first proxy for the actual (realized) thermal performance.

In this study, we compared performance across body sizes by calculating net energy gain.
Clearly, net energy gain is a very simplistic approximation of performance or fitness.
\citet{Kozlowski1996} has pointed out that an energetic definition of fitness is incomplete, and adding size-dependent mortality shifts optimum body size.
The energy currency in hand can be converted to, for instance, fecundity, by using additional power law relationships \citep{Kooijman2009}.
A notable missing component is the growth and development rate, but such questions have been investigated heavily \citep{VandH1996, Kozlowski2004,Kooijman2009}.
Another interesting extension would be to explicitly include competition in the model, as well as finding the optimal time to start warm-up and foraging, or to integrate net energy gain for a longer time span.
Using energy as a core currency for understanding species performance is appealing, and many studies have embraced this approach.
Energy budget models are now used to predict future species distributions \citep[e.g.,][]{Buckley2008}, but the extreme detail and large number of parameters in DEB models \citep{Kooijman2009} prevents wide application.
At the other end of the spectrum, there are parameter-poor models that can generate general insight, but their interpretation and application are hindered because the meanings of the parameters are unclear \citep[e.g.,][]{Brown1993}.
Our model is situated in the middle of this so-called tactical-strategic spectrum \citep{Holling1966}.
Our main goal is to use parameters that are measurable with clear biological meaning yet not too specific to allow us to get general insight when integrating the role of physiology, ecology, and behavior into one framework.

\subsection*{Warm-up}

Our work is unique in examining the thermodynamic features of insects to classify quantitative and qualitative patterns of warm-up. % E: Is "classify" really the word you want?  And, does "theory" or "model" need to go in this sentence for "unique" to be true?
The warm-up process was investigated empirically during the 1970's and 80's for endothermic insects  such as dung beetles, bees, and moths \citep{Heinrich1975, Bartholomew1978, Bartholomew1981}, but to the best of our knowledge no other model describes the entire warm-up phase let alone the effect of body size.
In general, our model validates intuition about the effect of surface area-to-body mass ratio in heat absorption and heat retention: the ability to complete warm-up increases with decreasing size for ectotherms and with increasing size for endotherms.
Our model also quantifies these relationships and thus provide a blueprint for empirical testing.
Further, we found an unintuitive intermediate scenario in which warming up ability (in terms of completion and not duration of warm-up) is best attained at intermediate body size.
Although the surface area-to-body mass ratio can benefit small ectotherms, it also acts against them in the presence of wind because convection becomes more effective as the ratio increases.
Thus, even though the intuition is true in the simplest situation, we show that the addition of the influence of wind, which should be prevalent in many natural conditions, can generate a more subtle pattern.

An important model parameter is conductance, which controls heat exchange between the thorax and the surface.
Data about conductance for insects are scare.
\citet{Bartholomew1978} found that even though conductance is controlled by very different layers for dung beetles, moths, and bees, they have similar cooling rates. %the air beneath the elythra for dung beetles and by an insulating pile for moths and bees,.  % E: "anatomical layers" or some other specific adjective?
In spite of those data, the homogeneity of such a quantity would be surprising. % as coloration can make a difference in heat exchange \citep{Forsman2002}.
If warm-up is crucial, endotherms can face a trade-off between low conductance that improves the ability to retain endogenously generated heat when solar radiation is weak, and high conductance that favors heat absorption when solar radiation is strong.
% when it leading to different optimal conductance vaulue we found that depending on the timing of warm-up, the optimal conductance should be higher for endotherms when warm-up occurs in the early morning  to increase the heat absorption, but the optimal conductance should be lower during the rest of the day to increase heat retention.
Low conductance  can be a problem because insects also need to dissipate heat during activity.
Yet, other studies on bees and beetles revealed that the process of cooling often happens through different mechanisms such as abdominal pumping or evaporative cooling \citep{Heinrich1979, Verdu2012}.
A different possibility that we have not explored here is whether conductance changes with body size.
However at this stage, we believe that the need for more empirical work supersedes the need for additional modeling complexity.

A central question is whether these thermodynamic features are actually important in real systems.
Large endothermic insects have been reported to have the ability to thermoregulate.
Studies on endothermic dung beetles have shown that below a certain mass (about 2 g), individuals become thermoconformers, i.e., not capable of endogeneous thermoregulation \citep{Bartholomew1978, Verdu2006}.
This tipping point can actually occur when endogenous heat production, which increases with body size, equals dissipation of that heat, which decreases with surface area-to-body mass ratio.
The completion of warm-up thus becomes independent of temperature when an individual reaches a certain size (dashed line \cref{fig3}a).
The same ability to warm up is known to allow large individuals to forage during colder periods of the day \citep{May1985}.
Furthermore, we found that the inverse relationship occurs for ectothermic insects such that smaller individuals can forage at lower temperature.
We are not aware of any dataset that documents foraging time or temperature as a function of body size for ectotherms and endotherms.
We hope that our model predictions can eventually be tested both in controlled and natural conditions.
% T: I removed the sentences below because the results relating the time of warm-up were removed, or what do you think?  E: okay
%A last example where the process of warm-up can play a significant role is in regulating diel activity. % E: "last example" would make more sense if "first example" or similar were used above
%For instance, precise temporal partitioning of activity has been shown in a community of dung beetles \citep{Halffter1966, Caveney1995}.
%The generally assumed mechanism behind temporal partitioning is to reduce competition.
%However, we found a significant difference in the duration of warm-up depending on the hour of the day (especially soon after sunrise) and on body size.
%We thus suggest an alternative explanation for diel partitioning: in a competitive community where resources can be depleted quickly \citep{Hanski1991}, optimizing warm-up time to arrive at the source can become crucial.

\subsection*{Broader ecological context}

Metabolic rate, especially resting rate, has been the subject of intense empirical and theoretical investigation with a major focus on the possible universality of the exponent---known as the 3/4 law \citep{Peters1986,West1997, Kozlowski1997, Brown2004, Isaac2010}.
Whereas the exponent of the resting metabolic rate is interesting on its own, our model suggests that, when comparing performance across body size, the exponent of the foraging rate is even more important.
We assumed that foraging rate always increases with body size ($b_3 > 0$), but the increase can be concave or convex (\crefp{fig1}b).
Concavity favors smaller individuals because it means that per unit mass, the efficiency in resource gathering increases with decreasing body size.
The converse is also true that convexity benefits larger individuals (\crefp{fig2,fig5}c).
Unlike the exponent of resting metabolic rate, there is no clear value for the exponent for foraging rate (see section Model Description: Power law and parameter justifications).
Some theoretical models adopted a single value of 0.75, similar to the exponent of resting metabolic rate, but the choice was not based on empirical data \citep{Yodzis1992, Brown1993}.
Recent studies and reviews have shown that the exponent $b_3$ is more variable than the resting metabolic rate exponent $b_2$, and that factors such as the spatial dimensionality of search (i.e., 2- vs. 3-dimensional), searching ability (e.g., visual acuity or maneuverability), and species interactions (e.g., competition) can all shape the exponent of foraging rate \citep{Pawar2012, Kalinkat2015}.
We suggest that how resource acquisition scales with body mass is worth much further empirical investigation because of its potential to generate different qualitative patterns in performance curves.

A non-intuitive and interesting result is the role of resource quality when foraging time is limited.
In general, we found that performance increases with body size, but for a certain range of resource quality and a highly concave foraging rate, performance peaks at intermediate body size.
If resource quality is too low, not enough energy is acquired and individuals of all sizes have negative net energy gain.
In contrast, if resource quality is high, individuals of all sizes have positive net energy gain, but the low mass-specific metabolic cost confers the advantage to large individuals.
If resource quality is within an certain range (see Appendix), resource quality alone (and not quantity) can select for different body sizes.
Optimal body size would then shift to lower values as resource quality decreases even if everything else remains constant.
These dual conditions might look restrictive, but our analytical results also show that the range of resource quality allowing this outcome increases with temperature.
There are no data to confirm this theoretical finding, but the fact that it is a possibility underscores the idea that one should look at the ecological context in determining species performance \citep{Sears2015}.

Although we focus on individual performance, the results have larger-scale implications.
For instance, Bergmann's rule is the macroecological pattern that animals tend to be larger in colder environments \citep{Bergmann1847, Blackburn1999}.
Several mechanisms have been proposed to generate the body size cline.
From an energetic perspective, large individuals do better in cold environments than smaller ones because they are better at conserving heat (due to lower surface area-to-volume ratio or to lower conductance) or better at resisting starvation.
Although our principal goal is not to explain Bergmann's rule, our model draws attention not only to energetic cost but also to energetic gain.
Our results suggest that efficiency in foraging and resource availability have the potential not only to explain Bergmann's rule but also its inverse, which has been documented for insects \citep{Cushman1993, Loder1997,Blackburn1999}.
Future model investigations could explore the optimal body size for a given set of parameter values and thus propose the shape of its dependence on temperature.


In summary, we attempted to understand how body size and temperature shape performance by developing and analyzing a mathematical model.
We found that there is no single theoretically-expected relationship of how thermal performance changes with body size.
Niche breadth can increase, decrease, or shift depending on the parameters for metabolic rate, foraging rate, thermoregulation, and resource availability.
We have illustrated here how the model can be used to verify verbal arguments such as the relationship between body size and warm-up behavior, and also to reveal patterns that arise beyond simple intuition such as the importance of resource quality and size-specific foraging rate in determining optimal body size.
However, the major contribution of this model is the ability to extend feedback between theory and data. % E: can you think of a more exciting word than "extend"?  "fuel" above was good...
We hope this work is helpful in highlighting potentially important parameters to measure, and also by providing a clear theoretical relationship among the variables that will guide future empirical work.
