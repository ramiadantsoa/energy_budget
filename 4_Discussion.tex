% consider transitions between paragraphs
% note to E: not too-too-long sentences!

\section*{Discussion}

We have developed a model to investigate how the thermal performance of a foraging adult insect varies with body size.
There is a dearth of theoretical work showing clearly how performance (e.g., optimal temperature, critical minimum or maximum temperature) changes with body size.
We take a first step by exploring the role of metabolism, foraging, and thermoregulation, which all scale allometrically with body size and all depend on temperature in shaping the net daily net energy gain of an individual.
% Unlike classical energy budget models \citep[e.g.,][]{Kooijman2009}, we opt for a process rather than pattern oriented modeling. % E: too vague for a criticism
We work up from basic processes, and additionally put a strong emphasis on the role of thermoregulation for heterotherms in shaping performance by analyzing in detail the temporal process of warm-up.
We identify several key parameters that have not been measured but which have important roles in shaping thermal performance across body size.
In general, we quantify how warm-up and its timing can decrease performance in cold environments.
We further find that the performance curve in warm environments is limited by physiological and ecological processes such as resource availability, foraging, and metabolism.
We do not mean that these are the only processes that shape thermal performance---other factors like competition, heat stress, and cold tolerance can be just as important---but net energy gain can be a first proxy for the actual (realized) thermal performance.

Metabolic rate, especially resting rate, has been the subject of intense empirical and theoretical investigation with a major focus on the possible universality of the exponent---known as the 3/4 law \citep{Peters1986,West1997, Kozlowski1997, Brown2004, Isaac2010}.
Whereas the exponent of the resting metabolic rate is interesting on its own, our model suggests that, when comparing performance across body size, the exponent of the foraging rate is even more important.
We assumed that foraging rate always increases with body size ($b_3 > 0$), but the increase can be concave or convex (\crefp{fig1}b).
Concavity favors smaller individuals because it means that per unit mass, the efficiency in resource gathering increases with decreasing body size.
The converse is also true that convexity benefits larger individuals (\crefp{fig2,fig5}c).
Unlike the exponent of resting metabolic rate, there is no clear value for the exponent for foraging rate (see section Model Description: Power law and parameter justifications). % E:  \label and \cref also work for sections, by the way, T: I did not work after few tries, maybe because I don't number the section.  E: Oh, right.  You can use \section{...} and \setcounter{secnumdepth}{0} to not print the numbers.  Or your solution is also effective. :)
Some theoretical models adopted a single value of 0.75, similar to the exponent of resting metabolic rate, but the choice was not based on empirical data \citep{Yodzis1992, Brown1993}.
Recent studies and reviews have shown that the exponent $b_3$ is more variable than the resting metabolic rate exponent $b_2$, and that factors such as the spatial dimensionality of search (i.e., 2- vs. 3-dimensional), searching ability (e.g., visual acuity or maneuverability), and species interactions (e.g., competition) can all shape the exponent of foraging rate \citep{Pawar2012, Kalinkat2015}.
We suggest that how resource acquisition scales with body mass is worth much further empirical investigation because of its potential to generate different qualitative patterns in performance curves.

% E: Maybe swap the prev and next paragraphs?  First detailed model results, then more general need for more info.  Maybe it doesn't matter.

A non-intuitive and interesting result is the role of resource quality when foraging time is limited.
In general, we found that performance increases with body size, but for a certain range of resource quality and a highly concave foraging rate, performance peaks at intermediate body size.
If resource quality is too low, not enough energy is acquired and individuals of all sizes have negative net energy gain.
In contrast, if resource quality is high, individuals of all sizes have positive net energy gain but the low mass-specific metabolic cost confers the advantage to large individuals.
If resource quality is within an certain range (see Appendix), resource quality alone (and not quantity) can select for different body sizes.
Optimal body size would then shift to lower values as resource quality decreases even if everything else remains constant.
These dual conditions might look restrictive, but our analytical results also show that the range of resource quality allowing this outcome increases with temperature.
There are no data to confirm this theoretical finding, but the fact that it is a possibility underscores the idea that one should look at the ecological context in determining species performance \citep{Sears2015}.

% E: Maybe move this paragraph below?  Then all the results directly from your model are together, and this is a bigger-picture conclusion.
Although we focus on individual performance, the results have larger-scale implications.
For instance, Bergmann's rule is the macroecological pattern that animals tend to be larger in colder environments \citep{Bergmann1847, Blackburn1999}.
Several mechanisms have been proposed to generate the body size cline.
From an energetic perspective, large individuals do better in cold environments than smaller ones because they are better at conserving heat (due to lower surface area-to-volume ratio or to lower conductance) or better at resisting starvation.
Although our principal goal is not to explain Bergmann's rule, our model draws attention not only to energetic cost but to also energetic gain.
Our results suggest that efficiency in foraging and resource availability have the potential not only to explain Bergmann's rule but also its inverse, which has been documented for insects \citep{Cushman1993, Loder1997,Blackburn1999}.
Future investigations with our model could explore the optimal body size for a given set of parameter values and thus propose the shape of its dependence on temperature.

% Warm-up

Our work is unique in examining the thermodynamic features of insects to classify quantitative and qualitative patterns of warm-up.
The warm-up process was investigated empirically during the 1970's and 80's for endothermic insects  such as dung beetles, bees, and moths \citep{Heinrich1975, Bartholomew1978, Bartholomew1981}, but to the best of our knowledge no other model describes the entire warm-up phase let alone the effect of body size.
In general, our model validates intuition about the effect of surface area-to-body mass ratio in heat absorption and heat retention: the ability to complete warm-up increases with decreasing size for ectotherms and with increasing size for endotherms.
Our model also quantifies these relationships and thus provide a blueprint for empirical testing.
Further, we found an unintuitive intermediate scenario in which warming up ability (in terms of completion and not duration of warm-up) is best attained at intermediate body size.
Although the surface area-to-body mass ratio can benefit small ectotherms, it also acts against them in the presence of wind because convection becomes more effective as the ratio increases.
Thus, even though the intuition is true in the simplest situation, we show here that the addition of the influence of wind, which should be prevalent in natural conditions, can generate a more subtle pattern.

An important model parameter is conductance, which controls heat exchange between the thorax and the surface.
Data about conductance for insects are scare.
\citet{Bartholomew1978} found that even though conductance is controlled by very different layers for dung beetles, moths, and bees, they have similar cooling rates. %the air beneath the elythra for dung beetles and by an insulating pile for moths and bees,.
In spite of those data, the homogeneity of such a quantity would be surprising. % as coloration can make a difference in heat exchange \citep{Forsman2002}.
If warm-up is crucial, we found that depending on the timing of warm-up, the optimal conductance should be higher for endotherms when warm-up occurs in the early morning  to increase the heat absorption, but the optimal conductance should be lower during the rest of the day to increase heat retention.
Low conductance  can be a problem because insects also need to dissipate heat during activity.
Yet, other studies on bees and beetles revealed that the process of cooling often happens through different mechanisms such as abdominal pumping or evaporative cooling \citep{Heinrich1979, Verdu2012}.
A different possibility that we have not explored here is whether conductance changes with body size.
However at this stage, we believe that the need for more empirical work supersedes the need to include additional modeling assumptions. % E: nice dodge!

A central question is whether these thermodynamic features are actually important in real systems.
Large endothermic insects have been reported to have the ability to thermoregulate.
Studies on endothermic dung beetles have shown that below a certain mass (about 2 g), individuals become thermoconformers, i.e., not capable of endogeneous thermoregulation \citep{Bartholomew1978, Verdu2006}.
This tipping point can actually occur when endogenous heat production, which increases with body size, equals dissipation of that heat, which decreases with surface area-to-body mass ratio.
The completion of warm-up becomes independent of temperature when an individual reaches a certain size (dashed line with low convection \cref{fig4}c).
The same ability to warm up is known to allow large individuals to forage during colder periods of the day \citep{May1985}.
Furthermore, we found that the inverse relationship occurs for ectothermic insects such that smaller individuals can forage at lower temperature.
We are not aware of any dataset that documents foraging time or temperature as a function of body size for ectotherms and endotherms. % E: or "compares" or whatever, but "correlates" would need to be more specific about which two variables
We hope that our model predictions can eventually be tested both in controlled and natural conditions.
A last example where the process of warm-up can play a significant role is in regulating diel activity. % E: "last example" would make more sense if "first example" or similar were used above
For instance, precise temporal partitioning of activity has been shown in a community of dung beetles \citep{Halffter1966, Caveney1995}.
The generally assumed mechanism behind temporal partitioning is to reduce competition.
However, we found a significant difference in the duration of warm-up depending on the hour of the day (especially soon after sunrise) and on body size.
We thus suggest an alternative explanation for diel partitioning: in a competitive community where resources can be depleted quickly \citep{Hanski1991}, optimizing warm-up time to arrive at the source can become crucial.

% E: maybe move Bergmann paragraph to here?

In this study, we compared performance across body sizes by calculating net energy gain.
Clearly, net energy gain is a very simplistic approximation of performance or fitness.
\citet{Kozlowski1996} has pointed out that an energetic definition of fitness is incomplete, and adding size-dependent mortality shifts optimum body size.
The energy currency in hand can be converted to, for instance, fecundity, by using additional power law relationships \citep{Kooijman2009}. % E: what was "assimilation" doing in this sentence? T: I think assimilation can depend on body size too, assimilation is indeed the real energy they can use.
A notable missing component is the growth and development rate, but such questions have been investigated heavily \citep{VandH1996, Kozlowski2004,Kooijman2009}.
Another interesting extension would be to explicitly include competition in the model, as well as finding the optimal time to start warm-up and foraging, or to integrate net energy gain for a longer time span.
Using energy as a core currency for understanding species performance is appealing, and many studies have embraced this approach.
Energy budget models are now used to predict future species distributions \citep[e.g.,][]{Buckley2008}, but the extreme detail and large number of parameters in DEB models \citep{Kooijman2009} prevents wide application.
At the other end of the spectrum, there are parameter-poor models that can generate general insight, but their interpretation and application are hindered because the meanings of the parameters are unclear \citep[e.g.,][]{Brown1993}.
Our model is situated in the middle of this so-called tactical-strategic spectrum \citet{Holling1966}.
Our main goal is to use parameters that are measurable with clear biological meaning yet not too specific to allow us to get general insight when integrating the role of physiology, ecology, and behavior into one framework.


In summary, we attempted to understand how body size and temperature shape performance by developing and analyzing a mathematical model.
We found that there is no single theoretically-expected relationship of how thermal performance changes with body size.
Niche breadth can increase, decrease, or shift depending on the parameters for metabolic rate, foraging rate, thermoregulation, and resource availability.
We have illustrated here how the model can be used to verify verbal arguments such as the relationship between body size and warm-up behavior, and also to reveal patterns that arise beyond simple intuition such the importance of resource quality and size-specific foraging rate in determining optimal body size.
However, the major contribution of this model is the ability to extend feedback between theory and data.
We hope this work is helpful in highlighting potentially important parameters to measure, and also by providing a clear theoretical relationship among the variables that will guide future empirical work.

%%%%%%%%%%%%%%%%%%%%%%%%%%%%%%%%%%%%%%%%%%%%%%%%%%%%%%%%%%%%%%%%

%Finally, heat exchange via conductance can even be selected for, some studies have found that color can influence thermoregulation behavior for grasshoppers (Forsman)
%It is undeniable that a good thermoregulation capacity can broaden the niche \citep{May1985} but it is unsure about the specific role of warm-up.
%Coevolution of color pattern and thermoregulatory behavior in polymorphic pygmy grasshoppers Tetrix undulata (Forsman A, Ringblom K, Civantos E, Ahnesjö J.)

%Several empirical studies in the past have emphasized the matching between thermoregulation capacity and thermal niche (refs).
%For dung beetles, it has been proposed difference in thermoregulation can facilitate the coexistence of sympatric  species by niche differentiation .e.g in diel activity (refs)

%Diel activity relates to taxon, diet, color, body size and functional group (Vulinec 2002, Krell-Westerwalbesloh et 2004, Feer and Pincebourde 2005)
%Verdu 2006 on partitioning, less than 2 grams thermoconformers so have to flight during warm time.
%However, it is not known how species use their thermoregulatory capacity or how species  behave according to the thermal quality of the habitat,  particularly in the case of the endothermic insects. In burying beetles, body size and some morphological features (such as wing loading and insulation) affect their thermoregulation
%pattern and activity times (Merrick and Smith,2004).
%In some stingless bees, thermoregulatory differences related to body size and coloration suggests niche differentiation and different biogeographic  distributions at the interspecific level (Pereboom & Biesmeijer, 2003 ).
%
%Thus, thermoregulation may broaden a thermal niche in both space and time ( May, 1985 ).
%
%The dung beetles of a given community tend to be quite exact in the timing of their daily activities ( Halffter & Matthews, 1966; Fincher et al. , 1971; Mena et al. , 1989; Caveney et al. , 1995 ).
%
%how species behave according to the thermal quality of the habitat, particularly in the case of the endothermic insects (Verdu2007)

%%%%%%%%%%%%%%%%%%%%%%%%%%%%%%%%%%%%%%%%%%%%
%
%Evaluating performance has been done in the past.
%There is a bunch of literature that is about growth  (e.g. Kozlowski2004, Vanbertanlafy, van der have).
%Other models focus on performance as a function of body size (e.g. Yodzis, Brown).
%The same assumption is that performance is defined as the difference between input and output.
%We use net energy gain here as a first proxy of performance.
%We do not mean it as fitness, as Kozlowski states, how mortality scales can change the optimum.
%%More resource can actually decrease mortality...
%A way to convert resource to offspring for instance if fecundity is linear
%Finally a key aspect that is missing is humidity.
%This is a simplifying version of DEB but they are necessary to illustrate the effect of the other factors.
%Competition to reduce resource availability, the only thing to do is to convert foraging rate so that it depends on the set of competitors.
%
%
%
%notes:
%Niche in the discussion
%In this work we wanted understand how thermal performance varies with body size.
%We wanted to answer questions like does thermal niche breadth increase or decrease with body size, does optimal temperature increase or decrease with body size.
%We focus on how the allometry of physiological processes (metabolism), ecological process (foraging), and behavioral thermoregulation (warm-up) shapes thermal performance.
%Unlike many energy budget models \citep[e.g.,][]{Brown1993,Kooijmann2009}, this is a process-oriented rather than pattern oriented modeling.
%We strive to find a balance between realism and abstraction that makes to model fully parameterizable and flexible enough to generate general insight.
%We are in the middle of what \citet{Hooling1966} called tactical-strategic spectrum (ranging from the need to perform detailed measurements and generally defined parameters that cannot be measured).
%%T: I just like the term tactical-strategic. It can be left off.
%
%CHOWN 2007 Scaling of insect metabolic rate is inconsistent with the nutrient supply network model
%KINGSOLVER 2016	Beyond Thermal Performance Curves: Modeling Time-Dependent Effects of Thermal Stress on Ectotherm Growth Rates
%
%What are the roles of metabolism in defining performance?
%A classical results is that resting metabolic rate scales with body size with a power 3/4  \citep{Kleiber1947, Peters1986, Brown2004}.
%It has been a source of debate for instance whether it is 2/3 or 3/4 \citep[for refs][]{Yodzis1992, Isaac2010}.
%Other theoretical models tried to explain the 3/4 power \citep[e.g.,][]{West1997, Kozlowski1997}.
%Although they are interesting on their own, when the goal is to define performance the picture is incomplete without foraging.
%
%What is the role of foraging?
%Scaling of foraging rate is a determinant factor when comparing performance across body size.
%Theoretical works often assume that the exponent is also 0.75 \citep{Yodzis1992, Brown1993}.
%Such value is not often supported empirically .
%For instance, \citet{Maino2015} found that at the interspecific level, the slope can exceed 3/4 (they look at consumption rate).
%\citet{Pawar2012} did a meta-analysis and found that the exponent depends on the dimension on the search space and can reach up to 1.08.
%Low exponent, say velocity of locomotion can be low (around 0.20)  \citep{Peters1986}.
%Say search time, speed, visual acuity all can shape encounter and consumption rate \citep[ref in][]{Kalinkat2015}.
%\citet{Yodzis1992} mentioned that the ecology context can define the exponent rate without specifying the dependence on body size.
%We explored here the consequences of these different exponents within a range 0.5 and 1.25.
%The concavity or convexity of that function generally determines where performance increases monotonically (i.e. large is always better) or not (intermediate body size is better).
%It is obvious that the smallest (at the limit of body size equals zero) cannot be optimal due to physical and physiological limit.
%Another aspect is that the exponent also influences total foraging time which in turn influences the metabolic cost of activities
%Comparing performance would first require the need to clarify the parameter values within the system under study.
%
%Whereas endogenous variables are important, external environment is also crucial.
%Resources are generally limited.
%When resources are limited, large which has an higher absolute metabolic cost thus have lower net energy gain, and also narrower thermal breadth.
%An interesting result is when foraging is limited in time.
%In general, performance increases with body size but for a certain parameter space intermediate body size is favored.
%Having a concave exponent for foraging rate is not enough (i.e. large are less efficient at gathering resource per unit of mass).
%Resource quality needs to be within a certain range.
%In fact, if resource quality is too low, there is not enough energetic gain and everyone suffers (we did not evaluate negative net energy gain), whereas if it is too high, the large are not penalized enough.
%%Small dung beetles are better at extracting higher nutrient from a dung source than large ones.
%Resource quality alone and not quantity can select for different body sizes even if everything else remain the same.
%Optimal body size would shift to lower values are resource quality decreases.
%Such selection is also more likely in warmer environment.
%Comparing performance is incomplete and one does not consider a better description of the environment (resource-wise) \citet{Sears2015}.
%
%Most energetic budget models (process-oriented) do not include ignores behavioral aspect.
%A novel approach we have is to include the effect of body size on thermoregulation and thermoregulation on performance.
%We developed and analyzed a thermodynamic model that determine the warm-up process.
%Such process has been analyzed empirically during the 80's  for endotherm  such dung beetles, bee, moth \citep{Heinrich1975, Bartholomew1978, Bartholomew1981}.
%A key parameter is conductance between the body and the thorax as it determines the heat exchange between the individual and the environment.
%Much studies on thermoregulation have been done for large ecotherms (lizard) (refs).
%80's studies actually investigated the heat exchange for endothermic insects.
%Although conductance is controlled by the air beneath the elythra for dung beetles and by an insulating pile for moths and bees, cooling rates are similar \citep{Bartholomew1978}.
%It would be interesting to know  how much conductance differ between ecotherm and endotherm.
%Color...
%%Two moths (income and capital breeding) with different warm-up style (whir vs quiver wings) are similar allometrically in energetics of endothermy \citep{Bartholomew1981}.
%Is there any adaptation for small endotherm to lose less heat, or as for mammals big is better because coat becomes thicker?
%
%The duration of warm-up can be essential.
%A dung beetle can take a while up to 40 min to warm-up \citep{Verdu2008}.
%Our  model shows that the timing of warm-up is important and too long warm-up can affect significantly performance.
%In a highly competitive community such as dung beetles reaching the site as soon as possible confers a great advantage, additionally fresh dung is more palatable \citep{Hanski1991}.
%Temporal partitioning has been reported to facilitate coexistence of sympatric species \citep{Verdu2007, Verdu2012}.
%Two species of dung beetles forage at different time because the first is better are retaining heat and thus is active during cold period and the second is better are dissipating heat (the mechanism is to control of heat transfer to the abdomen) \citep{Verdu2012}.
%Body size constraints on warm-up ability because of different scaling of surface and volume.
%As a consequence is a potential mechanisms to temporal partitioning of activity (also combined with conductance above).
%Our model actually showed that different values of conductance are preferred at different time of the day for endotherms.
%Does it happen in reality? % T: I am not sure where to place that topic, here or above
%It would be interesting to know if species that differs in their diel activity also have different conductance (this is relevant event without considering body size)
%More studies on warm-up: how it drives diel activity (and its partitioning), is there a correspondence between the conductance and the timing of foraging i.e. are they smart enough so they can optimize the timing of warm-up?
%Recent reviews emphasize the role of thermoregulation and should be included in evaluation performance \cref{Dial2008, Kalinkat2015}.
%
%
%\textbf{Other things (interesting?):}
%- verbal argument that small dung beetles just wait until food shows up whereas  large do more active searching.
%- Studies have shown that below 2 g, body temperature depends on ambient temperature and above it becomes independent, role of conductance?
%
%Macroecological consequences?
%One reason where variation in thermal performance across body size is to investigate macroecological pattern.
%One of the most known pattern is Bergmann's rule.
%There has been debate whether the rule holds or not.
%There are many examples that found the opposite, it can be common in invertebrates (refs).
%There are various explanations \citep{Chown2010}.
%Our models showed that many mechanisms can underly the pattern.
%1- resource limitation penalizes large ones: temperature plays a role in increasing cost for large.
%2- allometric scaling of foraging can penalize large ones: they don't get enough resource.
%3- reduction in resource quality
%There are ecological factors without the need to call for physiological, if it occurs to be true, theres should be a relationship between the magnitude of decrease in body size and resource availability (can also be due to competition) moderated by the temperature effect.
%%excess of resource and little competition??
%The inverse pattern can also occur if foraging rate is highly convex...
%Use of quantitative model can give insight and verify verbal assumption.
% %Chown Gaston 2010:
% % Large because excess of resource due to low competition (p98 1st col) thus grow faster and get bigger
% %Sawtooth pattern (intraspecific) voltinism
% %restistance hyp (Cushman et al 1993): large resist to starvation and whole section if ref is needed- and counterexample NVDI and wing length in africa
% %hostplant effect (mentioning quality) arctic as good example, those not on angiosperm. Resource limitation (Repasky 1991)
% %Pop dens scales with -0.75  and since R scales  with 0.75 thus equal ``energetic equivalence rule''- no supported Blackburn and ganston 1998
%
%What are we missing?
%We are in the middle of what \citet{Hooling1966} called tactical-strategic spectrum (ranging from the need to perform detailed measurements and generally defined parameters that cannot be measured).
%Level of details?
%What next?
%
%Evaluating performance has been done in the past.
%There is a bunch of literature that is about growth  (e.g. Kozlowski2004, Vanbertanlafy, van der have).
%Other models focus on performance as a function of body size (e.g. Yodzis, Brown).
%The same assumption is that performance is defined as the difference between input and output.
%We use net energy gain here as a first proxy of performance.
%We do not mean it as fitness, as Kozlowski states, how mortality scales can change the optimum.
%%More resource can actually decrease mortality...
%A way to convert resource to offspring for instance if fecundity is linear
%Finally a key aspect that is missing is humidity.
%This is a simplifying version of DEB but they are necessary to illustrate the effect of the other factors.
%Competition to reduce resource availability, the only thing to do is to convert foraging rate so that it depends on the set of competitors.
%
%
%link back to SDM, can't tell what causes absence.
%relates to piece in amarasekare, body size to fec
%
%Ways to take the model:
%temperature and habitat loss.
%amarasekare again???? Angilletta theoretical work.
%Bergmanns
%What to measure
%Extension.
%
%Conclusion:
%-more mec for niche theory
%- what to think about when comparing performance across body size.
%- multiplicative effect of temperature stronger effect on large
%- consequences of habitat loss on thermal performance: large will suffer most---past extinction? (ref Angiletta).
%- parameter values are unknown and empirical measurements are needed.
%- upper limit defined by physiology, lower limit defined by behavior and thermoregulation.
%Bergmann's rule.
