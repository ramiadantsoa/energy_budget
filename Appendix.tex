\documentclass[11pt]{article}
\usepackage{amnat}

\crefname{ineq}{inequality}{inequalities}
\creflabelformat{ineq}{#2{\upshape(#1)}#3}

\linenumbers{}
\modulolinenumbers[2]

\renewcommand{\thefigure}{A\arabic{figure}}
\renewcommand{\theequation}{A\arabic{equation}}


\title{Appendix: Mathematical definitions, derivations, and sensitivity analyses}
\date{\vspace{-5ex}}

\begin{document}

\maketitle

\section{Zenith angle and time of sunrise}

We use the same equation as in \citet{Campbell2012} to model zenith angle ($\psi$) as a function of the time of the day $t$ and latitude $\phi$.
We have
\begin{equation}  \label{eq:psi}
\cos(\psi) = \sin(\phi) \sin(\delta) + \cos(\phi) \cos(\delta) \cos[15 (t- t_0)],
\end{equation}
where $\delta$ represents solar declination which varies between 23.45$^\circ$ and  -23.45$^\circ$ between summer and winter solstice, respectively ($\delta$ is given by equation 11.2 in \citet{Campbell2012} and only depends of the day of the year).
The time of solar noon ($t_0$) varies with longitude but since we are interested in latitudinal variation, we fix $t_0 = 12$.

At any given latitude $\phi$ and day of the year, the time of sunrise ($t_\text{rise}$) is the solution of \cref{eq:trise} in the time interval $[0,12]$ when $\psi = 0$
%
\begin{equation} \label{eq:trise}
t_\text{rise} =  \frac{1}{15} \left( t_0 +  \arccos \left[ \frac{\cos(\psi) - \sin(\phi) \sin(\delta)}{\cos(\phi) \cos(\delta)} \right] \right).
\end{equation}

\section{Geometric properties of the body}

For simplicity we assume that the individual is shaped as half of a sphere (\crefp{fig:geo}; this approximation can be valid for insects like beetles).
Considering a sphere is practical as we only have one free variable: the radius $r$.

The volume of half a sphere is $V = \frac{2}{3} \pi r^3$ and the surface area is $A = \pi r^2 + 2 \pi r^ 2 = 3 \pi r^2$ (the first and the second term of the summation represents  respectively the surface at the bottom and the cap).
We assume that the thorax occupies half of the body with volume $V_{th} = V/2$ and surface $A_{th} = \pi r^2 + \pi r^2 = 2 \pi r^2$ (the second term is half of the cap whereas the first is the same as the surface of the bottom just folded upward).

Knowing body mass $z$ and density $d$, we can convert mass into volume $ V = z/d$.
The radius is then $r = \left( \frac{3V}{2 \pi} \right)^{1/3}$.

\section{Derivation of  mathematical conditions for optimal body mass}

We derive sufficient conditions where net energy gain peaks at intermediate body mass if foraging time is limited.
Because we do not consider warm-up time, $\tau_w $ is 0 and  will be omitted in the formulas.
We defined net energy gain by:
\begin{equation} \label{eq:main}
	E_n(z, \tau_f) = E_g(z,\tau_f) - E_d(z, \tau_f).
\end{equation}
The energetic gain is
\[
	E_g(z,\tau_f) = \tau_f e_g(z),
\]
with
\begin{equation} \label{eq:eg}
	e_g(z) = a_3 z^{b_3} \times \rho  = g(z) \times \rho.
\end{equation}
%
The energetic cost is
\begin{equation} \label{eq:ed}
	E_d(z, \tau_f) = \int_0^{t_i} e_b(z, t) dt + \int_{t_i}^{t_i + \tau_f } e_a(z,t) dt + \int_{t_i+\tau_f}^{24} e_b(z, t) dt
\end{equation}
where
\begin{equation} \label{eq:eb}
	e_b(z, t) = a_1 z^{b_1} e^{-E/[k (T_b(t)+ 273.15)]} =  a_1 z^{b_1} \theta_1
\end{equation}
and
\begin{equation} \label{eq:ea}
	e_a(z,t) = a_2 z^{b_2}  e^{-E/[k (\max(T_w(z_{th}), T_e(t))+ 273.15)]} =  a_2 z^{b_2} \theta_2.
\end{equation}

We assume that temperature is constant during the day so that $\theta_1$ and $\theta_2$ do not depend on time.
We want to derive a relationship where \cref{eq:main} becomes a non-monotonic function of body mass $z$.
Two conditions are  necessary: net energy gain should be positive and the derivative of the net energy gain should change sign from positive to negative.
By definition,
\begin{flalign*}
	E_n(z,\tau_f) & = e_g(z) \times \tau_f  - \left( e_a(z) \times \tau_f + e_b(z) \times (24 - \tau_f ) \right) \\
			&  = \rho a_3 z^{b_3} \times \tau_f  - \left( a_2 z^{b_2}  \theta_2 \times \tau_f +  a_1 z^{b_1} \theta_1 \times ( 24 -\tau_f) \right)
\end{flalign*}
After algebraic manipulation, $E_n$ is positive if and only if
\begin{equation}\label[ineq]{ineq:1}
	\rho > c_1 z^ {b_1 - b_3}  \tfrac{24 - \tau_f}{\tau_f}  + c_2  z^ {b_2 - b_3},
\end{equation}
where $c_1 = \dfrac{a_1}{a_3} \theta_1$ and $c_2 = \dfrac{a_2}{a_3} \theta_2$.
%
For the second condition, the derivative is
\begin{flalign*}
	\frac{d}{dz} E_n(z,\tau_f) & = \rho a_3  b_3 z^{b_3 - 1} \times \tau_f  - \left( a_2 b_2 z^{b_2 -1 }  \theta_2 \times \tau_f +  a_1  b_1 z^{b_1- 1} \theta_1 \times ( 24 -\tau_f) \right),
\end{flalign*}
and is negative if and only if
\begin{equation}\label[ineq]{ineq:2}
	\rho < c'_1 z^ {b_1 - b_3}  \tfrac{24 - \tau_f}{\tau_f}  + c'_2  z^ {b_2 - b_3},
\end{equation}
where $c'_1 = \dfrac{a_1 b_1}{a_3 b_3} \theta_1$ and $c'_2 = \dfrac{a_2 b_2}{a_3 b_3} \theta_2$.
Thus a non-monotonic net energy gain occurs only if
\begin{equation}\label[ineq]{ineq:3}
  c_1 z^ {b_1 - b_3}  \tfrac{24 - \tau_f}{\tau_f}  + c_2  z^ {b_2 - b_3} < \rho < c'_1 z^ {b_1 - b_3}  \tfrac{24 - \tau_f}{\tau_f}  + c'_2  z^ {b_2 - b_3}
 \end{equation}

The difference between the first and last term are only in the coefficients $(c_1, c_2)$ and $(c'_1, c'_2)$.
Thus a necessary condition for \cref{ineq:2} given \cref{ineq:1} is
\begin{equation}\label[ineq]{ineq:cond}
	c'_1 > c_1  \textnormal{ or } c'_2 > c_2.
\end{equation}
In fact, we have  $c'_1 = \dfrac{ b_1}{ b_3} c_1$ and $c'_2 = \dfrac{ b_2}{ b_3} c_2$.
Thus \cref{ineq:3} is true if $b_3 < b_1$ or $b_3 < b_2$ is true.
If we assume that the exponents for active metabolic and resting metabolic rate are equal $b_2 = b_1$ and thus the condition means that the exponent of foraging rate is smaller than the exponent of metabolic rate.
In addition,  the resource quality $\rho$ should also lie within a certain range.
We illustrate these analytical results with \cref{fig:ana}

\section{Sensitivity analyses}

We show that results from Fig.~3a and Fig.~5 in the main text are generally robust to parameter changes.
We varied independently the parameters that controls convection ($K_2$) (\crefp{fig:min}a--c), conductance $K_1$ (\crefp{fig:min}d--f), and intensity of solar radiation $r_3$ (\cref{fig:min}g--i).
The only qualitative difference is for endotherms when conductance is too high (thick line \crefp{fig:min}f), most of the heat generated endogenously dissipates in the environment leading to minor difference between small and large individuals.

We also relaxed the effect of constant temperature during the day (\crefp{fig:therm}).
The horizontal axis shows temperature from sunrise (coldest), increasing linearly until mid-afternoon (midpoint between noon and sunset), and then decreasing.
The hottest temperature is defined as $10 ^{\circ}\rm{C}$ hotter than the temperature at sunrise.
Varying environmental temperature does not change the qualitative results (\crefp{fig:therm} vs. fig.~5 in the main text).
For all the figures, the remaining parameters are the same as in the main text.

\begin{figure}
\begin{center}
	\scalebox{0.5}{\includegraphics{body_vita}}
	\caption{
		\setstretch{\stretchby}
		Geometric representation of the shape of an individual.
		The thorax is in red and the non-thorax in gray.
	}
	\label{fig:geo}
\end{center}
\end{figure}

\begin{figure}
\begin{center}
	\scalebox{0.75}{\includegraphics{appendix_fig}}
	\caption{
		\setstretch{\stretchby}
		a) The conditions in \cref{ineq:3} where $\rho$ needs to be within certain range.
		Cyan and red colors represents two environmental temperatures $T_e = 15, \ 25^\circ \rm{C}$, respectively.
	  The upper bound of the shaded area ($Q$) is the right-hand side of \cref{ineq:3}.
	  The lower bound of the shaded area ($dQ$) is the left-hand side of \cref{ineq:3}.
	  The black horizontal lines represent $\rho = 10 \textnormal{ (dashed)}, \ 24 \textnormal{ (thin)}, \ 45 \textnormal{ (thick)}$.
	  b) Net energy gain peaks at intermediate body mass when $\rho$ is within a certain range based on the left panel.
	  Remaining parameters: $\tau_f = 0.75, \ a_1 = 1., \ a_2 = 20 a_1, \ a_3 = 1, \ b_1 = b_2 = 0.75, \ a_3 = 0.5$.
	}
	\label{fig:ana}
\end{center}
\end{figure}

\begin{figure}
    \includegraphics[width=\textwidth]{figS1} % E: This is the generally-correct way to match the margins.
	\caption{
		\setstretch{\stretchby}
		Lowest temperature that allows the completion of warm-up, as a function of body mass.
		Panels a--c) and d--f) show sensitivity with respect to convection and conductance, respectively.
		Thin lines represent default values (see Table~1 in the main text).
		The default values are then multiplied by 10 (thick lines) or by 0.1 (dashed lines).
		Panels g--i) shows how increasing solar radiation affects the minimum temperature to complete warm-up.
		Solar radiation is 0.5 (dashed lines), 0.75 (thin lines) and 1 (thick lines) the value at 30$^\circ$ latitude during equinox.
		Remaining parameters as in Table~1.
	}
	\label{fig:min}
\end{figure}

\begin{figure}
    \includegraphics[width=\textwidth]{figS2}
		\caption{
			\setstretch{\stretchby}
			Thermal performance curves does not change qualitatively when temperature during the day is constant.
			Results are shown for small (thin lines, $z = 0.5$ in a--d, $z = 0.2$ in e--h) and large (thick lines, $z = 2$) individuals.
			Other assumptions are free convection, warm-up starting half an hour after sunrise, $\rho = 24,\ a_2 = 10, \ a_1 = 1, \ b_1 = b_2 = 0.75 $.
	    For a--d, foraging time is $\tau_f = 1$.
	    For e--h, $b_3 = 0.75$.
	    Units and other parameter values are in Table~1.
		}
	\label{fig:therm}
\end{figure}

\bibliography{refs_energy_budget}
\bibliographystyle{amnat}


\end{document}
